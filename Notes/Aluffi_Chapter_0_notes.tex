### On injections and surjections

#### Injections

Injections (which aren't also surjections) have multiple left-inverses (post-inverses). Eg:

A = ab
B = 123
f : A \to B = a2 b3, injective

g_1 : B \to A = 1a 2a 3b
g_2 : B \to A = 1b 2a 3b

$$g_1 \circ f = g_2 \circ f = id_A$$

It is precisely the free element with no antecedent in $B$ (here, $1$) which leaves room for multiple choices, but doesn't affect the overall inversion process.


#### Surjections

Surjections (which aren't also injections) have multiple right-inverses (pre-inverses), called sections.

B = 123
A = ab
f : B \to A = 1a 2a 3b, surjective

g_1 : A \to B = a1 b3
g_2 : A \to B = a2 b3

$$f \circ g_1 = f \circ g_2 = id_A$$

It is precisely the fact that there are multiple elements that map to the same element (here, $1$ and $2$ to $a$) which leaves room for multiple choices, but doesn't affect the overall inversion process.



#### Cancellations

Function Cancellation Lemma: If a composition of functions cancels out, then the first of the pair is an injection, and the second of the pair is a surjection. Algebraically:
$$
\forall A, B \in Obj(\textbf{Set}),
f \in (A \to B), g \in (B \to A), \;
	g \circ f = id_A
\Rightarrow
	\begin{cases}
		f \text{ is injective} \\
		g \text{ is surjective}
	\end{cases}
$$
Corollary 1: any post-inverse of an injection is a surjection.
Corollary 2: any pre-inverse of a surjection is an injection.

Proof: Let be 
$$A, B \in Obj(\textbf{Set}), f \in (A \to B), g \in (B \to A), \; g \circ f = id_A$$

a) Suppose $f$ is not an injection. This means:
$$\exists x, y \in B, \; x \neq y \text{ and } g(x) = g(y)$$
However, with such an $f$, we have:
$$g(x) = g(y) \Rightarrow f(g(x)) = f(g(y)) = id_A(x) = id_A(y) = x = y$$
This means that $f$ is an injection. Contradiction.
Conclusion: in this context, $f$ must be an injection.

b) Suppose $g$ is not a surjection. This means:
$$\exists a \in A, \; a \notin g(B)$$
Since $g \circ f = id_A$, that means that $g(f(a)) = id_A(a) = a$, this means that $a \in g(B)$. Contradiction.
Conclusion: in this context, $g$ must be a surjection.



### On sections and fibers

Section example with a tangent bundle.

Consider the cylinder $S^1 \times \mathbb{R}$, and the function $f : S^1 \times \mathbb{R} \to S^1$, the projection onto the circle. The cylinder is itself the space in which one can easily represent maps of $(S^1 \to \mathbb{R})$. Each such map corresponds to a section.

Let be 

$$
\begin{align*}
g_1 : S^1    & \longrightarrow  S^1 \times \mathbb{R} \\
      \theta & \longmapsto      (\theta, 1)
\end{align*}
$$


$$
\begin{align*}
g_2 : S^1    & \longrightarrow  S^1 \times \mathbb{R} \\
      \theta & \longmapsto      (\theta, cos(\theta))
\end{align*}
$$

We have
$$f \circ g_1 = f \circ g_2 = id_{S^1}$$

(TODO add diagrams for S1xR, g1 and g2)

A fiber is the preimage of a singleton. In the case of $f$ above, for every $q \in S^1$, $f^{-1}({q})$ is the copy of the real line on the cylinder that passes by $q$.

(TODO add diagram)






### Alternative characterization of a bijection
f is bijective \Leftrightarrow "every element of $B$ has a non-empty fiber" (surjection) and "every fiber is a singleton" (injection)





### On monomorphisms and epimorphisms

#### Failing the mono/epi condition

##### Example of failing the monomorphism definition for a non-injection 

Monomorphism definition:

$$
\text{$f : A \to B$ is a monomorphism}
\; \; \Leftrightarrow \; \; 
\forall Z \in \text{Obj}(\mathcal{C}), \;
\forall g_1, g_2 \in \text{Hom}(Z, A), \;
(f \circ g_1 = f \circ g_2 \Rightarrow g_1 = g_2)
$$

A = ab
B = 123
Z = xy
f : A \to B = 1a 2a 3b, not an injection

g_1 : Z \to A = x1 y3
g_2 : Z \to A = x2 y3

$$f \circ g_1 = f \circ g_2 = \{(x, a), (y, b)\} \in (Z \to B)$$

The multiple choice of elements in $A$ which map to $a$ in $B$ (here, $1$ and $2$) is precisely what allows the overall composition to be equal, even when $g_1 \neq g_2$. This provides insight into a proof of "$f$ is a monomorphism implies that $f$ is an injection". If you suppose that $f$ is a monomorphism and not an injection, you can easily reach a contradiction, since you can use elements like $1$ and $2$ that both map to the same $a$ to construct a counter-example to the implication that defines a monomorphism.


##### Example of failing the epimorphism definition for a non-surjection 

Epimorphism definition:

$$
\text{$f : A \to B$ is an epimorphism}
\; \; \Leftrightarrow \; \; 
\forall Z \in \text{Obj}(\mathcal{C}), \;
\forall g_1, g_2 \in \text{Hom}(B, Z), \;
(g_1 \circ f = g_2 \circ f \Rightarrow g_1 = g_2)
$$

A = ab
B = 123
Z = xy
f : A \to B = a1 b2, not an surjection

g_1 : B \to Z = 1x 2y 3x
g_2 : B \to Z = 1x 2y 3y

$$g_1 \circ f = g_2 \circ f = \{(a, x), (b, y)\} \in (A \to Z)$$

The element $3$ in $B$ not being reached by $f$ is precisely that which provides the opportunity to build $g_1 \neq g_2$ such that they compose to the same result with $f$, since the output of $3$ for them doesn't affect the overall composition. This provides insight into a proof of "$f$ is an epimorphism implies that $f$ is a surjection". If you suppose that $f$ is an epimorphism and not a surjection, you can easily reach a contradiction, since you can use elements like $3$ that are not reached by $f$ to construct a counter-example to the implication that defines an epimorphism.



#### Proofs of mono/inj and epi/surj equivalence

Let $f : A \to B$.

The parts which are "Injection => Monomorphism" and "Surjection => Epimorphism" both use the respective sided inverses to prove the implication.

The other parts use the following tautology to prove the implication by contradiction. "Suppose that $p$ and $\neg q$, show that it leads to a contradiction".

$$
(p \Rightarrow q) \Leftrightarrow
(\neg p \cup q)  \Leftrightarrow
(\neg (p \cap \neg q))
$$

##### Injection => Monomorphism

Suppose that $f$ is an injection. It thus has post-inverses.

$$\exists g \in (B \to A), g \circ f = id_A$$

From there:

$$
\forall Z \in \text{Obj}(\mathcal{C}), \;
\forall a, b \in \text{Hom}(Z, A),
$$
$$
\begin{array}{ccccc} \\
f \circ a = f \circ b & \Rightarrow &  g \circ (f  \circ a) &=&  g \circ (f  \circ b) \\
                      & =           & (g \circ  f) \circ a  &=& (g \circ  f) \circ b  \\
                      & =           &         id_A \circ a  &=&         id_A \circ b  \\
                      & =           &                    a  &=&                    b
\end{array}
$$

We conclude that $f$ is also a monomorphism.


##### Surjection => Epimorphism

Suppose that $f$ is a surjection. It thus has pre-inverses.

$$\exists g \in (B \to A), f \circ g = id_B$$

From there:

$$
\forall Z \in \text{Obj}(\mathcal{C}), \;
\forall a, b \in \text{Hom}(B, Z),
$$
$$
\begin{array}{ccccc} \\
a \circ f = b \circ f & \Rightarrow & (a \circ  f) \circ g  &=& (b \circ  f) \circ g  \\
                      & =           &  a \circ (f  \circ g) &=&  b \circ (f  \circ g) \\
                      & =           &  a \circ  id_B        &=&  b \circ  id_B        \\
                      & =           &  a                    &=&  b
\end{array}
$$

We conclude that $f$ is also an epimorphism.


##### Monomorphism => Injection

Suppose that $f$ is a monomorphism.

$$
\forall Z \in \text{Obj}(\mathcal{C}), \;
\forall g_1, g_2 \in \text{Hom}(Z, A), \;
f \circ g_1 = f \circ g_2 \Rightarrow g_1 = g_2
$$

Suppose now that $f$ is not an injection. Algebraically, this means that:

$$\exists (x, y) \in A^2, \text{ such that } x \neq y \text{ and } f(x) = f(y)$$

We can construct $g_1$ and $g_2$ such that $f \circ g_1 = f \circ g_2$ but $g_1 \neq g_2$, using such a pair $(x, y)$. Thereby, we prove that $f$ is not an monomorphism and arrive at a contradiction.

(If $Z$ is the empty set, being initial in \bold{Set}, there is only 1 map into $A$ (the empty map) and $a = b$ always hold. Therefore, any counterexample to the epimorphism definition needs to have at least 1 element.)

Let $Z = \{a\}$.

$$g_1(a) = x$$
$$g_2(a) = y$$

Clearly, $g_1 \neq g_2$. However, we also have:

$$
f(g_1(a)) = f(x) = f(y) = f(g_2(a)) \Rightarrow
f \circ g_1 = f \circ g_2
$$

This means that $f$ is not a monomorphism: contradiction.

Conclusion: $f$ is an injection.


##### Epimorphism => Surjection

Suppose that $f$ is an epimorphism.

$$
\forall Z \in \text{Obj}(\mathcal{C}), \;
\forall g_1, g_2 \in \text{Hom}(B, Z), \;
g_1 \circ f = g_2 \circ f \Rightarrow g_1 = g_2
$$

Suppose now that $f$ isn't a surjection. Algebraically, this means that:

$$\exists x \in B, x \notin f(A)$$

We can construct $g_1$ and $g_2$ such that $g_1 \circ f = g_2 \circ f$ but $g_1 \neq g_2$, using such an $x \notin f(A)$. Thereby, we prove that $f$ is not an epimorphism and arrive at a contradiction.

(If $Z$ is the singleton set, being terminal in \bold{Set}, there is only 1 map into $Z$ and $a = b$ always hold. Therefore, any counterexample to the epimorphism definition needs to have at least 2 elements. We will however use a 3-element set, since it makes things more intuitive and pedagogical.)

Let $Z = \{a, b, c\}$.

$$
g_1 =
\begin{cases}
	\forall x    \in f(A), g_1(x) = a \\
	\forall x \notin f(A), g_1(x) = b
\end{cases}
$$

$$
g_2 =
\begin{cases}
	\forall x    \in f(A), g_2(x) = a \\
	\forall x \notin f(A), g_2(x) = c
\end{cases}
$$

Clearly, $g_1 \neq g_2$. However, since $A$ is the domain of $f$, of $g_1 \circ f$, and of $g_2 \circ f$, we have:

$$
g_1 \circ f = g_2 \circ f = (x \mapsto a) \in (A \to Z)
$$

This means that $f$ is not an epimorphism: contradiction.

Conclusion: $f$ is a surjection.



### On terminal and initial objects in Set:

If $\empty$ is initial and $\{ \star \}$ is terminal, it is because a function in $Set$ (in categorical terms) must always have an output for every input. Ie, in category theory, all functions are maps ("applications").

Said algebraically:

$$
\forall A, B \in \text{Obj}(\bold{Set}), \;
\forall a \in A, \;
\forall f \in \text{Hom}(A, B), \;
\exists f(a) \in B
$$

In the case of $\empty$ as the input set, and there is only one function $f: \empty \to Z$ for any $Z$: $f$ is the empty mapping. But any $Z \to \empty$ (expept $\empty \to \empty$) contains no mapping (since we'd necessarily be ignoring at least one element of $Z$).

Similarly, in the case of the (unique up-to-isomorphism) singleton set $\{ \star \}$ as the output, you'd have multiple functions (precisely $2^{#Z}$) into it, if you could ignore some elements of the input set. However, if all elements of the input set are required, that leaves you with only one function possible from $Z \to \{ \star \}$: the function which maps all elements to $\star$.




### Restrictions and extensions of functions, and its consequences on a function's nature

8 possibilities to study, based on the following binary dichotomies:
- injection or surjection
- enlarging or restricting
- domain or codomain

Note that enlarging the domain sometimes implies enlarging the codomain, and restricting the codomain sometimes implies restricting the domain.

Legend: Yes, No, Depends

\begin{tabular}{c c c c c}
			& enlarge dom	& restrict dom	& enlarge cod	& restrict cod \\
injection	& D				& Y				& Y				& Y            \\
surjection	& Y				& D				& N				& Y
\end{tabular}


Theorems:

A) if $f \in (A \to B), f \text{ injective (resp. surjective)}$, then $\forall Z \subseteq B, \hat{f} \in ((A \ap f^{-1}(Z)) \to Z), \hat{f} = f|_{f^{-1}(Z)}$, the restriction of the function to the corresponding smaller codomain, is also an injection (resp. surjection).

B) if $f \in (A \to B), f \text{ injective (resp. surjective)}$, then $\forall Z \supseteq B, \hat{f} \in (A \to Z), \hat{f} = \iota \circ f$ (with the $\iota$ the canonical injection of $b \in B$ into its superset $Z$), is also an injection (resp. is never a surjection).

C) if $f \in (A \to B), f \text{ injective}$, then $\forall Z \subseteq A, \hat{f} \in (Z \to B), f = \iota_{(Z \to A)} \circ \hat{f}$, we have that $\hat{f}$ is also an injection. However, one can construct $Z \subseteq A$ such that $f$ stops being a surjection.

D) if $f \in (A \to B), f \text{ surjective}$, then $\forall Z \supseteq A, \hat{f} \in (Z \to (B \cup f(Z))), f = \iota_{(Z \to A)} \circ \hat{f}$, we have that $\hat{f}$ is also a surjection. However, one can construct $Z \subseteq A$ such that $f$ stops being a injection.

Proof: TODO

