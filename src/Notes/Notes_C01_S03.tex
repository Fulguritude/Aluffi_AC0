\chapter*{Chapter 1, Section 3}

\section*{Example summary}

\begin{itemize}
	\item (3.2): Set, category of sets as objects and set functions as morphisms.
	\item (3.3): preorder (or order, or equivalence relation) over a (single) set, transformed into a category; elements of the set as objects, and elements of the preorder (which is a relation, hence a subset of the cartesian product) as morphisms.
	\item (3.4): the powerset with the inclusion operator, transformed into a category; elements of the powerset (i.e., subsets of the set) as objects, and inclusion relations as morphisms (this is just an example of a preorder / order / equivalence category seen in 3.3).
	\item (3.5): slice categories $\mathcal{C}_A$, categories which isolate a specific object $A$ of a given category $\mathcal{C}$, and studies the morphisms into that object; an object of $\mathcal{C}_A$ is any morphism from any arbitrary objet $Z$ into $A$ (not the homset $Hom(Z, A)$ itself !) and a morphism in $\mathcal{C}_A$ (from $z_1 \in Z_1 \to A$ to $z_2 \in Z_2 \to A$) is a "raising" $\sigma_A$ into $\mathcal{C}_A$ of a morphism $\sigma \in Z_1 \to Z_2$ in $\mathcal{C}$ that preserves composition on morphisms in $\mathcal{C}$ (i.e., $z_1 = z_2 \sigma \Rightarrow \sigma_A z_1 = z_2$).
	\item (3.6): combining examples 3.3 and 3.5, first start with an order category on the set $\mathbb{Z}$ (there is a morphism $m \to n$ iff $m \leq n$), then select a specific object (here, $A = 3$) then study all morphisms of the category into $A$ (so the relation $n \leq 3$ for any $Z = n$); the morphisms $\sigma_3 = (m, 3) \to (n, 3)$ are then simply given by the transitivity of $\leq$, i.e., $m \leq n \leq 3$ ($(m, 3) \to (n, 3)$ corresponds to $m \leq 3 \Rightarrow n \leq 3$, meaning our $z_1 = z_2 \sigma$ transforming into $\sigma_A z_1 = z_2$, here, corresponds to $(m \leq 3) = (n \leq 3) \cap (m \leq n)$ is transformed into $(m \leq 3 \Rightarrow n \leq 3) \cap (m \leq 3) \Leftrightarrow (n \leq 3)$).
	\item (3.7): coslice categories (morphisms out of a chosen object).
	\item (3.8): the category \textbf{Set$^\star$} of pointed sets, a coslice category over \textbf{Set} and any singleton set $\{ \star \}$. Objects in \textbf{Set$^\star$} are regular sets, but with a unique distinguished element; morphisms are any set functions that map the domain's distinguised element to the codomain's distinguished element.
	\item (3.9): "bislice" and "bicoslice" categories, basically a similar construct as slice and coslice, but taking two objects of the starting category, and studying pairs of morphisms (from a common domain, resp codomain) into (resp from) this pair.
	\item (3.10): "fibered bislice" and "fibered bicoslice" categories, once again a similair construct, but this time taking two \textif{morphisms} into a common set C (resp. from a common set C).
\end{itemize}



\section*{On terminal and initial objects in \textbf{Set}}

If $\empty$ is initial and $\{ \star \}$ is terminal, it is because a function in $Set$ (in categorical terms) must always have an output for every input. Ie, in category theory, all functions are maps ("applications").

Said algebraically:

$$
\forall A, B \in \text{Obj}(\bold{Set}), \;
\forall a \in A, \;
\forall f \in \text{Hom}(A, B), \;
\exists f(a) \in B
$$

In the case of $\empty$ as the input set, and there is only one function $f: \empty \to Z$ for any $Z$: $f$ is the empty mapping. But any $Z \to \empty$ (expept $\empty \to \empty$) contains no mapping (since we'd necessarily be ignoring at least one element of $Z$).

Similarly, in the case of the (unique up-to-isomorphism) singleton set $\{ \star \}$ as the output, you'd have multiple functions (precisely $2^{|Z|}$) into it, if you could ignore some elements of the input set. However, if all elements of the input set are required, that leaves you with only one function possible from $Z \to \{ \star \}$: the function which maps all elements to $\star$.




\section*{Restrictions and extensions of functions, and its consequences on a function's nature}

8 possibilities to study, based on the following binary dichotomies:
\begin{itemize}
	\item injection or surjection
	\item enlarging or restricting
	\item domain or codomain
\end{itemize}

Note that enlarging the domain sometimes implies enlarging the codomain, and restricting the codomain sometimes implies restricting the domain.

Legend: Yes, No, Depends

\begin{tabular}{c c c c c}
			& enlarge dom	& restrict dom	& enlarge cod	& restrict cod \\
injection	& D				& Y				& Y				& Y            \\
surjection	& Y				& D				& N				& Y
\end{tabular}


Theorems:

A) if $f \in (A \to B), f \text{ injective (resp. surjective)}$, then $\forall Z \subseteq B, \hat{f} \in ((f^{-1}(Z) \subseteq A) \to Z), \hat{f} = f|_{f^{-1}(Z)}$, the restriction of the function to the corresponding smaller codomain, is also an injection (resp. surjection).

B) if $f \in (A \to B), f \text{ injective (resp. surjective)}$, then $\forall Z \supseteq B, \hat{f} \in (A \to Z), \hat{f} = \iota \circ f$ (with the $\iota$ the canonical injection of $b \in B$ into its superset $Z$), is also an injection (resp. is never a surjection).

C) if $f \in (A \to B), f \text{ injective}$, then $\forall Z \subseteq A, \hat{f} \in (Z \to B), f = \iota_{(Z \to A)} \circ \hat{f}$, we have that $\hat{f}$ is also an injection. However, one can construct $Z \subseteq A$ such that $f$ stops being a surjection.

D) if $f \in (A \to B), f \text{ surjective}$, then $\forall Z \supseteq A, \hat{f} \in (Z \to (B \cup f(Z))), f = \iota_{(Z \to A)} \circ \hat{f}$, we have that $\hat{f}$ is also a surjection. However, one can construct $Z \subseteq A$ such that $f$ stops being a injection.

Proof: TODO

