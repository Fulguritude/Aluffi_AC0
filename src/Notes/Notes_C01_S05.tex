\chapter*{Chapter 1, Section 5}

\section*{Initial and terminal objects}

\begin{itemize}
	\item there are categories without either initial or terminal objects, such as the preorder category of $\mathbb{Z}$ with $\leq$.
	\item there are categories with multiple initial or terminal objects (for example, in \textbf{Set}, every singleton set is a terminal object); however, these are respectively unique up to isomorphism
	\item any object which is both initial and terminal is called a zero object.
\end{itemize}

\section*{Universal properties}

\subsection*{"Normal" universal properties}

Verbatim: "The most natural context in which to introduce universal properties requires a good familiarity with the language of functors, which we will only introduce at a later stage. [...] We say that a construction satisfies a universal property (or: 'is the solution to a universal problem') when it may be viewed as a terminal object of a category."

Then: "The declaration/explanation of a universal property generally follows the pattern 'object X is universal with respect to the following property: for any Y such that..., there exists a unique morphism Y → X such that...'; this explanation hides the definition of an accessory category, and the statement that X is terminal."

This is a complicated way to say: there is some construct to decompose a morphism which is "universal" (always exists) and reduces the rest of the information of the morphism into something "unique" (hence terminal object of some subcategory).


\subsection*{Dual universal properties}

