\chapter*{Chapter 1, Section 2}

\section*{On injections and surjections}

\subsection*{Injections}

Injections (which aren't also surjections) have multiple left-inverses (post-inverses). Eg:

$$A = \{ a, b    \}$$
$$B = \{ 1, 2, 3 \}$$
$$f : A \to B = \{ (a, 2), (b, 3) \} \text{, injective}$$

$$g_1 : B \to A = \{ (1, a), (2, a), (3, b) \}$$
$$g_2 : B \to A = \{ (1, b), (2, a), (3, b) \}$$

$$g_1 \circ f = g_2 \circ f = id_A$$

It is precisely the free element with no antecedent in $B$ (here, $1$) which leaves room for multiple choices, but doesn't affect the overall inversion process.


\subsection*{Surjections}

Surjections (which aren't also injections) have multiple right-inverses (pre-inverses), called sections.



$$B = \{ 1, 2, 3 \}$$
$$A = \{ a, b    \}$$
$$f : B \to A = \{ (1, a), (2, a), (3, b) \} \text{, surjective}$$

$$g_1 : A \to B = \{(a, 1), (b, 3) \}$$
$$g_2 : A \to B = \{(a, 2), (b, 3) \}$$

$$f \circ g_1 = f \circ g_2 = id_A$$

It is precisely the fact that there are multiple elements that map to the same element (here, $1$ and $2$ to $a$) which leaves room for multiple choices, but doesn't affect the overall inversion process.



\subsection*{Cancellations}

Function Cancellation Lemma: If a composition of functions cancels out, then the first of the pair is an injection, and the second of the pair is a surjection. Algebraically:
$$
\forall A, B \in Obj(\textbf{Set}),
f \in (A \to B), g \in (B \to A), \;
	g \circ f = id_A
\Rightarrow
	\begin{cases}
		f \text{ is injective} \\
		g \text{ is surjective}
	\end{cases}
$$

Corollary 1: any post-inverse of an injection is a surjection.

Corollary 2: any pre-inverse of a surjection is an injection.

Proof: Let be 
$$A, B \in Obj(\textbf{Set}), f \in (A \to B), g \in (B \to A), \; g \circ f = id_A$$

a) Suppose $f$ is not an injection. This means:
$$\exists x, y \in B, \; x \neq y \text{ and } g(x) = g(y)$$
However, with such an $f$, we have:
$$g(x) = g(y) \Rightarrow f(g(x)) = f(g(y)) = id_A(x) = id_A(y) = x = y$$
This means that $f$ is an injection. Contradiction.

Conclusion: in this context, $f$ must be an injection.

b) Suppose $g$ is not a surjection. This means:
$$\exists a \in A, \; a \notin g(B)$$
Since $g \circ f = id_A$, that means that $g(f(a)) = id_A(a) = a$, this means that $a \in g(B)$. Contradiction.

Conclusion: in this context, $g$ must be a surjection.



\section*{On sections and fibers}

Section example with a tangent bundle.

Consider the cylinder $S^1 \times \mathbb{R}$, and the function $f : S^1 \times \mathbb{R} \to S^1$, the projection onto the circle. The cylinder is itself the space in which one can easily represent maps of $(S^1 \to \mathbb{R})$. Each such map corresponds to a section.

Let be 

$$
\begin{aligned}
g_1 : S^1    & \longrightarrow  S^1 \times \mathbb{R} \\
      \theta & \longmapsto      (\theta, 1)
\end{aligned}
$$


$$
\begin{aligned}
g_2 : S^1    & \longrightarrow  S^1 \times \mathbb{R} \\
      \theta & \longmapsto      (\theta, cos(\theta))
\end{aligned}
$$

We have
$$f \circ g_1 = f \circ g_2 = id_{S^1}$$

(TODO add diagrams for S1xR, g1 and g2)

A fiber is the preimage of a singleton. In the case of $f$ above, for every $q \in S^1$, $f^{-1}({q})$ is the copy of the real line on the cylinder that passes by $q$.

(TODO add diagram)






\section*{Alternative characterization of a bijection}
"$f$ is bijective" $\Leftrightarrow$ ("every element of $B$ has a non-empty fiber" (surjection) and "every fiber is a singleton" (injection))





\section*{On monomorphisms and epimorphisms}

\subsection*{Failing the mono/epi condition}


\subsubsection*{Example of failing the monomorphism definition for a non-injection}

Monomorphism definition:

$$
\text{$f : A \to B$ is a monomorphism}
\\ \Leftrightarrow \\
\forall Z \in \text{Obj}(\mathcal{C}), \;
\forall g_1, g_2 \in \text{Hom}(Z, A), \;
(f \circ g_1 = f \circ g_2 \Rightarrow g_1 = g_2)
$$



$$A = \{ a, b, c \}$$
$$B = \{ 1, 2    \}$$
$$Z = \{ x, y    \}$$
$$f : A \to B = \{ (a, 1), (b, 1), (c, 2) \} \text{, not injective}$$

$$g_1 : Z \to A = \{ (x, a), (y, c) \}$$
$$g_2 : Z \to A = \{ (x, b), (y, c) \}$$

$$f \circ g_1 = f \circ g_2 = \{(x, 1), (y, 2)\} \in (Z \to B)$$

The multiple choice of elements (here, $a$ and $b$) in $A$ which map to $1$ in $B$ is precisely what allows the overall composition to be equal, even when $g_1 \neq g_2$. This provides insight into a proof of "$f$ is a monomorphism implies that $f$ is an injection". If you suppose that $f$ is a monomorphism and not an injection, you can easily reach a contradiction, since you can use elements like $1$ and $2$ that both map to the same $a$ to construct a counter-example to the implication that defines a monomorphism.



\subsubsection*{Example of failing the epimorphism definition for a non-surjection}

Epimorphism definition:

$$
\text{$f : A \to B$ is an epimorphism}
\\ \Leftrightarrow \\ 
\forall Z \in \text{Obj}(\mathcal{C}), \;
\forall g_1, g_2 \in \text{Hom}(B, Z), \;
(g_1 \circ f = g_2 \circ f \Rightarrow g_1 = g_2)
$$

$$A = \{ a, b    \}$$
$$B = \{ 1, 2, 3 \}$$
$$Z = \{ x, y    \}$$
$$f : A \to B = \{ (a, 1), (b, 2) \} \text{, not surjective}$$

$$g_1 : B \to Z = \{ (1, x), (2, y), (3, x) \}$$
$$g_2 : B \to Z = \{ (1, x), (2, y), (3, y) \}$$

$$g_1 \circ f = g_2 \circ f = \{(a, x), (b, y)\} \in (A \to Z)$$

The element $3$ in $B$ not being reached by $f$ is precisely that which provides the opportunity to build $g_1 \neq g_2$ such that they compose to the same result with $f$, since the output of $3$ for them doesn't affect the overall composition. This provides insight into a proof of "$f$ is an epimorphism implies that $f$ is a surjection". If you suppose that $f$ is an epimorphism and not a surjection, you can easily reach a contradiction, since you can use elements like $3$ that are not reached by $f$ to construct a counter-example to the implication that defines an epimorphism.



\subsection*{Proofs of mono/inj and epi/surj equivalence}

Let $f : A \to B$.

The parts which are "Injection => Monomorphism" and "Surjection => Epimorphism" both use the respective sided inverses to prove the implication.

The other parts use the following tautology to prove the implication by contradiction. "Suppose that $p$ and $\neg q$, show that it leads to a contradiction".

$$
(p \Rightarrow q)
\Leftrightarrow ((\neg  p) \cup      q )
\Leftrightarrow ( \neg (p  \cap \neg q))
$$


\subsubsection*{Injection => Monomorphism}

Suppose that $f$ is an injection. It thus has post-inverses.

$$\exists g \in (B \to A), g \circ f = id_A$$

From there:

$$
\forall Z \in \text{Obj}(\mathcal{C}), \;
\forall a, b \in \text{Hom}(Z, A),
$$
$$
\begin{array}{ccccc} \\
f \circ a = f \circ b & \Rightarrow &  g \circ (f  \circ a) &=&  g \circ (f  \circ b) \\
                      & =           & (g \circ  f) \circ a  &=& (g \circ  f) \circ b  \\
                      & =           &         id_A \circ a  &=&         id_A \circ b  \\
                      & =           &                    a  &=&                    b
\end{array}
$$

We conclude that $f$ is also a monomorphism.


\subsubsection*{Surjection => Epimorphism}

Suppose that $f$ is a surjection. It thus has pre-inverses.

$$\exists g \in (B \to A), f \circ g = id_B$$

From there:

$$
\forall Z \in \text{Obj}(\mathcal{C}), \;
\forall a, b \in \text{Hom}(B, Z),
$$
$$
\begin{array}{ccccc} \\
a \circ f = b \circ f & \Rightarrow & (a \circ  f) \circ g  &=& (b \circ  f) \circ g  \\
                      & =           &  a \circ (f  \circ g) &=&  b \circ (f  \circ g) \\
                      & =           &  a \circ  id_B        &=&  b \circ  id_B        \\
                      & =           &  a                    &=&  b
\end{array}
$$

We conclude that $f$ is also an epimorphism.


\subsubsection*{Monomorphism => Injection}

Suppose that $f$ is a monomorphism.

$$
\forall Z \in \text{Obj}(\mathcal{C}), \;
\forall g_1, g_2 \in \text{Hom}(Z, A), \;
f \circ g_1 = f \circ g_2 \Rightarrow g_1 = g_2
$$

Suppose now that $f$ is not an injection. Algebraically, this means that:

$$\exists (x, y) \in A^2, \text{ such that } x \neq y \text{ and } f(x) = f(y)$$

We can construct $g_1$ and $g_2$ such that $f \circ g_1 = f \circ g_2$ but $g_1 \neq g_2$, using such a pair $(x, y)$. Thereby, we prove that $f$ is not an monomorphism and arrive at a contradiction.

(If $Z$ is the empty set, being initial in $\mathbf{Set}$, there is only 1 map into $A$ (the empty map) and $a = b$ always hold. Therefore, any counterexample to the epimorphism definition needs to have at least 1 element.)

Let $Z = \{a\}$.

$$g_1(a) = x$$
$$g_2(a) = y$$

Clearly, $g_1 \neq g_2$. However, we also have:

$$
f(g_1(a)) = f(x) = f(y) = f(g_2(a)) \Rightarrow
f \circ g_1 = f \circ g_2
$$

This means that $f$ is not a monomorphism: contradiction.

Conclusion: $f$ is an injection.


\subsubsection*{Epimorphism => Surjection}

Suppose that $f$ is an epimorphism.

$$
\forall Z \in \text{Obj}(\mathcal{C}), \;
\forall g_1, g_2 \in \text{Hom}(B, Z), \;
g_1 \circ f = g_2 \circ f \Rightarrow g_1 = g_2
$$

Suppose now that $f$ isn't a surjection. Algebraically, this means that:

$$\exists x \in B, x \notin f(A)$$

We can construct $g_1$ and $g_2$ such that $g_1 \circ f = g_2 \circ f$ but $g_1 \neq g_2$, using such an $x \notin f(A)$. Thereby, we prove that $f$ is not an epimorphism and arrive at a contradiction.

(If $Z$ is the singleton set, being terminal in $\mathbf{Set}$, there is only 1 map into $Z$ and $a = b$ always hold. Therefore, any counterexample to the epimorphism definition needs to have at least 2 elements. We will however use a 3-element set, since it makes things more intuitive and pedagogical.)

Let $Z = \{a, b, c\}$.

$$
g_1 =
\begin{cases}
	\forall x    \in f(A), g_1(x) = a \\
	\forall x \notin f(A), g_1(x) = b
\end{cases}
$$

$$
g_2 =
\begin{cases}
	\forall x    \in f(A), g_2(x) = a \\
	\forall x \notin f(A), g_2(x) = c
\end{cases}
$$

Clearly, $g_1 \neq g_2$. However, since $A$ is the domain of $f$, of $g_1 \circ f$, and of $g_2 \circ f$, we have:

$$
g_1 \circ f = g_2 \circ f = (x \mapsto a) \in (A \to Z)
$$

This means that $f$ is not an epimorphism: contradiction.

Conclusion: $f$ is a surjection.
