\part{Summaries}

Chapter I)

Section 1) Explains fundamentals of set theory and basic set operations

Section 2) Explains set relations, set functions and some more advanced set operations

Section 3) Presents categories, and multiple examples of categories. Some are simple, some are advanced.

Section 4) Presents monomorphisms and epimorphisms in more detail, taking care to distinguish general morphisms from set functions and their accolytes (inj, surj, etc)

Section 5) Presents more advanced concepts from category theory, mostly some important universal properties


Chapter II)



\newpage


\part{Group Weekly Reports}

Week 1 : Today we mostly talked about the first chapter first section's reading; going over the vocabulary term by term (see the lexicon on the github repo), and going more in depth over certain concepts (particularly relating to set relations). We also saw a bit of a "teaser" of how these notions are used. We did not go over the exercises since not everyone had done them.

Week 2 : Today we continued on discussing the first chapter, it was mostly freeform. We mostly talked about foundations of set theory (mostly stemming from the discussion of exercise 1 on russell's paradox), why we use function notation the way we do, and about some of the operators over sets themselves (including through some examples from linear algebra and things like the subobject classifier which is seen at the end of section 3). 

Week 3 : Today we finished discussing the first chapter. We went over all exercises. We mostly spoke about equivalence relations and partitions. We also spoke about the geometry/topology of quotients of sets by equivalence relations. This was naturally related to exercises 1.2 to 1.7.

Week 4 : We went over monomorphisms and epimorphisms in more depth. We corrected exercises 2.1 to 2.3 (included)

Week 5 : We went in depth over the distinction between isomorphisms and bijections (foreshadowing a bunch of category theory while we were at it) and corrected exercises 2.4 and 2.5.

Week 6 : We went in depth over the notion of section. We corrected exercises 2.6 and 2.7. For the latter exercise, we understood Tristan's solution by ourselves ! (written by Amric)

Week 7 : We reviewed the notions of algebraic quotient and well-definition. We broached the notion of universal property. We used this to correct exercises 2.8 and 2.9.

Week 8 : We corrected exercises 2.10 and 2.11. We then did some preliminary explanations to present categories and help with the reading of section 3.

Week 9 : We spoke more in depth about category theory, concrete categories, local smallness, algebraic structures (and their vocabulary) and applied category theory.

Week 10 :
We reviewed examples 3.2, 3.3, 3.4 and gave a bunch of disambiguation ideas for 3.5. Next week we'll go over 3.5 and 3.6 in a bit more detail, and start correcting the exercises for this section. We'll leave 3.7 and above for when we get to their respective exercises

\newpage
