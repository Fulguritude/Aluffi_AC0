\section*{Section 5)}

\subsection*{5.1)}

Prove that a final object in a category $\mathcal{C}$ is initial in the opposite category $\mathcal{C}^{op}$

Let $\mathcal{C}$ be a category. Let $\mathcal{C}^{op}$ be the dual category on $\mathcal{C}$. Let $F$ be a final object in $\mathcal{C}$. This means that for every object $Z$ in $\mathcal{C}$, there is a single morphism from $Z$ to $F$. We will call this morphism $f_Z$ (respectively).

We remind how we defined composition in $\mathcal{C}^{op}$ as $\star$, respecting:
$$
\begin{aligned}
	& \forall f \in Hom_{\mathcal{C}^{op}} (B, A) = Hom_{\mathcal{C}} (A, B), \\
	& \forall g \in Hom_{\mathcal{C}^{op}} (C, B) = Hom_{\mathcal{C}} (B, C), \\
	& \exists h \in Hom_{\mathcal{C}^{op}} (C, A) = Hom_{\mathcal{C}} (A, C), \\
	& f \star g \coloneqq g \circ f = h
\end{aligned}
$$

In this case, we see that $\forall Z \in Obj(\mathcal{C}^{op}) = Obj(\mathcal{C}), f_Z \in Hom_{\mathcal{C}^{op}} (F, Z) = Hom_{\mathcal{C}} (Z, F)$. This implies that the homset $Hom_{\mathcal{C}^{op}} (F, Z)$ contains a single morphism, $f_Z$. This means that $F$ is initial in $\mathcal{C}^{op}$.




\subsection*{5.2)}

Prove that $\emptyset$ is the \textit{unique} initial object in \textbf{Set}.

First we will prove that it is initial, then that it is unique.

Initiality: we take an arbitrary set $Z$ in \textbf{Set}. We wish to study $Hom_{\text{\textbf{Set}}}(\emptyset, Z) = Z^\emptyset$. We recall that functions (in category theory) are defined as "applications" / "mappings" are in traditional set theory (i.e., as a relation between sets where every antecedent in the domain has a singleton image in the codomain; the key point being that "no input has no result when passed through the function"). Let $I$ be an initial element in \textbf{Set}. We write $|I| = n$ and $|Z| = m$. We know that $|Z^I| = |Z|^{|I|} = m^n$. For $I$ to be initial, this is true if and only if $m^n = 1$ for all $m$, and so if and only if $n = 0$. We recall that the empty set is the only set with $|\emptyset| = 0$, therefore $I = \emptyset$.

Now this is already enough to prove unicity, but let us spell it out for pedagogy's sake.

Unicity: We recall that two objects of \textbf{Set} are isomorphic if, and only if, there exists a bijection between them. This is equivalent to saying that two sets have the same cardinal. We once again recall that the empty set is the only set with $|\emptyset| = 0$; there are no bijections relating to the empty set, other than its identity, the unique morphism in $Hom_{\text{\textbf{Set}}}(\emptyset, \emptyset)$ . Using proposition 5.4 (that terminal objects are unique up-to-isomorphism), we finally deduce that $\emptyset$ is the unique initial object in \textbf{Set}.

NB: the unique function in $Z^\emptyset$ is always the empty function.



\subsection*{5.3)}

Prove that final objects are unique up to isomorphism.

Let us suppose we have a category $\mathcal{C}$ with two final objects, $F_1$ and $F_2$.

For every object $A$ of $\mathcal{C}$ there is at least one element in $Hom_{\mathcal{C}} (A, A)$, namely the identity $1_A$. If $F$ is final, then there is a unique morphism $F \to F$, which therefore must be the identity $1_F$.

We assumed that $F_1$ and $F_2$ are both final in $\mathcal{C}$. Since $F_1$ is final, there is a unique morphism $f : F_2 \to F_1$ in $\mathcal{C}$. Since $F_2$ is final, there is a unique morphism $g : F_1 \to F_2$ in $\mathcal{C}$. Consider $fg : F_1 \to F_1$ ; as observed, necessarily the composite $fg = 1_{F_1}$
since $F_1$ is final. By the same token $gf = 1_{F_2}$. Thus $f$ and $g$ are inverses of each other, proving $f$ is an isomorphism. Since there exists an isomorphism between $F_1$ and $F_2$, $F_1 \simeq F_2$.



\subsection*{5.4)}

What are initial and final objects in the category of "pointed sets" (Example 3.8)? Are they unique?

We recall that a pointed set is just a regular set with a special, identified point, and that the category of pointed sets \textbf{Set*} is built upon the same objects as \textbf{Set}, but where each object $A$ in \textbf{Set} is multiplied into $|A|$ copies of itself in \textbf{Set*} (one for each choice of special point; this implies that the empty set is not a part of \textbf{Set*}, since it has no point). Morphisms in \textbf{Set*} are set functions, but with the restriction of mapping the special point in the domain to the special point in the codomain.

Given this information, we will prove that the initial and final objects in \textbf{Set*} are the singleton sets.

Let $(O, o)$ be a singleton set in \textbf{Set*}. Let $o$ be the single element of $O$; it is necessarily also the special point, as there is no other choice. For any codomain $(Z, z_0)$ in \textbf{Set*}, the condition that "special points map to special points" restricts our choice of function to the unique function ${(o, z_0)}$, thus, $O$ is initial. If $O$ had more than one element, there would exist some $Z$ (non-singletons) for which the other element would allow another degree of freedom (and thus $O$ would not be initial).

Similarly, now studying $Z$ as a domain and $O$ as a codomain, we see that that only function from $Z$ to $O$ is (like in \textbf{Set}) the function which maps everything (including $Z$'s special point) to $o$. Thus, $O$ is final. If $O$ had more than one element, there would similarly be many choices for any $Z$ of cardinal $\geq 2$, so long as the special point maps to the special point.

Every singleton pointed set is both initial and final in \textbf{Set*} and is thus a zero object. These are also the only such pointed sets.



\subsection*{5.5)}

What are the final objects in the category considered in §5.3?

The category considered in paragraph 5.3 is the coslice category over some set $A$, written $\mathcal{C}_A$. However, what is presented in this paragraph is some extra structure that arises when considering the statement "The quotient $A/\sim$ is universal with respect to the property of mapping $A$ to a set in such a way that equivalent elements have the same image". We thus give some equivalence relation $\sim$ on $A$ and study the quotient set $A/\sim$ in the general coslice category; to do this, we consider the subcategory $\mathcal{Q}$ of $\mathcal{C}_A$ where only $\varphi$ such that "equivalence is preserved" (i.e., such that $\forall a', a'' \in A, a' \sim a'' \Rightarrow \varphi(a') = \varphi(a'')$).

With:
\begin{itemize}
	\item $s$ the canonical surjection of $A$ onto its quotient $A/\sim$,
	\item $\varphi_1$ (resp. $\varphi_2$) being some arbitrary function from $A$ to some arbitrary $Z_1$ (resp. $Z_2$),
	\item $\sigma$ any function (if it exists) such that $\sigma \varphi_1 = \varphi_2$
	\item $f_1$ (resp. $f_2$) is the (unique!) function such that $s f_1 = \varphi_1$ (resp. $s f_2 = \varphi_2$)
\end{itemize}

The following diagram commutes, and summarizes the situation.

% https://tikzcd.yichuanshen.de/#N4Igdg9gJgpgziAXAbVABwnAlgFyxMJZABgBpiBdUkANwEMAbAVxiRAEEQBfU9TXfIRRkAjFVqMWbdgHoAOnOwBbbrxAZseAkRGkx1es1aIQALQD6I1X02Cd5cYakmLAJm7iYUAObwioADMAJwgVRDIQHAgkXQkjNgRqBjoAIxgGAAV+LSEQIKxvAAscaxBg0JjqKKRXA0ljMssQJNT0rNttE3yikp5AkLCI6sRauOcQBXogtEKsJpa0zOy7EywwbFhmkDSwKCQAWgBmYj6ygaQh6MRDuviTSbpp2fN3U-Kw0eGbsYaFbG8lHQtslFu0BJ08gViqV3pVIldvjs9ogjhEQW1lhC1htWLdxgEXh4uEA
\begin{tikzcd}
A \arrow[d, "s"'] \arrow[rd, "\varphi_1" description, bend right] \arrow[r, "\varphi_2"] & Z_2                      \\
A/\sim \arrow[r, "f_1"'] \arrow[ru, "f_2" description, bend right]                       & Z_1 \arrow[u, "\sigma"']
\end{tikzcd}

Objects in this category $\mathcal{C}_A$ (and \textit{a fortiori} $\mathcal{Q}$) are denoted as $(\varphi, Z)$ and are obtained from what used to be \textit{morphisms} (regular functions) in \textbf{Set}. Morphisms are mappings $\sigma_{\mathcal{Q}} : (\varphi_1, Z_1) \to (\varphi_2, Z_2)$ such that one exists if and only if $\exists \sigma \in (Z_1 \to Z_2), \sigma \varphi_1 = \varphi_2$, and $\forall a', a'' \in A, a' \sim a'' \Rightarrow \varphi(a') = \varphi(a'')$.

Since the textbook also asks whether such a category has initial objects, we will first also answer this and consider all terminal objects.

The initial object of a general coslice category is $(id_A, A)$. This is easily verified by doing $\varphi_1 = id_A$, necessarily $\sigma \varphi_1 = \sigma id_A = \sigma = \varphi_2$, in $\mathcal{C}$. This implies that, for the domain $(id_A, A)$ in $\mathcal{C}_A$ and any codomain $(\phi_2, Z_2)$, there always exists a unique morphism $\sigma_A \in Hom_{C_A}((id_A, A), (\phi_2, Z_2))$ in $\mathcal{C}_A$, corresponding to the (existing and unique) $\sigma = \phi_2$ in $\mathcal{C}$. We also see that this object satisfies the "equivalence preservation" condition, hence it exists in $\mathcal{Q}$, and is also the initial object in $\mathcal{Q}$.

The below are this description first in $\mathcal{C}$, followed by the description in $\mathcal{C}_A$.

\begin{tikzcd}
A \arrow[rd, "\phi_1 = id_A"'] \arrow[r, "\phi_2"] & Z_2                                   \\
                                                   & Z_1 = A \arrow[u, "\sigma = \phi_2"']
\end{tikzcd}

Now in $\mathcal{C}_A$.

\begin{tikzcd}
A \arrow[dd, "\phi_1 = id_A"'] &  & A \arrow[dd, "\phi_2"] \\
{} \arrow[rr, "\sigma_A"]      &  & {}                     \\
Z_1 = A                        &  & Z_2                   
\end{tikzcd}

A general coslice category has a final object $(t, F)$ (or many final objects $(t_i, F_i)$) iff $\mathcal{C}$ has a final object $F$ (or many final objects $F_i$). In that case, any final object $(t, F)$ in $\mathcal{C}_A$ corresponds to the unique morphism from $A$ to $F$ (for any final $F$) in $\mathcal{C}$. Let us verify this.

Let $F$ be final in $\mathcal{C}$, and $t$ be the unique morphism $t \in Hom_{\mathcal{C}} (A, F)$. Let $(\varphi, Z)$ be an arbitrary object of $\mathcal{C}_A$. Let be $\sigma$ such that $\sigma \varphi = t$. We consider the following diagram:

% https://tikzcd.yichuanshen.de/#N4Igdg9gJgpgziAXAbVABwnAlgFyxMJZABgBpiBdUkANwEMAbAVxiRAEEQBfU9TXfIRQBGclVqMWbAGLdeIDNjwEiZYePrNWiEAC1u4mFADm8IqABmAJwgBbJGRA4ISURK1scIanAAWWCy9EYh5LG3tEACZqZ1dqTSkdAB0k7GNbOm8QBjoAIxgGAAV+ZSEQKyxjXy9QkGs7BxiXKPjJbRAU+is0fyyc-KKSwTYKqpqKLiA
\begin{tikzcd}
A \arrow[r, "t"] \arrow[d, "\varphi"'] & F \\
Z \arrow[ru, "\sigma"']                &  
\end{tikzcd}

Since $F$ is final in $\mathcal{C}$, $\sigma$ is unique and always exists. Also, since $\sigma$ is unique and always exist, the choice of $\varphi$ is irrelevant: this same $\sigma$ works for all choices of $\varphi$ for a given arbitrary $Z$. This proves that $\sigma_{\mathcal{C}_A}$ exists and is unique for all $(\varphi, Z)$. Finally, since $\sigma$ works for all choices of $\varphi$, it works for those that satisfy the "equivalence preservation" condition, and so does $t$: this means that $(t, F)$ is indeed a final object in $\mathcal{Q}$.


% TODO the below might useful for a discussion of how one could find the facts about terminal objects used above.
% For an object $(i, I)$ to be initial in $\mathcal{C}_A$, we need it to have a single possible $\sigma_\mathcal{C}_A$ mapping it to every $(\varphi, Z)$. Between any two objects in a coslice category, there is at most one morphism $\sigma_\mathcal{C}_A$ iff the $\varphi_1$ (for the corresponding $\sigma$) is an epimorphism (see C01 S03 notes, "On the morphisms of slice and coslice categories"). There is and at least one morphism iff

% source ncatlab
% In a slice category: 
% If C has an initial object ∅, then C / X has an initial object, given by ⟨∅→X⟩.
% The final object of C / X is id X.
% source ChatGPT based on the above
% In a coslice category:
% If C has a final object 1, then X/C has a final object, given by ⟨X→1⟩.
% The initial object of X/C is idX.



\subsection*{5.6)}

Consider the category corresponding to endowing (as in Example 3.3) the set $\mathbb{Z}^+$ of positive integers with the divisibility relation. Thus there is exactly one morphism $d \to m$ in this category if and only if $d$ divides $m$ without remainder; there is no morphism between $d$ and $m$ otherwise. Show that this category has products and coproducts. What are their 'conventional' names? [§VII.5.1]

Like example 3.3, this is a case of "category made from an order relation over a set", since divisibility is an order relation (reflexive, antisymmetric, transitive).

Let us remind the definition of categorical products and coproducts. We consider some general category $\mathcal{C}$.

An object $A \prod B$ is the product of two objects $A$ and $B$ iff there is a unique morphism $\pi_A$ (resp. $\pi_B$) from $A \prod B$ to $A$ (resp. $B$), and for every $Z$ in $\mathcal{C}$, and for every pair of morphisms $f_A : Z \to A$ and $f_B : Z \to B$, there exists a single morphism $\sigma = f_A \prod f_B$ such that $\pi_A \sigma = f_A$ and $\pi_B \sigma = f_B$. This is summarized in the following commutative diagram.

% https://tikzcd.yichuanshen.de/#N4Igdg9gJgpgziAXAbVABwnAlgFyxMJZARgBoAGAXVJADcBDAGwFcYkQAtEAX1PU1z5CKcqWLU6TVuwCCPPiAzY8BImXE0GLNohAyABAB1DeALbx9AIXn9lQogCYxErdN3XuEmFADm8IqAAZgBOEKZIoiA4EEhkktrsgQD6cjSM9ABGMIwACgIqwiDBWD4AFjg2ICFhETTRSE7xbiDG2D6m9PoAvPrJBsZmFsnWaZnZeXaqulhg2LCV1eGIkfWIAMyaUjpVSR4Kiw11MYhxrtvGaFgpC6FLjasbTeeGl7sgo1m5+fa6xWUVnm4QA
\begin{tikzcd}
  & Z \arrow[ld, "f_A"'] \arrow[d, "\sigma = f_A \prod f_B" description] \arrow[rd, "f_B"] &   \\
A & A \prod B \arrow[l, "\pi_A"] \arrow[r, "\pi_B"']                                       & B
\end{tikzcd}

An object $A \coprod B$ is the coproduct of two objects $A$ and $B$ iff there is a unique morphism $i_A$ (resp. $i_B$) from $A$ (resp. $B$) into $A \coprod B$, and for every $Z$ in $\mathcal{C}$, and for every pair of morphisms $f_A : A \to Z$ and $f_B : B \to Z$, there exists a single $\sigma = f_A \coprod f_B)$ such that $\sigma i_A = f_A$ and $\sigma i_B = f_B$. This is summarized in the following commutative diagram.

% https://tikzcd.yichuanshen.de/#N4Igdg9gJgpgziAXAbVABwnAlgFyxMJZARgBpiBdUkANwEMAbAVxiRAC0QBfU9TXfIRQAGUsKq1GLNgEFuvEBmx4CRMuOr1mrRCBkACADqGAxhDQAnaPoBC8vssFEATGIlbpuu1wkwoAc3giUAAzKwBbJDIQHAgkUUltNhCAfTlqBjoAIxgGAAV+FSEQCyx-AAscexAwiEjEVxi4xASPHRBjbH9wun0AXn1Ug2MzS2tUuwzs3ILHVV0sMGxYatr6gGZqWPjNKXaJ1YiorebGtrYsNMO6pE2mpDO9i5TJkEyc-MKnXVKKqp8uEA
\begin{tikzcd}
A \arrow[rd, "f_A"'] \arrow[r, "i_A"] & A \coprod B \arrow[d, "\sigma = f_A \coprod f_B" description] & B \arrow[ld, "f_B"] \arrow[l, "i_B"'] \\
                                      & Z                                                             &                                      
\end{tikzcd}

We now return to our "divisibility order category". We write its objects as simple integers, and the (if it exists, unique) morphism representing "divisibility of $m$ by $n$" as $(n | m)$. The conventional name of the product for this category is "greatest common divisor" (or "meet"), and of the coproduct, "least common multiple" (or "join").

The following commutative diagrams represent this fact. Take two arbitrary naturals $m$ and $n$. Any number $k$ which divides both $m$ and $n$ also divides their GCD. Likewise, if $k$ is a multiple of both $n$ and $m$, then it is a multiple of their LCM. 

% https://tikzcd.yichuanshen.de/#N4Igdg9gJgpgziAXAbVABwnAlgFyxMJZARgBoAGAXVJADcBDAGwFcYkQBrEAX1PU1z5CKcqWLU6TVu0K9+2PASIAmMRIYs2iEAFsefEBgVCiZcTQ3TtAcQDCAEQAUYUgAIdASh4SYUAObwRKAAZgBOEHqIoiA4EEjEciBhEUjRsUjKicmRZDFxiADMWeGRqnlIRQbZqTTphdyU3EA
\begin{tikzcd}
            & k \arrow[ld] \arrow[rd] \arrow[d] &             \\
n \arrow[r] & {GCD(n, m)}                       & m \arrow[l]
\end{tikzcd}

% https://tikzcd.yichuanshen.de/#N4Igdg9gJgpgziAXAbVABwnAlgFyxMJZARgBpiBdUkANwEMAbAVxiRAGsQBfU9TXfIRQAGUsKq1GLNoR59seAkTLjq9Zq0QgAMgGEAsgAowpAAQBbAJTdeIDAsFEATGInrpW89wkwoAc3giUAAzACcIL0QyEBwIJFEQBjoAIxgGAAV+RSEQUKw-AAscGxDwyJcYuMQEpNSMrMctPMLiuRAwiKQAZmpY+LaOyOi+xCcBsu7eqoratMyHJSb8ou8uIA
\begin{tikzcd}
n \arrow[rd] \arrow[r] & {LCM(n, m)} \arrow[d] & m \arrow[ld] \arrow[l] \\
                       & k                     &                       
\end{tikzcd}



\subsection*{5.7)}

Redo Exercise 2.9 ("Show that if $A \simeq A'$ and $B \simeq B'$ , and further $A \cap B = \emptyset$ and $A' \cap B' = \emptyset$, then $A \cup B \simeq A' \cup B'$. Conclude that the operation $A \coprod B$ (as described in §1.4) is well-defined up to isomorphism") this time using Proposition 5.4. (the unicity up-to-isomorphism of terminal objects).

TODO, fix, since $\text{\textbf{Set}}^{A,B}$ and $\text{\textbf{Set}}^{A',B'}$ need to both be treated.

This is what we are give by the prompt:

\begin{tikzcd}
A \arrow[r, "f_A", two heads, hook] \arrow[d, "i_A"']                                                         & A' \arrow[d, "i_{A'}"]                                                 &                                                &                                &   \\
A \cup B \arrow[r, "f_{AB} ?", two heads, hook]                                                               & A' \cup B'                                                             &                                                &                                &   \\
B \arrow[r, "f_B"', two heads, hook] \arrow[u, "i_B"]                                                         & B' \arrow[u, "i_{B'}"'] 
\end{tikzcd}

with $A \cup B = A \coprod B$ and $A' \cup B' = A' \coprod B'$ since $A \cap B = \emptyset$ and $A' \cap B' = \emptyset$.


We define $\text{\textbf{Set}}^{A,B}$ as the "bicoslice category of $A$ and $B$ over \textbf{Set}". Objects in this category are pairs of morphisms $(f,g)$ from $A$ and $B$, respectively, into sets $Z$. They can be diagrammed as follows.

% https://tikzcd.yichuanshen.de/#N4Igdg9gJgpgziAXAbVABwnAlgFyxMJZABgBpiBdUkANwEMAbAVxiRAEEQBfU9TXfIRQBGUsKq1GLNgC1uvEBmx4CRMgCYJ9Zq0QgAQtwkwoAc3hFQAMwBOEALZIyIHBCSjJOtlfnW7jxHVqV3dqbWk9UyMuIA
\begin{tikzcd}
A \arrow[rd, "f"] &   \\
                  & Z \\
B \arrow[ru, "g"] &  
\end{tikzcd}

Morphisms are defined between objects as

% https://tikzcd.yichuanshen.de/#N4Igdg9gJgpgziAXAbVABwnAlgFyxMJZAVgBoBGAXVJADcBDAGwFcYkQAtAfQCYQBfUuky58hFAAZSE6nSat2AQQFCQGbHgJEpPWQxZtEIAEIrhGsUXIU98w5y7kzakZvHIALNNsGlz9aJaKF66NPoKRqaC5oHuPDZhduz+rpYoAMwJcr5GArIwUADm8ESgAGYAThAAtkjWIDgQSJnZESBljiA0jPQARjCMAAqpQSCMMGU4zpU1SPENTYgt4faFnd19A8MWoxVYhQAWU9HtVbWIZAtIUq2rvF1jm0Mj4iB7h8eqM+deV4g3K3YHT4J2+SAAbDRGkgAOyJHIgAA6iOwhWq9AAesBFKRjPw8vwgA
\begin{tikzcd}
A \arrow[rd, "f_1"]  &     &                                &    & A \arrow[rd, "f_2"]  &     \\
                     & Z_1 & {} \arrow[r, "{\sigma^{A,B}}"] & {} &                      & Z_2 \\
B \arrow[ru, "g_1"'] &     &                                &    & B \arrow[ru, "g_2"'] &    
\end{tikzcd}

such that the following diagram commutes in \textbf{Set}

% https://tikzcd.yichuanshen.de/#N4Igdg9gJgpgziAXAbVABwnAlgFyxMJZABgBpiBdUkANwEMAbAVxiRAEEQBfU9TXfIRRkAjFVqMWbAFoB9Ed14gM2PASIjSY6vWatEIOQCZFfVYKJkj43VIMAhbuJhQA5vCKgAZgCcIAWyQyEBwIJCMdSX0QL1kTagY6ACMYBgAFfjUhEAYYLxxTGL9AxGDQpE0JPTZYhQTk1IzzdQMfLFcACwKeb2KK6nLECKq7EAAdMexXfzoQepT0zIsDXPzC3wCkAGYBsKHI6oNXOLmchsXm7LbO7qUNkp2QvcrEhaaBFpy8goPR44UuBQuEA
\begin{tikzcd}
A \arrow[rd, "f_2"] \arrow[d, "f_1"'] &     \\
Z_1 \arrow[r, "\sigma"]               & Z_2 \\
B \arrow[ru, "g_2"'] \arrow[u, "g_1"] &    
\end{tikzcd}

Let us call $I$ the following object of $\text{\textbf{Set}}^{A,B}$, where $A \coprod B$ is any choice of valid disjoint union of $A$ and $B$:

% https://tikzcd.yichuanshen.de/#N4Igdg9gJgpgziAXAbVABwnAlgFyxMJZABgBpiBdUkANwEMAbAVxiRAEEQBfU9TXfIRQBGUsKq1GLNuwAEAHXkBjCGgBO0WQCFuvEBmx4CRMgCYJ9Zq0QgdXCTCgBzeEVAAzDQFskZEDggkUUkrNiwAfU4eD28kU2oAoOpLaRsInWoGOgAjGAYABX4jIRA1LCcACxxuCi4gA
\begin{tikzcd}
A \arrow[rd, "i_A"]  &             \\
                     & A \coprod B \\
B \arrow[ru, "i_B"'] &            
\end{tikzcd}

By definition of a coproduct, we know that in such a configuration, a morphism $\sigma^{A,B}$ from this object into any other object of $\text{\textbf{Set}}^{A,B}$ exists and is unique, and so is the $\sigma$ on which it is based. This means that $I$ is initial in $\text{\textbf{Set}}^{A,B}$.

TODO fix and explain

We observe that $i_A$, $i_B$, $i_{A'}$ and $i_{B'}$ are all injections, and thus are post-cancellable, by maps $\pi_A$, $\pi_B$, $\pi_{A'}$ and $\pi_{B'}$ (which are surjections).

$f_{A'} = i_{A'} \circ f_A$ and $f_{B'} = i_{B'} \circ f_B$ together define a unique map $f_{AB}$ from $A \coprod B$ to $A' \coprod B'$.

We set arbitrary $g_A : A' \to Z$ and $g_B : B' \to Z$; we have $g_A \circ f
_A = g_A \circ i_{A'} \circ f_A$ and $g_B \circ f_B = g_B \circ i_{B'} \circ f_B$. These define a unique map from $A \coprod B$ to $Z$, which we'll write $g_{AB} \circ f_{AB}$.

% Let $h : A' \coprod B' \to A \coprod B$ be defined as the map which makes $i_A \circ f^{-1}(A) \circ \pi_{A'}$

\begin{tikzcd}
A \arrow[rr, "f_A", two heads, hook] \arrow[dd, "i_A"', shift right] \arrow[rrdd, "f_{A'} = i_{A'} \circ f_A", bend right] \arrow[rrrrdd, "g_A \circ f_A", bend left, shift right]        &  & A' \arrow[dd, "i_{A'}"', shift right] \arrow[rrdd, "g_A", bend left=49]                                 &  &   \\
                                                                                                                                                                                          &  &                                                                                                         &  &   \\
A \cup B \arrow[rr, "f_{AB}", two heads, hook] \arrow[rrrr, "g_{AB} \circ f_{AB}" description, bend right, shift right] \arrow[uu, "\pi_A"', shift right] \arrow[dd, "\pi_B", shift left] &  & A' \cup B' \arrow[rr, "g_{AB}"] \arrow[uu, "\pi_{A'}"', shift right] \arrow[dd, "\pi_{B'}", shift left] &  & Z \\
                                                                                                                                                                                          &  &                                                                                                         &  &   \\
B \arrow[rr, "f_B"', two heads, hook] \arrow[uu, "i_B", shift left] \arrow[rruu, "f_{B'}=i_{B'} \circ f_B"', bend left] \arrow[rrrruu, "g_B \circ f_B"', bend right, shift left]          &  & B' \arrow[uu, "i_{B'}", shift left] \arrow[rruu, "g_B"', bend right=49]                                 &  &  
\end{tikzcd}



\subsection*{5.8)}

Show that in every category $\mathcal{C}$ the products $A \times B$ and $B \times A$ are isomorphic, if they exist. (Hint: observe that they both satisfy the universal property for the product of $A$ and $B$, then use Proposition 5.4.)

Let us first remind the definition of the product of two sets. It is the set made of all pairs of $A$ and $B$ (ordered sequences of two elements, where the first element in the sequence comes from $A$ and the second comes from $B$) It is the structure such that the following diagram commutes, and $(f,g)$ is unique.

% https://tikzcd.yichuanshen.de/#N4Igdg9gJgpgziAXAbVABwnAlgFyxMJZARgBoAGAXVJADcBDAGwFcYkQBBEAX1PU1z5CKMsWp0mrdhwAEAHTl4AtvBkAhHnxAZseAkTIAmcQxZtEIDb366hRcqTE1TUiwC0e4mFADm8IqAAZgBOEEpIZCA4EEgOEmbsaAD6XDSM9ABGMIwACgJ6wiDBWD4AFjiaQaHhiJHRSIbOkubaSRppmdl5tvoWjDCBFdYgIWFIAMw09YhxLi2BlSPVE1MxiI3xriA+IB1Zufl2FsVlQ1qjNZNRa5Fz7AAUgaQ+AJSe3EA
\begin{tikzcd}
                                                        & A                                             \\
Z \arrow[ru, "f"] \arrow[rd, "g"'] \arrow[r, "{(f,g)}"] & A \times B \arrow[u, "p_A"'] \arrow[d, "p_B"] \\
                                                        & B                                            
\end{tikzcd}

Now to extend the diagram to consider both $A \times B$ and $B \times A$.

% https://tikzcd.yichuanshen.de/#N4Igdg9gJgpgziAXAbVABwnAlgFyxMJZARgBoAGAXVJADcBDAGwFcYkQBBEAX1PU1z5CKcqWLU6TVuwBCAAgA6CvAFt4crr37Y8BImXE0GLNohAAtHnxAYdQogCYxE49LMdFyrGrhyZV7UE9FDIHFylTEH9uCRgoAHN4IlAAMwAnCBUkMhAcCCRRSRN2JTQsAH0uGkZ6ACMYRgAFAV1hEEYYFJwAkHTM7Jo8pAAWIwiShTLy-2q6hua7YPbO7q1ejKzEJ1z8xELXSJSQWfqmlvszDq6evs3tocRRorcQeOP2ubPFtrSseIALVbWW5Ie67ADMY2KZgAFClSHJ4gBKd41U4LIJtK5A1IbUGDXY5A7sGHxUgpFEnebnJa-AE49b9RCQnYFKEvNCVVGfDGtdh0wE3PHMgkjdmRTkzD7omlYlY8SjcIA
\begin{tikzcd}
                                                   & A                                                                            &                                                 \\
B \times A \arrow[ru, "\pi_A"] \arrow[rd, "\pi_B"] & Z \arrow[u, "f"] \arrow[d, "g"'] \arrow[r, "{(f, g)}"] \arrow[l, "{(g,f)}"'] & A \times B \arrow[lu, "p_A"'] \arrow[ld, "p_B"] \\
                                                   & B                                                                            &                                                
\end{tikzcd}

We seek to prove that given such a commutative diagram (with a unique $(f,g)$ and $(g,f)$), which we will call $D$, then we have $A \times B \simeq B \times A$.

We define $\mathcal{C}_{A,B}$ as the "bislice category of $A$ and $B$ over $\mathcal{C}$". Objects in this category are pairs of morphisms $(f,g)$ from sets $Z$, into $A$ and $B$, respectively. They can be diagrammed as follows.

% https://tikzcd.yichuanshen.de/#N4Igdg9gJgpgziAXAbVABwnAlgFyxMJZARgBoAGAXVJADcBDAGwFcYkQBBEAX1PU1z5CKMgCZqdJq3YAhHnxAZseAkXKliEhizaIQALR4SYUAObwioAGYAnCAFskomjghJ1kneyvzrdx4jOIK5IZJ7SeqYgNIz0AEYwjAAKAirCIDZYpgAWOEbcQA
\begin{tikzcd}
                                   & A \\
Z \arrow[ru, "f"] \arrow[rd, "g"'] &   \\
                                   & B
\end{tikzcd}

Morphisms are defined between objects as

% https://tikzcd.yichuanshen.de/#N4Igdg9gJgpgziAXAbVABwnAlgFyxMJZARgBoAGAXVJADcBDAGwFcYkQBBEAX1PU1z5CKAKwVqdJq3Zde-bHgJFypYhIYs2iEAC0A+sR58QGBUKIAmVeqlaQR+YKUoAzNZobp2hyYGLhyAAs7pKa7PoWPqZOAWQWNmHaAEJRfuaipPEetuwp3BIwUADm8ESgAGYAThAAtkhWIDgQSCqhXiDlBiA0jPQARjCMAAppziCVWEUAFjg+VbVIYo3NiGRtdp2RPf2DI2ZjE9Ozch3VdYhuy0jB6+wAOnfYRTX0AHrAHKRJ3N0gvQPDUbCP4wcrHYzzc4NJpIABs2USICKXW2AL2MXYjFB4IqZ0WNBhiAA7Aj2sitn8doD9sCsWCeJRuEA
\begin{tikzcd}
                                         & A &                                &    &                                          & A \\
Z_1 \arrow[ru, "f_1"'] \arrow[rd, "g_1"] &   & {} \arrow[r, "{\sigma_{A,B}}"] & {} & Z_2 \arrow[ru, "f_2"'] \arrow[rd, "g_2"] &   \\
                                         & B &                                &    &                                          & B
\end{tikzcd}

such that the following diagram commutes in $\mathcal{C}$:

% https://tikzcd.yichuanshen.de/#N4Igdg9gJgpgziAXAbVABwnAlgFyxMJZARgBoAGAXVJADcBDAGwFcYkQBBEAX1PU1z5CKcqWLU6TVuwBaAfWI8+IDNjwEiZcTQYs2iEPIBMS-mqGbSRibukGAQjwkwoAc3hFQAMwBOEALZIRjQ4EEiiknrsXnImNIz0AEYwjAAKAurCID5YrgAWOKYgvgFIZCCh4TpS+sUKIPFJKenmGgaMMF6FvN5+gYjllYjBkXYgADrj2K7+9A0gCclpGRbtnd3KJf0jQwDM1VEGrrHzi80rbQvrRVtlIWGI+6O1x4qNSy2Clzn53ZTcQA
\begin{tikzcd}
                                                             & A                                      \\
Z_1 \arrow[ru, "f_1"] \arrow[r, "\sigma"] \arrow[rd, "g_1"'] & Z_2 \arrow[u, "f_2"'] \arrow[d, "g_2"] \\
                                                             & B                                     
\end{tikzcd}

We now define the following objects:

% https://tikzcd.yichuanshen.de/#N4Igdg9gJgpgziAXAbVABwnAlgFyxMJZARgBoAGAXVJADcBDAGwFcYkQBBEAX1PU1z5CKcqWLU6TVuw4ACADry8AW3iyAQjz4gM2PASJkATBIYs2iEJu4SYUAObwioAGYAnCMqRkQOCElFJc3Y0AH0uGkZ6ACMYRgAFAX1hEDcsewALHC1XDy9EHz8kIxozaUswzUiYuMS9IXZGGBdsm24gA
\begin{tikzcd}
                                                & A \\
A \times B \arrow[ru, "p_A"'] \arrow[rd, "p_B"] &   \\
                                                & B
\end{tikzcd}

and

% https://tikzcd.yichuanshen.de/#N4Igdg9gJgpgziAXAbVABwnAlgFyxMJZARgBoAGAXVJADcBDAGwFcYkQBBEAX1PU1z5CKcqWLU6TVuwBCAAgA6CvAFt4crr37Y8BImQBMEhizaIQMnhJhQA5vCKgAZgCcIKpGRA4ISUZNN2JTQsAH0uGkZ6ACMYRgAFAV1hEBcsWwALHB4+EFd3TxofJAMaE2lzYLDLSJi4xJ0hdkYYJ2zuSm4gA
\begin{tikzcd}
                                                    & A \\
B \times A \arrow[ru, "\pi_A"'] \arrow[rd, "\pi_B"] &   \\
                                                    & B
\end{tikzcd}

Using the diagram $D$ defined above, and the definition of a product (ie, that the maps from any $Z$ to it in the appropriate configuration are unique). We deduce that both $A \times B$ and $B \times A$ are final objects in $\mathcal{C}_{A,B}$. Finally, using Proposition 5.4, i.e., that final objects in a category are unique up-to-isomorphism, we conclude that $A \times B \simeq B \times A$.



\subsection*{5.9)}

Let $\mathcal{C}$ be a category with products. Find a reasonable candidate for the universal property that the product $A \times B \times C$ of three objects of $\mathcal{C}$ ought to satisfy, and prove that both $(A \times B) \times C$ and $A \times (B \times C)$ satisfy this universal property. Deduce that $(A \times B) \times C$ and $A \times (B \times C)$ are necessarily isomorphic.

Given 3 objects $A, B, C$ of $\mathcal{C}$. We propose the following universal property that $A \times B \times C$ should respect: for all objects $Z$ of $\mathcal{C}$, and triplet of maps $f$, $g$, and $h$ from $Z$ to $A$, $B$ and $C$ respectively, there exists a unique triplet-map $(f,g,h)$ is unique such that the following diagram commutes.

% https://tikzcd.yichuanshen.de/#N4Igdg9gJgpgziAXAbVABwnAlgFyxMJZAJgBpiBdUkANwEMAbAVxiRAEEACAHW7wFt4nAEI8+WQXE4BhEAF9S6TLnyEUZAAxVajFm3bzFIDNjwEiARlIXt9Zq0QhhhpadVEN5W7ochZC1xVzFE8tajs9RwAteW0YKABzeCJQADMAJwh+JE8QHAgkKx17NjQAfQNqBjoAIxgGAAVlMzUQdKwEgAscFxAMrJzqfKQyEGq6xub3RywwbFgQcJ9SsucAvszsxFzhxABmJZLHcv8jfq2AFiGCxCKI31Te86QrvJuDsdr6prdgto7uotipEQJ0npsXtdBsDfAAKVKkBKkToASiB42+Uz+s3mrGodTAUCQAFo9hp1s9EK9dqN7mwEuivpNfq0cVgFnIKHIgA
\begin{tikzcd}
	Z \arrow[rr, "f"] \arrow[dd, "h"'] \arrow[rrdd, "{(f,g,h)}" description, bend right] \arrow[rd, "g" description] &   & A                                                                                      \\
																													 & B &                                                                                        \\
	C                                                                                                                &   & A \times B \times C \arrow[uu, "p_A"'] \arrow[lu, "p_B" description] \arrow[ll, "p_C"]
\end{tikzcd}

We will now show that both $(A \times B) \times C$ and $A \times (B \times C)$ satisfy this universal property.

\subsubsection*{5.9.a)}

We start with

% https://tikzcd.yichuanshen.de/#N4Igdg9gJgpgziAXAbVABwnAlgFyxMJZARgBoAGAXVJADcBDAGwFcYkQBBEAX1PU1z5CKcqWLU6TVuwBaPPiAzY8BImXE0GLNok4ACADoG8AW3h6AQvP7Kha0gCYJW6bqvcJMKAHN4RUABmAE4QJkhkIDgQSKKS2uwBIDSM9ABGMIwACgIqwiCMMAE41iDBoeE0UUgAzJpSOiDeSflpGdm2qrpBWN4AFsW8gSFhiBFViA518boAFAGk3gCUzSnpWTl2ulhg2LAlZSOTkdGIsS4NaAD6XMmt6x153X0DCgdIR+O1ca6Kl1a3a3agk6+UKA0o3CAA
\begin{tikzcd}
                                                                    & A                                             \\
Z \arrow[ru, "f"] \arrow[rd, "g"'] \arrow[r, "{(f,g)}" description] & A \times B \arrow[u, "p_A"'] \arrow[d, "p_B"] \\
                                                                    & B                                            
\end{tikzcd}

and

% https://tikzcd.yichuanshen.de/#N4Igdg9gJgpgziAXAbVABwnAlgFyxMJZAJgBoAGAXVJADcBDAGwFcYkQBBAAgB0e8AtvC4AhEAF9S6TLnyEU5UgEZqdJq3YAtCVJAZseAkTIqaDFm0QgAFNz6DhIgJS9+WIXC4BhHdINzjUmJVcw0rH3FVGCgAc3giUAAzACcIASQlGhwIJABmM3VLEAALEBpGegAjGEYABRlDeRBkrBjinF8QFLSMrJzERTULdmtE0hinMpAK6rqGgKtGGESOySTU9MRMkGykMiGwm1Hxp1JiyfKqmvr-IyssMGxYTu7N-d2BguGrNAB9YDsbg8onEUxm13md2arXaLw2ez6eS+hz+PkusxusihSxWEko4iAA
\begin{tikzcd}
	&  & A \times B                                                          \\
Z \arrow[rrd, "h"'] \arrow[rru, "{(f,g)}"] \arrow[rr, "{((f,g),h)}" description] &  & (A \times B) \times C \arrow[u, "p_{A \times B}"'] \arrow[d, "p_C"] \\
	&  & C                                                                  
\end{tikzcd}

From both of these universal products, we deduce the following commutative diagram.

% https://tikzcd.yichuanshen.de/#N4Igdg9gJgpgziAXAbVABwnAlgFyxMJZABgBoBGAXVJADcBDAGwFcYkQAtEAX1PU1z5CKAMwVqdJq3YBBAAQAdBXgC28OQCEefEBmx4CRMiIkMWbRCADC2-vqFEATKRM0z0ywAp5S1eo0AlIrKWGpwcja8doKGKM7EplIWIDK2ugIGwsjOjonm7FrcEjBQAObwRKAAZgBOECpIZCA4EEjOkvmWABYgNIz0AEYwjAAKGQ6WNVilXThptfWNNC1I5G5J7J5VpKUBvSD9Q6PjsQcwVXNRIAsNiE0riGIdHiCeWzsBpF17fYPDY-ZTlgwNhYPM6rcng81s9kmgAPrAHwhMKabj7Q7-E7CEBTGaXHQ3JBQ1qIdruOHwyKEiFLZqkgAs6061wxf2OgJxwNBbCuRMQMIeTNh7ARqV+RwBMS5IKwYL5tIFy1JAFZmS8EVoJVjOexuXLeTTFndlUg1SLLKU2ZLsXrZfLKNwgA
\begin{tikzcd}
	&  & A                                                                      &                                                                        \\
Z \arrow[dd, "h"'] \arrow[rrr, "{(f,g)}"] \arrow[rrdd, "{((f,g),h)}" description] \arrow[rru, "f" description] \arrow[rrd, "g" description] &  &                                                                        & A \times B \arrow[lu, "p_A" description] \arrow[ld, "p_B" description] \\
	&  & B                                                                      &                                                                        \\
C                                                                                                                                           &  & (A \times B) \times C \arrow[ruu, "p_{A \times B}"'] \arrow[ll, "p_C"] &                                                                       
\end{tikzcd}

\subsubsection*{5.9.b)}

We start with

\begin{tikzcd}
	& B                                             \\
Z \arrow[ru, "g"] \arrow[rd, "h"'] \arrow[r, "{(g,h)}"] & B \times C \arrow[u, "p_B"'] \arrow[d, "p_C"] \\
	& C                                            
\end{tikzcd}

and

% https://tikzcd.yichuanshen.de/#N4Igdg9gJgpgziAXAbVABwnAlgFyxMJZAJgBoAGAXVJADcBDAGwFcYkQBBEAX1PU1z5CKcqQCM1Ok1bsAWjz4gM2PASJkJNBizaJOAAgA6hvAFt4+gBQAhIyazm4+gMIBKBfxVD1pYpO0yerbGZhbOPJIwUADm8ESgAGYAThCmSGI0OBBIAMxa0rogltGkABbuNIz0AEYwjAAKAqrCIElY0aU4HiDJqemZ2YiiUjrsCSCVNXWNXmp6jDAJXbyJKWmIGSBZSGQjgUUJpMVlrhUgVbUNTd56WGDYsN2967vbQ-mjemgA+lyTlzNBHNWu1Ok81jsBrkPvsfsBgvZHC5uBNzlMrrMWgslhFuEA
\begin{tikzcd}
                                                                                 &  & A                                                                   \\
Z \arrow[rrd, "{(g,h)}"'] \arrow[rru, "f"] \arrow[rr, "{(f,(g,h))}" description] &  & A \times (B \times C) \arrow[u, "p_A"'] \arrow[d, "p_{B \times C}"] \\
                                                                                 &  & B \times C                                                         
\end{tikzcd}

From both of these universal products we deduce the following commutative diagram:

% https://tikzcd.yichuanshen.de/#N4Igdg9gJgpgziAXAbVABwnAlgFyxMJZAJgBoAGAXVJADcBDAGwFcYkQAhEAX1PU1z5CKcqQCM1Ok1bsAWjz4gM2PASIBmcZIYs2iTgAIAOkbwBbeAYDCC-iqFEyxbdL0gbvO4LUjS6l7rsAIK2SgKqwiR+ATL6QcamWBZwBgAUHAnmllYAlDySMFAA5vBEoABmAE4QZkhiNDgQSAAsNDqxIOUgNIz0AEYwjAAK4Q76lVhFABY4oVU1dQ1NiGRSgfqpRaRTeT39gyP2PiCMMOWznp3VtYj1II1IAKxtruyp5aSb2zm7J-vDo2OWDA2Fgc2ui3uy1Eaw6RW6fwGAKOwhAwNBbEu8xudweiE0sLcUwRvSRh28qPRWDBWIhKyWSBh7TcaAA+lw9mTAZSQdTMYpsUhVniCcz2GybJyDtz2FSaQK6c8oUKXuslKzgBkTFkUlZuCT-uSIuwJtMLgqFoglXjWoTxayQlLkRT2KdzvluEA
%\begin{tikzcd}
%	&  & B                                                                      &                                                                        \\
%Z \arrow[dd, "f"'] \arrow[rrr, "{(g,h)}"] \arrow[rrdd, "{(f,(g,h))}" description] \arrow[rru, "g" description] \arrow[rrd, "h" description] &  &                                                                        & B \times C \arrow[lu, "p_B" description] \arrow[ld, "p_C" description] \\
%	&  & C                                                                      &                                                                        \\
%A                                                                                                                                           &  & A \times (B \times C) \arrow[ruu, "\pi_{B \times C}"'] \arrow[ll, "\pi_A"] &                                                                       
%\end{tikzcd}

% https://tikzcd.yichuanshen.de/#N4Igdg9gJgpgziAXAbVABwnAlgFyxMJZARgBoAGAXVJADcBDAGwFcYkQAtEAX1PU1z5CKAMwVqdJq3YBBHnxAZseAkXKkATBIYs2iEACF5-ZUKIbN2qXpABhY4oErhyMVpo7p+mQAIAOn54ALbwPgAUBv6BWCFwPrYAlA5KgqooZCJWuuyRAcGh9twSMFAA5vBEoABmAE4QQUjqIDgQSGSS2fpVIDSM9ABGMIwACk5m+jVYpQAWOA619Y00LUgArB7W7GGlpNNJvQNDo6ZpIIwwVXO81XUNiE0riAAsG50gYVWk27sJ+2eHIzGpywYGwsHmtyWzVaiAsHS8IFKPX+g0BJ2EIBBYLY1xACzuDxhYnhNmmyL6qOOqQxWKw4Nx+KQL2ha1eCICaCwAH1gLlorF4txyQCqc52JMZlcFIznssYe1PDYOdy5AdKUCMedLhDFoh1izYWylX5OVyjGqjhr2LT6dLIXq5UhiYr2MqufYLWjqdbQXScZRuEA
\begin{tikzcd}
  & Z \arrow[rr, "f"'] \arrow[ddd, "{(g,h)}"] \arrow[rrdd, "{(f,(g,h))}" description] \arrow[ldd, "g" description] \arrow[rdd, "h" description] &   & A                                                                          \\
  &                                                                                                                                             &   &                                                                            \\
B &                                                                                                                                             & C & A \times (B \times C) \arrow[lld, "\pi_{B \times C}"'] \arrow[uu, "\pi_A"] \\
  & B \times C \arrow[lu, "\pi_B" description] \arrow[ru, "\pi_C" description]                                                                  &   &                                                                           
\end{tikzcd}


\subsubsection*{5.9.c)}

We now consider the $\mathcal{C}_{A, B, C}$ as the "trislice category of $A$, $B$ and $C$ over $\mathcal{C}$". Objects in this category are of the form:

\begin{tikzcd}
    & A \\
Z \arrow[ru, "f"] \arrow[r, "g"'] \arrow[rd, "h"'] & B \\
    & C
\end{tikzcd}

Morphisms are defined between objects as:

\begin{tikzcd}
    & A &                                  &    &                                                            & A \\
Z_1 \arrow[ru, "f_1"] \arrow[r, "g_1"'] \arrow[rd, "h_1"'] & B & {} \arrow[r, "{\sigma_{A,B,C}}"] & {} & Z_2 \arrow[ru, "f_2"] \arrow[r, "g_2"'] \arrow[rd, "h_2"'] & B \\
    & C &                                  &    &                                                            & C
\end{tikzcd}

such that there exists a $\sigma$ making the following diagram commute in $\mathcal{C}$:

% https://tikzcd.yichuanshen.de/#N4Igdg9gJgpgziAXAbVABwnAlgFyxMJZARgBoAGAXVJADcBDAGwFcYkQBBEAX1PU1z5CKcqWLU6TVuwBaAfWI8+IDNjwEiZcTQYs2iEACEl-NUKIAmMRN3SD8iyZUD1wkqQs2p+kAGEeEjBQAObwRKAAZgBOEAC2SGQgOBBIopJ67BEKIDSM9ABGMIwACi7mBlhg2LBO0XEJNMlIAMw63uwAOh3YwbH0OSB5hSVlGgaMMBE4A4VgUC1pcAAWWFNIALTEvJEx8YiJTYhW6XYgwdm5BUWlZmMgldVs2yB1ewcpiAAsbRkGSxeDK4jW7Ce5VLA1Z6vFqND5pWw+LKOS7DG6CO4PCFPZTQxCtJIfY4I9jnZGA1GjUGYyE43YwglIb4nHz-MlDa6U9jUp6UbhAA
\begin{tikzcd}
 & A &                                                                                              \\
Z_1 \arrow[ru, "f_1" description] \arrow[rr, "\sigma", bend left, shift left] \arrow[r, "g_1" description] \arrow[rd, "h_1" description] & B & Z_2 \arrow[lu, "f_2" description] \arrow[l, "g_2" description] \arrow[ld, "h_2" description] \\
 & C &
\end{tikzcd}

We now define the following object:

% https://tikzcd.yichuanshen.de/#N4Igdg9gJgpgziAXAbVABwnAlgFyxMJZARgBoAGAXVJADcBDAGwFcYkQBBEAX1PU1z5CKcqWLU6TVuw4ACADry8AW3iyAQgqVZVcWQGEefEBmx4CRMuJoMWbRCHVH+ZoZdIAmCbekPD3CRgoAHN4IlAAMwAnCGUkMhAcCCRRSTt2NAB9LhpGegAjGEYABQFzYRAorGCACxxnEGjY+JokpA8bKXsTTKdcgqLS1wsHLDBsWAamuMQEtsQAZk70hyzDfsKSsrcHRhgI+oDuIA
\begin{tikzcd}
                                                                                      & A \\
A \times B \times C \arrow[ru, "p_A"] \arrow[r, "p_B" description] \arrow[rd, "p_C"'] & B \\
                                                                                      & C
\end{tikzcd}

and using the two universal properties we have shown above, we know that necessarily $(A \times B) \times C$ and $A \times (B \times C)$ are terminal objects in $\mathcal{C}_{A,B,C}$. This is because they are final objects in respective bislice categories, given that $((f,g),h)$ and $(f,(g,h))$ are both unique; and we can compose $p_A \circ p_{A \times B}$, etc, to get immediate projections, making them final in the trislice category as well. To be precise:

$$(((A \times B) \times C), \; p_A \circ p_{A \times B}, \; p_B \circ p_{A \times B}, \; p_C)$$

$$((A \times (B \times C)), \; \pi_A, \; \pi_B \circ \pi_{B \times C}, \; \pi_C \circ \pi_{B \times C})$$

are both final objects in $\mathcal{C}_{A,B,C}$, as is shown by the following commutative diagram, which is composed from the diagrams above:

\begin{tikzcd}
  &  &   & A \times B \arrow[ldd, "p_B" description, two heads, dashed, bend left] \arrow[ld, "p_A" description, two heads, dashed]   &  \\
  &  & A &   & (A \times B) \times C \arrow[lldd, "p_C" description, two heads, dashed, bend left] \arrow[lu, "p_{A \times B}" description, two heads, dashed, bend right]  \\
Z \arrow[rrd, "f_C" description] \arrow[rrrru, "f_{(A \times B) \times C}" description, dotted] \arrow[rr, "f_B" description] \arrow[rru, "f_A" description] \arrow[rrruu, "f_{A \times B}" description, dotted, bend left] \arrow[rrrrd, "f_{A \times (B \times C)}" description, dotted] \arrow[rrrdd, "f_{B \times C}" description, dotted, bend right] &  & B &   &  \\
  &  & C &   & A \times (B \times C) \arrow[ld, "\pi_{B \times C}" description, two heads, dashed, bend left] \arrow[lluu, "\pi_A" description, two heads, dashed, bend right] \\
  &  &   & B \times C \arrow[luu, "\pi_B" description, two heads, dashed, bend right] \arrow[lu, "\pi_C" description, two heads, dashed] & 
\end{tikzcd}

We use Proposition 5.4 to deduce that $(A \times B) \times C$ and $A \times (B \times C)$, as terminal objects in $\mathcal{C}_{A,B,C}$, are necessarily isomorphic. Therefore, given this "associativity up-to-isomorphism" of a triple product, it is legitimate to call $A \times B \times C$ (with no parentheses) "the" (unique, up-to-isomorphism) final object in $\mathcal{C}_{A,B,C}$.


\subsection*{5.10)}

Push the envelope a little further still, and define products and coproducts for families (i.e., indexed sets) of objects of a category. Do these exist in \textbf{Set}? It is common to denote the product $A \times \cdot \cdot \cdot \times A$ ($n$ times) by $A^n$.

Let $\mathcal{F}$ be an (ordered) family of objects in some category $\mathcal{C}$.

\subsubsection*{5.10.a)}

A product of the elements of $\mathcal{F}$ is defined as an object $P$ of $\mathcal{C}$ together with a family of morphisms $\{p_A : P \to A\}_{A \in \mathcal{F}}$ such that for any object $Z$ and family of morphisms $\{f_A : Z \to A\}_{A \in \mathcal{F}}$, there exists a unique morphism $f : Z \to P$ such that the following diagram commutes for all $A \in \mathcal{F}$:

\begin{tikzcd}
Z \arrow[rd, "f_A"'] \arrow[r, "f"] & P \arrow[d, "p_A"] \\
                                    & A                 
\end{tikzcd}

These exist in \textbf{Set} for all finite families of sets, however, for infinite families of sets, their existence is conditional on the (famous) axiom of choice (it is actually precisely the point of the axiom of choice).

\subsubsection*{5.10.b)}

A coproduct of the elements of $\mathcal{F}$ is defined as an object $C$ of $\mathcal{C}$ together with a family of morphisms $\{i_A : A \to C\}_{A \in \mathcal{F}}$ such that for any object $Z$ and family of morphisms $\{f_A : A \to Z\}_{A \in \mathcal{F}}$, there exists a unique morphism $f : C \to Z$ such that the following diagram commutes for all $A \in \mathcal{F}$:

\begin{tikzcd}
A \arrow[r, "i_A"] \arrow[rd, "f_A"'] & C \arrow[d, "f"] \\
									  & Z                
\end{tikzcd}

These exist in \textbf{Set} for all families of sets, however, for infinite families of sets, their existence is conditional on the axiom of choice.






\subsection*{5.11)}

Let $A$, resp. $B$, be sets, endowed with equivalence relations $\sim_A$, resp. $\sim_B$. Define a relation $\sim$ on $A \times B$ by setting $(a_1, b_1) \sim (a_2, b_2) \Leftrightarrow a_1 \sim_A a_2 \text{ and } b_1 \sim_B b_2.$ (This is immediately seen to be an equivalence relation.)
\begin{itemize}
	\item Use the universal property for quotients (§5.3) to establish that there are functions $(A \times B)/\sim \; \to \; A/\sim_A$, and $(A \times B)/\sim \; \to \; B/\sim_B$;
	\item prove that $(A \times B)/\sim$, with these two functions, satisfies the universal property for the product of $A/\sim_A$ and $B/\sim_B$;
	\item conclude (without further work) that $(A \times B)/ \sim \; \simeq (A / \sim_A) \times (B / \sim_B)$.
\end{itemize}

\subsubsection*{5.11.a)}

First, we remind the universal property for quotients. Given an equivalence relation $\sim$ over a set $A$, there is a single map $\pi : A \to A/\sim$ such that, for any map $f : A \to Z$ verifying $f(a_1) = f(a_2)$ whenever $a_1 \sim a_2$ (i.e., the morphism $f$ is "well-defined" for the equivalence relation $\sim$), there exists a unique map $\overline{f} : A/\sim \to Z$ such that $f = \overline{f} \circ \pi$. This is summarized in the following commutative diagram:

% https://tikzcd.yichuanshen.de/#N4Igdg9gJgpgziAXAbVABwnAlgFyxMJZABgBpiBdUkANwEMAbAVxiRAEEB6AHW+wFsQAX1LpMufIRQBGclVqMWbAFrDRIDNjwEiZafPrNWiDsPkwoAc3hFQAMwBOEQYjIgcEJLIVG2vCDQwDgxYYDDAdkJq9k4uAEzUHkhuhkomvGhY0SCOzkgJ7p6I3qnGOSDUDHQARjAMAAri2lIgDliWABY4ZkJAA
\begin{tikzcd}
    A/\sim \arrow[r, "\overline{f}"]    & Z \\
    A \arrow[u, "\pi"] \arrow[ru, "f"'] &  
\end{tikzcd}

We now apply the universal property of products to obtain the following commutative diagram:

% https://tikzcd.yichuanshen.de/#N4Igdg9gJgpgziAXAbVABwnAlgFyxMJZABgBpiBdUkANwEMAbAVxiRAEEQBfU9TXfIRQBGclVqMWbdgAIAOnLwBbeDIBC3XiAzY8BIgCYx1es1aIQGnn12Cio4eNNSLALW7iYUAObwioADMAJwglJDIQHAgkAGYTSXMQAIB9TmoGOgAjGAYABX49IRAsMGxYTUCQsMRRSOjECOdEtFSQdKyc-Nt9CxKy1mskqqRaqKQjCTM2Fo127LyCu17SrHLB4NCR6jHEOMmXJLaQDPmugR7jmACcCqHNxAmdvaa2FNnjjoXuor7VgYouEA
\begin{tikzcd}
A \arrow[rd, "f_A" description] & A \times B \arrow[l, "p_A" description] \arrow[r, "p_B" description] \arrow[d, "f"] & B \arrow[ld, "f_B" description] \\
                                & Z                                                                                   &                                
\end{tikzcd}

We then consider a well-defined map $f : (A \times B) \to Z$ (i.e., $f(a_1, b_1) = f(a_2, b_2)$ whenever $(a_1, b_1) \sim (a_2, b_2)$). We apply the universal property of quotients to the relation $\sim$ defined over $A \times B$. We define the map $\pi : A \times B \to (A \times B)/\sim$ as $\pi(a,b) = [(a,b)]_\sim$. We define the map $\overline{f} : (A \times B)/\sim \; \to \; Z$ as $\overline{f}([(a,b)]_\sim) = f(a,b)$. We remind that such an $f$ is unique, and that $\overline{f} \circ \pi = f$.

We do a similar construction for $A/\sim_A$ with $f_A$ and $B/\sim_B$ with $f_B$. We can do so since these maps are also well-defined. Indeed, we have:

$$
\begin{array}{l}
	\left\{
	\begin{array}{l}
		(a_1, b_1) \sim (a_2, b_2) \\
		(a_1, b_1) \sim (a_2, b_2) \Rightarrow f(a_1, b_1) = f(a_2, b_2) \\
		f = f_A \circ p_A \\
		f = f_B \circ p_B
	\end{array}
	\right.
\cr \Rightarrow \\
	\left\{
	\begin{array}{l}
		a_1 \sim_A a_2 \text{ and } b_1 \sim_B b_2 \\
		f_A(p_A(a_1, b_1)) = f_A(p_A(a_2, b_2)) \\
		f_B(p_B(a_1, b_1)) = f_B(p_B(a_2, b_2)) 
	\end{array}
	\right.
\cr \Rightarrow \\
	\left\{
	\begin{array}{l}
		f_A(a_1) = f_A(a_2) \\
		f_B(b_1) = f_B(b_2) 
	\end{array}
	\right.
\end{array}
$$

The combination of all previous steps gives us the following commutative diagram.

% https://tikzcd.yichuanshen.de/#N4Igdg9gJgpgziAXAbVABwnAlgFyxMJZAJgBoBGAXVJADcBDAGwFcYkQBBAegB0fsAtiAC+pdJlz5CKAMykADNTpNW7DiLEgM2PASLlSxJQxZtEIAEK9+WIaPE6pRMkZonV5gFoaHkvSnlSGWMVM0sfLQldaWRAxTdQtQACPjwBeCSLCO0-GIMqBNN2AAoOFJ40jIsASmtBESUYKABzeCJQADMAJwghRAMQHAgkQOUi8z40LAB9dRpGegAjGEYABSincywwbFgI7t6kMkHhxDkxjxA+CFoYLsZtmGAO6YthEHmllfXHfxBt3ZsewgA59AAsNCGR0Kl0mMyyn2Waw2fwBWD2wNBSAhJyQ5wWSJ+uXYaL2MLCLyymJ6fQGULO5PYLzmIAJ3xR0n+O3RQM0WMQo3p53cYWut3uj2es3eiPZv05pN5nRpSAArJDTgMRew0LMPqyvsj5STuRi+SrEOrcYgcdrzLqEQbCRyTYD9harfSAGyMiY8Kb6tlG4lbU1KkEWn3W4WJP03O4PMBPDoyp1ykNct3Uw6WjV4mjLMBQJAAWhkoztIMDhqJ0VdPIawiAA
% \begin{tikzcd}
%     A \times B \arrow[rrr, "p_A" description] \arrow[ddd, "p_B" description] \arrow[rd, "\pi" description] \arrow[rrdd, "f" description, bend right] &                                                          &                                                & A \arrow[ld, "\pi_A" description] \arrow[ldd, "f_A" description] \\
%                                                                                                                                                      & (A \times B)/\sim \arrow[rd, "\overline{f}" description] & A/\sim \arrow[d, "\overline{f_A}" description] &                                                                  \\
%                                                                                                                                                      & B/\sim \arrow[r, "\overline{f_B}" description]           & Z                                              &                                                                  \\
%     B \arrow[ru, "\pi_B" description] \arrow[rru, "f_B" description]                                                                                 &                                                          &                                                &                                                                 
% \end{tikzcd}

% https://tikzcd.yichuanshen.de/#N4Igdg9gJgpgziAXAbVABwnAlgFyxMJZABgBpiBdUkANwEMAbAVxiRAEEACAHW7wFt4nAEIgAvqXSZc+QigCM5KrUYs27cZJAZseAkQBMS6vWatEHAPS9s-TVN2yiZectNqLoiQ5n6FpVxNVcxAACi5eASFhAEprblt7bWk9ORJSAzdgtmF4xO9kxz9kI0ygszYALXFlGCgAc3giUAAzACcIO0QyEBwIJEUVCos0AH0NagY6ACMYBgAFFKcLLDBsWCT2zqQevqQAZnKPbVHRSZm5xaK5EFX11gKtrt3+xAAWI5DeNCwQc9mFks-Lc1lgNo8Os9qHtEAA2T5sFp-EBTAFXXw3O5g1jUWZgKBIAC0+2IEO2iEGMKMQ2O3yw42RqMuQMxoPBWieA2hr3hNJCLQZ-2Z1zYWI2uJg+J2ZK61JhvPcX24EBoMDaDFWMGAAvYYkZF0BIpWbIeHMhB25SAArAiLHTTvq0SzRSbNubEIdejzbSABWcUQb0akXfdkXiCYhiaSzeSPl6kArsnblar1ZrtXqhYaMSHsW7yTb43CfbwVWqNWAtX7MwGnUaQaGxBQxEA
\begin{tikzcd}[row sep = 2.75cm, column sep = 1.5cm]
A \times B \arrow[r, "p_A" description] \arrow[d, "p_B" description] \arrow[rd, "\pi" description] \arrow[rrdd, "f" description, bend right] & A \arrow[r, "\pi_A" description] \arrow[rdd, "f_A" description] & A/\sim_A \arrow[dd, "\overline{f_A}" description] \\
B \arrow[d, "\pi_B" description] \arrow[rrd, "f_B" description, bend right]                                                                  & (A \times B)/\sim \arrow[rd, "\overline{f}" description]        &                                                 \\
B/\sim_B \arrow[rr, "\overline{f_B}" description]                                                                                              &                                                                 & Z                                              
\end{tikzcd}

Let us recap a little what's in diagram a little, to show that it is a justified construction:
\begin{itemize}
	\item $A \times B$ is a product, and so we may obtain $p_A$, $p_B$, $f_A$, and $f_B$;
    \item $f$ is well-defined for the equivalence relation $\sim$, and so we may obtain $\overline{f}$ and $\pi$;
    \item $f_A$ is well-defined for the equivalence relation $\sim_A$, and so we may obtain $\overline{f_A}$ and $\pi_A$;
    \item $f_B$ is well-defined for the equivalence relation $\sim_B$, and so we may obtain $\overline{f_B}$ and $\pi_B$;
\end{itemize}

Therefore, we have the maps $\Pi_A = \pi_A \circ p_A \in (A \times B) \to A/\sim_A$ and $\Pi_B = \pi_B \circ p_B \in (A \times B) \to B/\sim_B$. $\Pi_A$ and $\Pi_B$ are well-defined because they are compositions of well-defined maps. For $\pi_A$ and $\pi_B$, we know that the projector and unique function part (of the universal property of quotients) are necessarily well-defined, otherwise they wouldn't compose to a well-defined map. As for $p_A$ and $p_B$, they are well-defined because they are projections, and projections are always well-defined.

Since $\Pi_A$ and $\Pi_B$ are well-defined, we can "quotient through" (i.e., use the universal property of quotient for) $(A \times B)/\sim$.

This gives us the following commutative diagram (compatible with the one above, but it's too messy to represent both at the same time):

% https://tikzcd.yichuanshen.de/#N4Igdg9gJgpgziAXAbVABwnAlgFyxMJZAJgBoAGAXVJADcBDAGwFcYkQAhEAX1PU1z5CKAIwVqdJq3YBBAAQAdBXgC28OV179seAkXLiaDFm0QgZPPiAw6hRMiInHpZjgHol2FZe2C9o0kcjKVMQAAp5JVV1DgBKDwUvH2sBXWFkAyDJE1kEpO4JGCgAc3giUAAzACcIb0QDEBwIJABmYJyzJTQsAH0uGkZ6ACMYRgAFVLszLDBsWGTq2qQxRubEMmyXax6LAeHRidt-EBm5ti0QRbqVpqQG51C0PpA9kfHJ49OseYur5ZpbogACztLZdLAvECDN6HPzCE6zb7nKx-RA3NYAVlBoSUY16uyh+3eR3hXx+KJq1wBazamxxCjxz1eBw+pMR5MqlKQG0BWLp7HBO0h0JZJPYZORnKWwOpSD5DwFCggtBgVUYMxgwFx+O4wqJsLS7EYMAqOAWXJlq1a2MVytV6rAmu1fV1zOJcPYVSwxQAFmaCtwgA
\begin{tikzcd}
A \arrow[d, "\pi_A" description] & A \times B \arrow[l, "p_A" description] \arrow[r, "p_B" description] \arrow[d, "\pi" description] \arrow[ld, "\Pi_A" description] \arrow[rd, "\Pi_B" description] & B \arrow[d, "\pi_B" description] \\
A/\sim_A                           & (A \times B)/\sim \arrow[l, "\overline{\Pi_A}"] \arrow[r, "\overline{\Pi_B}"']                                                                                    & B/\sim_B                          
\end{tikzcd}

These are precisely our functions $\overline{\Pi_A} \in (A \times B)/\sim \; \to \; A/\sim_A$ and $\overline{\Pi_B} \in (A \times B)/\sim \; \to \; B/\sim_B$, and they are unique, as per the universal property of quotients.


\subsubsection*{5.11.b)}

What we wish to prove now is that $(A \times B)/\sim$ is the product of $A/\sim_A$ and $B/\sim_B$. Said otherwise, we wish to show that, for all $Z$, $g_A$ and $g_B$ in the appropriate configuration, the following diagram commutes:

% https://tikzcd.yichuanshen.de/#N4Igdg9gJgpgziAXAbVABwnAlgFyxMJZARgBoAGAXVJADcBDAGwFcYkQAtEAX1PU1z5CKcqWLU6TVuwCCAegA6C7AFsefEBmx4CRMuJoMWbRCAAUMgARK8K+JYBCASkXKsa3v21CiAJjESRtKmDq6qPBIwUADm8ESgAGYAThBqiKIgOBBI-pLG7GbRAPoypJbFziA0jPQARjCMAAoCOsIgWGDYsOqJKWkZWUhkecEgxTJVIDX1TS0+ph1dbJ4gyalIA9mIAMyGUiZjRQ6T0w3N3roLnVjdK2tpuYOIw0EHShC0MEmMHTDASo0sCVuCc6mc5pcpjAEjgeqs+jkaE9diM3goPl8fmA-gCgQ4QdUwbMLm0klhogALWHcSjcIA
\begin{tikzcd}
       & Z \arrow[d, "{(g_A, g_B)}" description] \arrow[ld, "g_A" description] \arrow[rd, "g_B" description] &        \\
A/\sim_A & (A \times B)/\sim \arrow[l, "\overline{\Pi_A}"] \arrow[r, "\overline{\Pi_B}"']                      & B/\sim_B
\end{tikzcd}

Using the fact that $(A \times B)$ is already a product, and studying maps $h_A$ and $h_B$ from $Z$ into $A$ and $B$ respectively, we can make the following commutative diagram:

% https://tikzcd.yichuanshen.de/#N4Igdg9gJgpgziAXAbVABwnAlgFyxMJZARgBoAGAXVJADcBDAGwFcYkQAtEAX1PU1z5CKcqWLU6TVuwCCPPiAzY8BImXE0GLNohAyABAB1DeALbx9AIXn9lQogCYxErdN3XetwapGkHLqR09AHpjbFMAfTlPRQEVYRI-AO12AAoDYzMLSwBKUMNwm1i7H2Qnf01A9kt88IiPCRgoAHN4IlAAMwAnCFMkURAcCCQySRTdAAsokBpGegAjGEYABTj7XSwwbFgi7t7+miGkAGZK8ZAp61mFpdWS4RBN7bYYvb7EAaPEJzG3EFSpjJSJccjMQHNFis1j5HlssDtXj13p9hogACxnP7NabXSF3bwPJ7wtg0RZgKBIAC0aIAnDQ4BMsB0cOxjrskQdBqiAGyYoLYq7gm5Q+7sIk7UkwclIWn0xnMqnHRH7RCjL4Y35BYxoLA4oV46GEuEIhRvJA-L6jVxBNB6iG3Q1i40vU0c76HVGnTXsW2C+0iglO57slUW1EAVj57G1WDB-vx8SDxJD7zDMqjumMy11clxDtFG2dKfNHqQvO9mcM2fqceFCfWsODyveXq+5et0cMOpreYDicLTddKsjXPTFZAxggtBgXUYmxgwCzOe4tYNBfBMAVzaQI7bGYnhinM7nYAXS-qK979ZhXSwzQmLO4lG4QA
\begin{tikzcd}
                                 & Z \arrow[ld, "h_A" description] \arrow[rd, "h_B" description] \arrow[d, "{(h_A,h_B)}" description] \arrow[ldd, "g_A" description, bend right=49, shift right=3] \arrow[rdd, "g_B" description, bend left=49, shift left=3] &                                  \\
A \arrow[d, "\pi_A" description] & A \times B \arrow[l, "p_A" description] \arrow[r, "p_B" description] \arrow[d, "\pi" description] \arrow[ld, "\Pi_A" description] \arrow[rd, "\Pi_B" description]                                                          & B \arrow[d, "\pi_B" description] \\
A/\sim_A                         & (A \times B)/\sim \arrow[l, "\overline{\Pi_A}"] \arrow[r, "\overline{\Pi_B}"']                                                                                                                                             & B/\sim_B                        
\end{tikzcd}

The internal parts of the diagram (except $h_A$ and $h_B$ which are arbitrary, all other arrows are unique morphisms) force $g_A$ and $g_B$ to commute with the rest of the diagram: any pairs of maps $g_A$ and $g_B$ that make the diagram commute must necessarily be constrained by the existing morphisms (as the composition of unique morphisms). Any choice of maps $g_A$ and $g_B$ must respect $g_A = \pi_A \circ h_A$ and $g_B = \pi_B \circ h_B$, for corresponding arbitrary (but inferred) $h_A$ and $h_B$; said otherwise, maps $g_A$ and $g_B$ in this configuration can only exist if they make the diagram commute. Therefore, we have shown that $(A \times B)/\sim$ satisfies the universal property of the product of $A/\sim_A$ and $B/\sim_B$, and we have $(g_A, g_B) = \pi \circ (h_A, h_B)$ (though it's not shown on the above diagram, in order to improve legibility).

\subsubsection*{5.11.c)}

We know that elements that verify a common universal property are terminal objects (of the same kind) in some comma category. Here, both $(A \times B)/ \sim$ and $((A / \sim_A) \times (B / \sim_B))$ are final objects in the bislice category $\text{\textbf{Set}}_{A,B}$. Since they are both final objects in the same category, they are isomorphic.

Conclusion: $((A \times B)/ \sim) \; \simeq \; ((A / \sim_A) \times (B / \sim_B))$



\subsection*{5.12)}

Define notions of fibered products and coproducts, as terminal objects of the categories $\mathcal{C}_{\alpha,\beta}$, $C^{\alpha,\beta}$ considered in Example 3.10 (cf. also Exercise 3.11), by stating carefully the corresponding universal properties.
As it happens, \textbf{Set} has both fibered products and coproducts. Define these objects 'concretely', in terms of naive set theory. [II.3.9, III.6.10, III.6.11]


\subsubsection*{5.12.a)}

We first define the fibered product (or "pullback"). We suggest the reader goes to check section 3.11.3 of this exercises solutions document for a refresher on the concept "fibered bislice category" $\mathcal{C}_{\alpha,\beta}$. We will use the same notation, i.e., fixing two morphisms $\alpha : A \to C$ and $\beta : B \to C$ (with common codomain) in a category $\mathcal{C}$.

This means, for any generic object $(Z, f, g)$ of $\mathcal{C}_{\alpha,\beta}$, we have the following diagram:

% https://tikzcd.yichuanshen.de/#N4Igdg9gJgpgziAXAbVABwnAlgFyxMJZARgBoAGAXVJADcBDAGwFcYkQBBEAX1PU1z5CKAEyli1Ok1bsAwjz4gM2PASJkRkhizaIQAIQX8VQouXFbpukAC0ekmFADm8IqABmAJwgBbJOZAcCCQyKR12AB0IpjQAC3ojEC9fJDFA4MRQ7Rk9KIAjGBwEmkZ6AsYABQFVYRBPLCdYnETkv0QAZhog-xps63cW7zbO9NSSsphK6tM9esbm3qt2J3tuIA
\begin{tikzcd}
                                   & A \arrow[rd, "\alpha"] &   \\
Z \arrow[ru, "f"] \arrow[rd, "g"'] &                        & C \\
                                   & B \arrow[ru, "\beta"'] &  
\end{tikzcd}

We first propose a candidate for what it means to be a "fibered product", which we will be able to show is a final object in $\mathcal{C}_{\alpha,\beta}$. A fibered product $P$ of $A$ and $B$ over $C$ in a category $\mathcal{C}$ is an object $P$ together with morphisms $p_A : P \to A$ and $p_B : P \to B$ such that for any object $Z$ in $\mathcal{C}$ and morphisms $f : Z \to A$ and $g : Z \to B$ such that $\alpha \circ f = \beta \circ g$, there exists a unique morphism $h : Z \to P$ such that the following diagram commutes:

% https://tikzcd.yichuanshen.de/#N4Igdg9gJgpgziAXAbVABwnAlgFyxMJZAJgBoAGAXVJADcBDAGwFcYkQBBEAX1PU1z5CKcqQCM1Ok1bsAWjz4gM2PASIBmcZIYs2iEAGEF-FUKJli26XpAAhY0oGrhyMVpo6Z+gAo9JMKABzeCJQADMAJwgAWyRREBwIJDIpXXYAHXSmNAALehAaRnoAIxhGbyczfUYYMJwHSJikNwSkxHjPGzCCkCLS8sq1atr63nCo2MQWxKRNVK8QQJ6+sorTIZAIrECc0cVGybmZxBTOjPTSnHzCktXB4U3t3YaJpAAWGmOO63Y0AH0uDd+mtBBstjs9uMmogPq1Zh4fvp-vYxiADs1Pm1YSsBusHlgwNhYD0zvocn5uEA
\begin{tikzcd}
                                                                &                                        & A \arrow[rd, "\alpha"] &   \\
Z \arrow[rru, "f"] \arrow[rrd, "g"'] \arrow[r, "h" description] & P \arrow[ru, "p_A"'] \arrow[rd, "p_B"] &                        & C \\
                                                                &                                        & B \arrow[ru, "\beta"'] &  
\end{tikzcd}

We will now show that this object $P$ induces a terminal object in $\mathcal{C}_{\alpha,\beta}$; this boils down to showing that $(P, p_A, p_B)$ is final in $\mathcal{C}_{\alpha,\beta}$. Let $(Z, f, g)$ be an arbitrary object of $\mathcal{C}_{\alpha,\beta}$. A morphism $\sigma$ from $(Z, f, g)$ to $(P, p_A, p_B)$ is a "raising" of a morphism $h : Z \to P$ such that $p_A \circ h = f$ and $p_B \circ h = g$. This raising $\sigma$ is unique if and only if $h$ is unique. Since $P$ is presupposed to verify the universal property, such an $h$, if it exists, is indeed unique. Therefore, $(P, p_A, p_B)$ is final (\textit{a fortiori} terminal) in $\mathcal{C}_{\alpha,\beta}$.


\subsubsection*{5.12.b)}

We now define the fibered coproduct (or "pushout"). We suggest the reader goes to check section 3.11.4 of this exercises solutions document for a refresher on the "fibered bicoslice category" $C^{\alpha,\beta}$. We will use the same notation, i.e., fixing two morphisms $\alpha : C \to A$ and $\beta : C \to B$ (with common domain) in a category $\mathcal{C}$.

This means, for any generic object $(f, g, Z)$ of $C^{\alpha,\beta}$, we have the following diagram:

% https://tikzcd.yichuanshen.de/#N4Igdg9gJgpgziAXAbVABwnAlgFyxMJZARgBoAGAXVJADcBDAGwFcYkQBBEAX1PU1z5CKcqWLU6TVuwDCPPiAzY8BIgCYxEhizaIQALXn9lQomTVapukACEeEmFADm8IqABmAJwgBbJKJAcCCQNSR12dxAaRnoAIxhGAAUBFWEQTywnAAscIxAvXyQyQODEAO1pPQAdKqY0LPookBj4pJTTPQzs3N4Pbz9EYqCkAGYaCusa+JxG6LiE5JNVPUYYdx6FAoGxkpDxq3YnJpaF9uXmtZ7KbiA
\begin{tikzcd}
                                            & A \arrow[rd, "f"'] &   \\
C \arrow[ru, "\alpha"'] \arrow[rd, "\beta"] &                    & Z \\
                                            & B \arrow[ru, "g"]  &  
\end{tikzcd}

We first propose a candidate for what it means to be a "fibered coproduct", which we will be able to show is an initial object in $C^{\alpha,\beta}$. A fibered coproduct $P$ of $A$ and $B$ over $C$ in a category $\mathcal{C}$ is an object $P$ together with morphisms $p_A : A \to P$ and $p_B : B \to P$ such that for any object $Z$ in $\mathcal{C}$ and morphisms $f : A \to Z$ and $g : B \to Z$ such that $f \circ \alpha = g \circ \beta$, there exists a unique morphism $h : P \to Z$ such that the following diagram commutes:

% https://tikzcd.yichuanshen.de/#N4Igdg9gJgpgziAXAbVABwnAlgFyxMJZARgBoAGAXVJADcBDAGwFcYkQBBEAX1PU1z5CKcqWLU6TVuwDCPPiAzY8BIgGYxEhizaIQALXn9lQomQBMWqbpAAhI4oErhyc5prbpegAo8JMKABzeCJQADMAJwgAWyRREBwIJDdJHXYwkBpGegAjGEZvJ1M9RhgwnAdImKQyBKTEeM8bAB1mpjQAC3pMkGy8gqLVPQisQI6K3nCo2MRaxKQNVK8QVrycbqzc-MKTId6yiYUqmcX5xBSm9kCevu3B4RARscOp6sQAFhozlNuB3YesGBsLAepc9B1KtM4l96p8ljYsAB9Libfo7QR7J7jSFvU6wjzWdhI+zcSjcIA
\begin{tikzcd}
                                            & A \arrow[rrd, "f"] \arrow[rd, "i_A"'] &                              &   \\
C \arrow[ru, "\alpha"'] \arrow[rd, "\beta"] &                                       & P \arrow[r, "h" description] & Z \\
                                            & B \arrow[rru, "g"'] \arrow[ru, "i_B"] &                              &  
\end{tikzcd}

We will now show that this object $P$ induces an initial object in $C^{\alpha,\beta}$; this boils down to showing that $(i_A, i_B, P)$ is initial in $C^{\alpha,\beta}$. Let $(f, g, Z)$ be an arbitrary object of $C^{\alpha,\beta}$. A morphism $\sigma$ from $(i_A, i_B, P)$ to $(f, g, Z)$ is a "raising" of a morphism $h : P \to Z$ such that $h \circ i_A = f$ and $h \circ i_B = g$. This raising $\sigma$ is unique if and only if $h$ is unique. Since $P$ is presupposed to verify the universal property, such an $h$, if it exists, is indeed unique. Therefore, $(i_A, i_B, P)$ is initial (\textit{a fortiori} terminal) in $C^{\alpha,\beta}$.


\subsubsection*{5.12.c)}

(I wouldn't have come up with this without Wikipedia as a hint...)

In \textbf{Set}, the fibered product (pullback) of $(A, \alpha)$ and $(B, \beta)$ over $C$, is a special subset of the cartesian product $A \times B$ that registers some extra information, pertaining to the functions $\alpha$ and $\beta$. We write this object as $$P = A \times_{\alpha, C, \beta} B = A \times_C B = \{ (a,b) \in A \times B \mid \alpha(a) = \beta(b) \}$$.

Concretely, this can also be expressed as $$P = \bigcup_{c \in \alpha(A) \cap \beta(B)} \alpha^{-1}[\{c\}] \times \beta^{-1}[\{c\}]$$. This is the set of all pairs (of inputs) $(a,b)$ such that $\alpha(a) = \beta(b)$. Let us show this is the case with a simple example.

Let $A = \{1,2,3\}$, $B = \{w, x, y, z\}$, and $C = \{l, m, n, p\}$. We define $\alpha : A \to C$ as $\{(1, m), (2, m), (3, n) \}$, and $\beta : B \to C$ as $\{ (w, l), (x, m), (y, n), (z, n) \}$. We have $\alpha(A) = \{m,n\}$ and $\beta(B) = \{l, m, n\}$, therefore $\alpha(A) \cap \beta(B) = \{m,n\}$. The fibered product is then:

$$
\begin{array}{lll}
P &=& (\alpha^{-1}[\{m\}] \times \beta^{-1}[\{m\}]) \; \cup \; (\alpha^{-1}[\{n\}] \times \beta^{-1}[\{n\}]) \\
  &=& \{ \{1,2\} \times \{x\} \} \; \cup \; \{ \{3\} \times \{y,z\} \} \\
  &=& \{ (1,x), (2,x) \} \; \cup \; \{ (3,y), (3,z) \} \\
  &=& \{ (1,x), (2,x), (3,y), (3,z) \}
\end{array}
$$

A simple way to verify this is to verify that for each pair $(a,b)$ in $P$, $\alpha(a) = \beta(b)$, and that no such other pairs are missing.


\subsubsection*{5.12.d)}

https://math.stackexchange.com/questions/3021738/pushout-in-the-category-of-sets-proof

https://math.stackexchange.com/questions/2240882/understanding-an-example-of-a-pushout-in-mathbfset

In \textbf{Set}, the fibered coproduct (pushout) of $(\alpha, A)$ and $(\beta, B)$ over $C$ is a special quotient of the disjoint union $A \sqcup B$ that registers some extra information, pertaining to the functions $\alpha$ and $\beta$. We write this object as $$P = A \sqcup_{\alpha, C, \beta} B = A \sqcup_C B = (A \sqcup B) / \sim$$, where $\sim$ is an equivalence relation defined as  $\sim \; := cl_{eq}(R)$, the equivalence closure of the relation $$R = \{((0, a), (1, b)) \in (A \sqcup B) \times (A \sqcup B) \mid \exists c \in C, a = \alpha(c) \text{ and } b = \beta(c) \}$$ (i.e., two elements are equivalent if they are both the output of some common $c$, or if there is any chain of such equivalences between them).

(We remind that the equivalence closure, if elements of the set are seen as vertices and the relation pairs as edges of a graph, can be thought of, visually, as "completing the cliques" of each connected component in the graph, including self-loops; hence the idea that if there is any path between two elements, then they are equivalent and should be directly linked. We also remind that this corresponds uniquely to a partition of the set; the elements of which, called cosets, are what allow us to do an algebraic quotient.)

The fibered coproduct is the set $P$ obtained by taking the disjoint union $A \sqcup B$ and identifying $a \in A$ with $b \in B$ if there exists $c \in C$ such that $\alpha(c) = a$ and $\beta(c) = b$ (and all identifications that follow to keep the equality relation an equivalence relation). There is no more concrete of a definition than this; it really boils down to identifying elements with a common preimage element through $\alpha^{-1}(A)$ and $\beta^{-1}(B)$ in $C$, via an equivalence relation.

Let us show this is the case with a simple example. We will keep the elements of $A$ and $B$ distinct in order to remove the visual clutter that comes with the $(0, ...)$ and $(1, ...)$ of the general disjoint union.

Let $A = \{ 1, 2, 3 \}$, $B = \{ w, x, y, z \}$, and $C = \{ l, m, n \}$. We define $\alpha : C \to A$ as $\{ (l, 1), (m, 1), (n, 2) \}$, and $\beta : C \to B$ as $\{ (l, x), (m, y), (n, z) \}$. We have:

\begin{itemize}
	\item $\alpha^{-1}(\{1\}) = \{l,m\}$ and $\beta^{-1}(\{x\}) = \{l\}$, so $1 \sim x$;
	\item $\alpha^{-1}(\{1\}) = \{l,m\}$ and $\beta^{-1}(\{y\}) = \{m\}$, so $1 \sim y$, and by closure, $1 \sim x \sim y$;
	\item $\alpha^{-1}(\{2\}) = \{n\}$ and $\beta^{-1}(\{z\}) = \{n\}$, so $2 \sim z$;
	\item $\alpha^{-1}(\{3\}) = \emptyset$ and $\beta^{-1}(\{w\}) = \emptyset$, so you might think that $3 \sim w$, however, since there is no $c \in C$ such that $\alpha(c) = 3$ and $\beta(c) = w$, we have $3 \nsim w$;
\end{itemize}

This information corresponds to the following partition of $A \sqcup B$: $\{ \{1,x,y\}, \{2,z\}, \{3\}, \{w\} \}$. The fibered coproduct is then:

$$
\begin{array}{lll}
P &=& (A \sqcup B) / \sim \\
  &=& \{1,2,3,w,x,y,z\} / \sim \\
  &=& \{\{1,x,y\}, \{2,z\}, \{3\}, \{w\}\} \\
  &=& \{[1], [2], [3], [w]\}
\end{array}
$$

A way to verify this is to verify that each equivalence class is disjoint, and that all pairs of elements within an equivalence class are related by $\sim$ (by applying $\alpha$ or $\beta$ where appropriate and drawing the graph of the relation).
