\section*{Section 2)}

\subsection*{2.1)}

How many different bijections are there between a set $S$ with $n$ elements and itself?

Any bijection is a choice of a pairs from 2 sets of the same size, where each element is used only once, and each pair has one element from each set. At first there are $n$ choices in each set. We go through each possible input element in order (no choice), for each one, we pick one amongst $n$ possibilities for an output.

There are then $(n-1)$ choice of output left, etc.

Ccl°: $\prod_{i=1}^{i=n} i = n!$



\subsection*{2.2)}

Prove that a function has a right-inverse (pre-inverse) iff it is surjective (can use AC).

Let $f \in (A \to B)$.

\subsubsection*{2.2.a) $\Rightarrow$}

Suppose that $f$ has a right-inverse (pre-inverse).
We have $\exists g \in (B \to A), f \circ g = id_B$

Suppose that $f$ is not a surjection. This means $\exists b \in B, \nexists a \in A, b = f(a)$

$f(g(b))= id_B (b) = b$ Necessarily, $g(b)$ is such an $a$, so $\exists a \in A, b = f(a)$. Contradiction.

% Supposing ∃g (g : B → A) (f∘g = id_B)
% Then
% (∀b ∈ B) (∃a ∈ A)  g(b) = a            (by definition of a function)
% (∀b ∈ B) (∃a ∈ A)  f(g(b)) = f(a)      (by applying f to both sides)
% (∀b ∈ B) (∃a ∈ A)  b = f(a)            (by cancellation)

Ccl°:: f is a surjection.


\subsubsection*{2.2.b) $\Leftarrow$}

Suppose that f is a surjection.

$\forall b \in B, \exists a \in A, b = f(a)$

We will construct a pre-inverse for $f$.

The insight here is to realize that a surjection divides its input set into a partition, where each 2-by-2 disjoint subset corresponds to $f^{-1}(\{q\})$, for every $q$ in the output set. More formally, each "fiber" (preimage of a singleton) is a disjoint subset of the input set, and the union of fibers is the input set itself. You can see this in the following diagram:

(add diagram)
1234 to ab
1a 2a (fiber from a)
3b 4b (fiber from b)
https://tex.stackexchange.com/questions/157450/producing-a-diagram-showing-relations-between-sets
https://tex.stackexchange.com/questions/79009/drawing-the-mapping-of-elements-for-sets-in-latex

Using AC, we select a single element from each such fiber. For each $q \in B$, we name $p_q \in f^{-1}(\{q\})$ the chosen element. We define $g$ as $g \in (B \to A), g = (q \mapsto p_q)$. With this, $\forall b \in B, f \circ g (b) = b$, and so $f \circ g = id_A$. Thus, $f$ has a preinverse.

A summary of this idea: all surjection preinverses are simply a choice of a representative for each fiber of the surjection as the output to the respective singleton.



\subsection*{2.3)}

Prove that the inverse of a bijection is a bijection, and that the composition of two bijections is a bijection.

\subsubsection*{2.3.a)} Using the fact that a function is a bijection iff it has a two-sided inverse (Corollary 2.2) we can see from this defining fact, $f \in (A \to B) \text{ bijective } \Leftrightarrow \exists f^{-1} \in (B \to A), (f^{-1} \circ f = id_A \text { and } f \circ f^{-1} = id_B)$ that $f$ is naturally $f^{-1}$'s (unique) two-sided inverse, and so $f^{-1}$ is also a bijection.

\subsubsection*{2.3.b)} Let be $f \in (A \to B), g \in (B \to C)$, both bijective (hence with inverses in the respective function spaces). Let $h \in (A \to C), h = g \circ f$ and $h^{-1} \in (C \to A), h^{-1} = f^{-1} \circ g^{-1}$. We have:

$$
\begin{aligned}
h^{-1} \circ h &= (f^{-1} \circ g^{-1}) \circ (g \circ f) \\
               &=  f^{-1} \circ g^{-1}  \circ  g \circ f  \\
               &=  f^{-1} \circ          id_B    \circ f  \\
               &=  f^{-1} \circ                        f  \\
               &=  id_A
\end{aligned}
$$

$$
\begin{aligned}
h \circ h^{-1} &= (g \circ f) \circ (f^{-1} \circ g^{-1}) \\
               &=  g \circ f  \circ  f^{-1} \circ g^{-1}  \\
               &=  g \circ     id_B         \circ g^{-1}  \\
               &=  g \circ                        g^{-1}  \\
               &=  id_C
\end{aligned}
$$

Therefore $h$ and $h^{-1}$ are two-sided inverses of each other, and thus bijections. From this we conclude that the composition of any two bijections is also a bijection.



\subsection*{2.4)}

Prove that ‘isomorphism’ is an equivalence relation (on any set of sets).

\subsubsection*{2.4.a) Problem statement}

Let $\mathcal{A}$ be a set of sets. We define the relation $\simeq$ between the elements of $\mathcal{A}$ as the following:

$$\forall X, Y \in \mathcal{A}, \; X \simeq Y \Leftrightarrow \text {there exists a bijection between $X$ and $Y$}$$

Let us show that $\simeq$ is an equivalence relation.


\subsubsection*{2.4.b) Reflexivity}

For any set $A \in \mathcal{A}$, the identity mapping on $A$ is a bijection. This means that $\forall A \in \mathcal{A}, A \simeq A$, ie, $\simeq$ is reflexive.


\subsubsection*{2.4.c) Symmetry}

$$
\begin{aligned}
\forall X, Y \in \mathcal{A}, \; X \simeq Y & \Rightarrow \exists f      \in (X \to Y) \text{ bijective} \\
                                            & \Rightarrow \exists f^{-1} \in (Y \to X) \text{ bijective} \\
                                            & \Rightarrow Y \simeq X
\end{aligned}
$$

Therefore, $\simeq$ is symmetric.


\subsubsection*{2.4.d) Transitivity}

Let be $X, Y, Z \in \mathcal{A}$.
Suppose that $X \simeq Y$ and $Y \simeq Z$.
This means $\exists f \in (X \to Y), g \in (Y \to Z)$, both bijections.
Let be $h \in (X \to Z), h = g \circ f$. $h$ is also a bijection since the composition of two bijections is also a bijection (exercise 2.3).


The existence of $h$ implies $X \simeq Z$.

Therefore $\simeq$ is transitive.


\subsubsection*{2.4.e) Conclusion}

Isomorphism, $\simeq$, is a relation on an arbitrary set (of sets) which is always reflexive, symmetric and transitive. It is thus an equivalence relation.



\subsection*{2.5)}

Formulate a notion of epimorphism and prove that epimorphisms and surjections are equivalent.

See "notes" file: section "Proofs of mono/inj and epi/surj equivalence".


\subsection*{2.6)}

With notation as in Example 2.4, explain how any function $f \in (A \to B)$ determines a section of $\pi_A$.

A section is the preinverse of a surjection. Here, the surjection in question is $\pi_A$ the projection of $A \times B$ onto $A$.

Let $f \in (A \to B)$.

% We remind that all functions are technically themselves sets, more precisely, they are relations between pairs of sets, hence $(A \to B) \subseteq (A \times B)$. We first define the map $\iota \in ((A \to B) \to (A \times B)), \; \iota = (f \mapsto \Gamma_f)$, which canonically injects a function from the function space $(A \to B)$ into the cartesian product $(A \times B)$, by simply keeping every pair $(a, f(a))$ as-is. $\Gamma_f$ is the \textit{graph} or \textit{graphical representation} of $f$.

We now consider the function which maps an input $a \in A$ of $f$ to its "geometric representation" (its coordinates in the enclosing space $A \times B$, corresponding to a point of the graph $\Gamma_f$). 
$$\hat{f} \in (A \to (A \times B)), \hat{f} = ( \; a \mapsto (a, f(a)) \; )$$
We notice that $\hat{f}(A) = \Gamma_f$.

Naturally, $\pi_A \circ \hat{f} = (a \mapsto a) = id_A$, therefore, $\hat{f}$ is a pre-inverse (section) of $\pi_A$.

This set of relationships can be expressed in the following commutative diagram:

% incorrect because of iota's nature; shifting rank (sets of sets to simple sets) is a bad idea
% https://tikzcd.yichuanshen.de/#N4Igdg9gJgpgziAXAbVABwnAlgFyxMJZARgBoAGAXVJADcBDAGwFcYkQAdDgIxgHMsYYAFt6OAE5YAHgF8AggAIueYfAUAKRcogKAQgEolXDhvqkFAM31cYYKCLGTZIGaXSZc+QinKli1OiZWdi5eASFRCWkZBS1jBXojW3tIpxkXNxAMbDwCIgAmPwCGFjZETh5+QQco2Q04nB0DIy5LJLsatIz3HK8iMnzioLKKsOrU6K4AcXphUQB9CyM4Zm44GBx6oxU1ZuMTdTNLQ-1DGw6J5xkAmCg+eCJQC3EIYSRfEEakMkDS9jR5nIQDRGPReIwAAoeXLeECSPgACxwwJACJg9Cg7BwAHcIGiMQhXE8Xm9EB8vohCr9guUAcBNNsmvp0iCwTBIdC+uVGDALMiaPjMeUcXj0VBCZlnq9vjQKQBmGglGkVBFiYAWFkgUHgqG9PLleFIlE4ehYRjsBEQCAAaxRcARWD5MpAvDs7yJIClpKp8sVwxCHHwJpR2vZus8+q1vP5qLFSDAzEYjFlpvN5UtNu6npJSAVnwgzqVIy4aCwgJDbI5ethhpjgqxuMFCBo9sdyMQxGuMiAA
% \begin{tikzcd}
%                                                                              & {\begin{matrix}A \times (A \to B) \\ (a, f)\end{matrix}} \arrow[ld, "p_A"', two heads] \arrow[rd, "p_{(A \to B)}", two heads] &                                                                      \\
% \begin{matrix} A \\ a \end{matrix} \arrow[rd, "\hat{f}"', hook, shift right] &                                                                                                                               & \begin{matrix} (A \to B) \\ f \end{matrix} \arrow[ld, "\iota", hook] \\
%                                                                              & {\begin{matrix}\Gamma_f \subseteq (A \times B) \\ (a, f(a)) \end{matrix}} \arrow[lu, "\pi_A"', two heads, shift right]          &                                                                     
% \end{tikzcd}
%Where $p_A$ and $p_B$ are the appropriate left and right projectors for the corresponding space.


%this one is much simpler and neater
% https://tikzcd.yichuanshen.de/#N4Igdg9gJgpgziAXAbVABwnAlgFyxMJZARgBoAGAXVJADcBDAGwFcYkQAdDgIxgHMsYYAFt6OAE5YAHgF8ABFwDi9YaID6AMwUc4zbnBg45AQW15h8OQCFtXOQAp6pORscBKN9phgoIsZNkQGVJ0TFx8QhRyUmJqOiZWdi5eASFRCWl5Uy47ei8fPwzA4NDsPAIiACYYuIYWNkROHn5BQoD5GxyOF3d833T2oLiYKD54IlANcQhhJDIQHAgkaPj6pI4ACzFgDRkQGjgNrA0cJABaeZx6LEZ2DYgIAGsgkJApmeWaRbmDo5Pz+YbGD0KDsHAAdwgQJBCBodUSjS4aCwamML0m01miEuS0Q1VWCLe+xAjHovEYAAUwuVIiBJHwNqcSm9MZ8Frj8dDQY0IVDgVBYSBDsdTni4QkGk1kWorEMZEA
\begin{tikzcd}
                                                                                            & {\begin{matrix} \Gamma_f \subseteq A \times B \\ (a, f(a)) \end{matrix}} \arrow[ld, "\pi_A", two heads, shift left] \arrow[rd, "\pi_B", two heads, shift right=2] &                                       \\
\begin{matrix} A \\ a \end{matrix} \arrow[ru, "\hat{f}", hook, shift left] \arrow[rr, "f"'] &                                                                                                                                                                 & \begin{matrix} B \\ f(a) \end{matrix}
\end{tikzcd}


PS: see "On sections and fibers" in the "notes" file for a worked example.



\subsection*{2.7)}

Let $f = (A \to B)$ be any function. Prove that the graph $\Gamma_f$ of $f$ is isomorphic to $A$.

Using the elements from the previous exercise, we know that $\hat{f}$ is injective from $A$ into $A \times B$. This property is inherited to any restriction of the codomain $Z \subseteq B$, and corresponding implied restriction of the domain to $Y = f^{-1}(Z) \subseteq A$. In particular, here, $Y = A$ and $Z = \Gamma_f = \hat{f}(A)$. We now consider $\overline{f} \in (A \to \Gamma_f), \overline{f} = (a \mapsto \hat{f}(a))$. We can see that $\overline{f}$ is injective from being a restriction of an injective function to a smaller codomain. We also know that $\overline{f}$ is surjective, since its domain is its image. Therefore, $\overline{f}$ is a bijection. This means that $A \simeq \Gamma_f$.



\subsection*{2.8)}

Describe as explicitly as you can all terms in the canonical decomposition of the function $f \in (\mathbb{R} \to \mathbb{C})$ defined by $f = (r \mapsto e^{2 \pi i r})$. (This exercise matches one assigned previously, which one?)

Firstly, elements of $\mathbb{R}$ are equivalent by this map (they have the same output) if they vary by $1$ from each other. This is a reference to the equivalence relation $\sim$ in exercise 1.6. Therefore, we will use $\mathbb{R}/\sim \simeq S^1$ in our decomposition. Obviously, the map from $(\mathbb{R} \to \mathbb{R}/\sim)$, which maps each element of $\mathbb{R}$ to respective their equivalence class is a surjection (since there's no empty equivalence class).

Secondly, as mentioned, we have a bijection $\tilde{f}$ between $\mathbb{R}/\sim$ and $S^1$, the circle group of unit complex numbers, namely $\tilde{f} = (x \mapsto e^{2 \pi i x}$, where each element $x$ of $\mathbb{R}/\sim$ can be understood to correspond to a (class representative) value in the interval $[0, 1[$.

Finally, we do the canonical injection of $S^1$ into its superset $\mathbb{C}$.



\subsection*{2.9)}

Show that if $A \simeq A'$ and $B \simeq B'$ , and further $A \cap B = \emptyset$ and $A' \cap B' = \emptyset$, then $A \cup B \simeq A' \cup B'$. Conclude that the operation $A \coprod B$ (as described in §1.4) is well-defined up to isomorphism.

We suppose the aforementioned.

Let $f_A$ be a bijection from $A \to A'$, and $f_B$ be a bijection from $B \to B'$.

We define the following:

$$
f \in (A \cup B \to A' \cup B'),
\text{ such that }
\begin{cases}
	\forall a \in A, \; f(a) = f_A(a) \\
	\forall b \in B, \; f(b) = f_B(b)
\end{cases}
$$

This function is a well-defined function, since $A \cap B = \emptyset$: every element of the domain has one, and only one, possible image.

Similarly, we define:

$$
g \in (A' \cup B' \to A \cup B),
\text{ such that }
\begin{cases}
	\forall a \in A', \; g(a) = f_A^{-1}(a) \\
	\forall b \in B', \; g(b) = f_B^{-1}(b)
\end{cases}
$$

Similarly, because $A' \cap B' = \emptyset$, $g$ is well-defined.

Let us study $g \circ f$. We have:
$$
\begin{cases}
	\forall a \in A, \; g(f(a)) = f_A^{-1}(f_A(a)) = a \\
	\forall b \in B, \; g(f(b)) = f_B^{-1}(f_B(b)) = b
\end{cases}
$$

Hence, $g \circ f = id_{A \cup B}$.
Similarly, $f \circ g = id_{A' \cup B'}$.
Therefore, $g = f^{-1}$, $f$ is a bijection, and $A \cup B \simeq A' \cup B'$.

We'll now do a shift in notation. Let be some arbitrary sets $A$ and $B$. Let be $A_1, A_2, B_1, B_2$ such that $A_1 = \{ 1 \} \times A$, $A_2 = \{ 2 \} \times A$, $B_1 = \{ 1 \} \times B$, and $B_2 = \{ 2 \} \times B$. This means $A \simeq A_1$, $A \simeq A_2$, $B \simeq B_1$, and $B \simeq B_2$. It also means $A_1 \cap A_2 = \emptyset$ and $B_1 \cap B_2 = \emptyset$. From the above, this implies $A_1 \cup B_1 \simeq A_2 \cup B_2$.

This means that the disjoint union of $A$ and $B$ is indeed well-defined, up to isomorphism: so long as 2 respective copies of $A$ and $B$ are made in a way that their intersection is empty, the 2 respective unions of 1 copy each will be isomorphic.



\subsection*{2.10)}

Show that if $A$ and $B$ are finite sets, then $|B^A| = |B|^{|A|}$.

The number of $|B^A|$ functions in $B^A = (A \to B)$ can be counted in the following way.

For each element $a$ of $A$, of which there are $|A|$, we can pick any element of $|B|$ as the image. This means $|B| \times ... \times |B|$, a total of $|A|$ times. Hence, $|B^A| = |B|^{|A|}$.



\subsection*{2.11)}

In view of Exercise 2.10, it is not unreasonable to use $2^A$ to denote the set of functions from an arbitrary set $A$ to a set with $2$ elements (say $\mathbb{B} = \{0, 1\}$). Prove that there is a bijection between $2^A$ and the power set $\mathcal{P}(A)$ of $A$.

Simply put, every subset $A_i$ of $A$ is built through a series of $|A|$ choices: for each element $a$ in $A$, do we keep the element $a$ in our subset $A_i$ (output $1$) or do we remove it (output $0$) ? It is then easy to see that such a series of choices can easily be encoded as a unique function in $A \to \mathbb{B}$. The totality of such series of choices thus corresponds both to the space $A \to \mathbb{B}$, and to the powerset $\mathcal{A}$, and there is a bijection between the two. 
