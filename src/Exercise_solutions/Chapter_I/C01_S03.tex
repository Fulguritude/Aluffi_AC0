\section*{Section 3)}

\subsection*{3.1)}

Let $\mathcal{C}$ be a category. Consider a structure $\mathcal{C}^{op}$ with:
 - $Obj(\mathcal{C}^{op}) \coloneqq Obj(\mathcal{C})$;
 - for $A$, $B$ objects of $\mathcal{C}^{op}$ (hence, objects of $\mathcal{C}$), $Hom_{\mathcal{C}^{op}} (A, B) \coloneqq Hom_{\mathcal{C}} (B, A)$
Show how to make this into a category.

\subsubsection*{3.1.a) Composition}

First, to make things clearer and more rigorous, let us distinguish composition in $\mathcal{C}$ as $\circ$ and composition in $\mathcal{C}^{op}$ as $\star$. We define $\star$ as:
$$
\begin{aligned}
	& \forall f \in Hom_{\mathcal{C}^{op}} (B, A) = Hom_{\mathcal{C}} (A, B), \\
	& \forall g \in Hom_{\mathcal{C}^{op}} (C, B) = Hom_{\mathcal{C}} (B, C), \\
	& \exists h \in Hom_{\mathcal{C}^{op}} (C, A) = Hom_{\mathcal{C}} (A, C), \\
	& f \star g \coloneqq g \circ f = h
\end{aligned}
$$

We will now show that $\mathcal{C}^{op}$ with $\star$ verifies the other axioms of a category (namely identity and assossiativity of composition).

\subsubsection*{3.1.b) Identity}

Since $\mathcal{C}$ is a category, since $\mathcal{C}^{op}$ has the same objects, and since, by definition, for all object $A$, we have $Hom_{\mathcal{C}^{op}} (A, A) = Hom_{\mathcal{C}} (A, A)$, we can take every $id_A \in Hom_{\mathcal{C}}(A, A)$ as the same identity in $\mathcal{C}^{op}$. We can verify that this is compatible with $\star$:

$$
\begin{aligned}
	\forall A, B & \in Obj (\mathcal{C})        &=& \;  Obj (\mathcal{C}^{op})        , \\
	\exists id_A & \in Hom_{\mathcal{C}} (A, A) &=& \;  Hom_{\mathcal{C}^{op}} (A, A) , \\
	\exists id_B & \in Hom_{\mathcal{C}} (B, B) &=& \;  Hom_{\mathcal{C}^{op}} (B, B) , \\
	\forall f    & \in Hom_{\mathcal{C}} (A, B) &=& \;  Hom_{\mathcal{C}^{op}} (B, A) , \\
	f            & =   f    \circ id_A          &=& \;  id_A \star f                  , \\
	f            & =   id_B \circ    f          &=& \;  f    \star id_B                 \\
\end{aligned}
$$

\subsubsection*{3.1.c) Associativity}

Using associativity in $\mathcal{C}$, we have:

$$
\begin{aligned}
	\forall A, B, C, D & \in Obj (\mathcal{C})        &=& \;  Obj (\mathcal{C}^{op})        , \\
	\forall f          & \in Hom_{\mathcal{C}} (A, B) &=& \;  Hom_{\mathcal{C}^{op}} (B, A) , \\
	\forall g          & \in Hom_{\mathcal{C}} (B, C) &=& \;  Hom_{\mathcal{C}^{op}} (C, B) , \\
	\forall h          & \in Hom_{\mathcal{C}} (C, D) &=& \;  Hom_{\mathcal{C}^{op}} (D, C) , \\
\end{aligned}
$$
$$
\begin{aligned}
	h \star (g \star f) &=&  h \star (f  \circ g) \\
						&=& (f \circ  g) \circ h  \\
						&=&  f \circ  (g \circ h) \\
						&=&  (g \circ h) \star f  \\
						&=&  (h \star g) \star f  \\
\end{aligned}
$$

Therefore, $\star$ is associative.

We conclude that $\mathcal{C}^{op}$ is a category.



\subsection*{3.2)}

If $A$ is a finite set, how large is $End_{\text{Set}}(A)$ ?

We know that, in Set, $End_{\text{Set}}(A) = (A \to A) = A^A$. From a previous exercise, we know that $|B^A| = |B|^|A|$, therefore $|End_{\text{Set}}(A)| = |A|^|A|$.



\subsection*{3.3)}

Formulate precisely what it means to say that "$1_a$ is an identity with respect to composition" in Example 3.3, and prove this assertion.

Example 3.3 is that of a category over a set $S$ with a (reflexive, transitive) relation $\sim$, where the objects of the category are the elements of $S$, and the homset between two elements $a$ and $b$ is the singleton $(a,b)$ if $a \sim b$, and $\emptyset$ otherwise. Composition $\circ$ is given by transitivity of $\sim$, where $(b,c) \circ (a,b) = (a,c)$. Reflexivity gives the identities ($id_a = (a,a)$ for any element $a$).

In this context, to say that "$1_a$ is an identity with respect to composition" means that we can cancel out an element of the form $(a,a)$ from a composition.

Formally, we have:

$$\forall a,b \in S, (b,b) \circ (a,b) = (a,b) = (a,b) \circ (a,a)$$

proving that $(b,b)$ is indeed a post-identity, and $(a,a)$ a pre-identity, in this context.



\subsection*{3.4)}

Can we define a category in the style of Example 3.3, using the relation $<$ on the set $\mathbb{Z}$ ?

(Description of example 3.3 in the exercise 3.3 just above.)

Naively, saying like in example 3.3 "there is a singleton homset $\text{Hom}(a,b)$ each time we have $a < b$", we cannot define such a category, since $<$ is not reflexive, and we would thus lack identity morphisms.

However, in a roundabout way, we can define a category over the \textit{negation} of $<$: "there is a singleton homset $\text{Hom}(a,b)$ each time we DO NOT have $a < b$". Namely this corresponds to the relation $\ge$, which is, itself, reflexive, transitive (and antisymmetric), and is a valid instance of the kind of category presented in example 3.3.

In fact, the pair $(\mathbb{Z}, \geq)$ is an instance of what is called a "totally ordered set", which is a more restrictive kind of "partially ordered set" (also called "poset" for short). Consequently, this kind of category is called a "poset category".



\subsection*{3.5)}

Explain in what sense Example 3.4 is an instance of the categories considered in Example 3.3.

(Description of example 3.3 in the exercise 3.3 just above.)

Example 3.4 describes a category $\hat{S}$ where the objects are the subsets of a set $S$ (equivalently: elements of the powerset $\mathcal{P}(S)$ of $S$), and morphisms between two subsets $A$ and $B$ of $S$ are singleton (or empty) homsets based on whether the inclusion is true (or false).

Inclusion of sets, $\subset$, is also an order relation, this time between the elements of a set of sets (here, $\mathcal{P}(S)$). This means inclusion is reflexive, transitive, and antisymmetric. This makes $\hat{S}$ a poset category, and thus another instance of example 3.3. 



\subsection*{3.6)}

Define a category $V$ by taking $Obj(V) = \mathbb{N}$, and $Hom_V(n, m) = Mat_\mathbb{R}(m, n)$, the set of $m \times n$ matrices with real entries, for all $n, m \in \mathbb{N}$. (I will leave the reader the task to make sense of a matrix with 0 rows or columns.) Use product of matrices to define composition. Does this
category 'feel' familiar ?

The formulation of the exercise is strange. It says to use the product of matrices to define composition, and to have homsets be sets of matrices, but objects of the category are supposed to be integers. I don't know of any matrix with real entries that maps an integer to an integer in this way.

We thus infer that the meaning of the exercise can be one of two things.

Either we suppose the set of objects could rather be understood as "something isomorphic to $\mathbb{N}$", ie, the collection of real vector spaces with finite bases (ie, $\forall n \in \mathbb{N}, \mathbb{R}^n$). In which case, this is just the category of real vector spaces with finite basis (and linear maps as morphisms), which is a subcategory of the category real vector spaces (commonly called $Vect_{\mathbb{R}}$). In this context, any morphism starting from $0 \simeq \mathbb{R}^0 = \{0\}$ is just the injection of the origin into the codomain; and any morphism ending at $0$ is the mapping of all elements to the origin.

Otherwise, we understand this as "yes, the objects of the category are integers: this means you should ignore the actual content of the matrices, and instead consider only their effect on the dimensionality of domains and codomains". In this case, this category is a complete directed graph over $\mathbb{N}$ where each edge corresponds to the change in dimension (from domain to codomain) caused by a given linear map.



\subsection*{3.7)}

Define carefully objects and morphisms in Example 3.7, and draw the diagram corresponding to composition.

Example 3.7 (on coslice categories) refers to example 3.5 (on slice categories). Let's go over slice categories (since example 3.5 asks the reader to "check all [their various properties]").

\subsubsection*{3.7.1) Slice categories}

Slice categories are categories made by singling out an object (say $A$) in some parent (larger) category (say $\mathcal{C}$), and studying all morphisms into that object. These morphisms become the objects of a new category (ie, for any $Z$ of $\mathcal{C}$, $f \in (Z \to A)$ is an object of the slice category, called $\mathcal{C}_A$ in this context). In the slice category, morphisms are defined as those morphism in $\mathcal{C}$ that preserve composition between 2 morphisms into $A$.

Note that there exist pairs of morphisms $f_1 \in (Z_1 \to A)$ and $f_2 \in (Z_2 \to A)$ between which there is no morphism that exists in the slice category. One such example we can make is in $(Vect_\mathbb{R})_{\mathbb{R}^2}$. If we take the maps:

$$f_1 = \begin{bmatrix} 1 & 0 \\ 0 & 0 \end{bmatrix} \in \mathcal{L}(\mathbb{R}^2)$$
$$f_2 = \begin{bmatrix} 0 & 0 \\ 0 & 1 \end{bmatrix} \in \mathcal{L}(\mathbb{R}^2)$$

There exists no map $\sigma$ such that the following diagram commutes (since the output of $f_1$ will always be null in its second coordinate, and the output of $f_2$ will always be null in the first):

% https://tikzcd.yichuanshen.de/#N4Igdg9gJgpgziAXAbVABwnAlgFyxMJZABgBpiBdUkANwEMAbAVxiRAB12BbOnACwBGA4ACUAvgD0ATCDGl0mXPkIoAjOSq1GLNpx78hoyTLkLseAkTKrN9Zq0QduvQcPHTZmmFADm8IqAAZgBOEFxIZCA4EEhS1HY6joEA+qog1Ax0AjAMAAqKFiogwVg+fDiy8iAhYUjqUTGIcVr2bCkmVTXhiJHRdfHaDk7YPjyeYkA
\begin{tikzcd}
\mathbb{R}^2 \arrow[d, "f_1"'] \arrow[r, "\sigma"] & \mathbb{R}^2 \arrow[ld, "f_2"] \\
\mathbb{R}^2                                       &                               
\end{tikzcd}

Now, let us prove that $\mathcal{C}_A$ is indeed a category for an arbitrary object $A$ of an arbitrary category $\mathcal{C}$.

3.7.1.a) Identity

A generic identity morphism is expressed diagrammatically in $\mathcal{C}_A$ as:

% https://tikzcd.yichuanshen.de/#N4Igdg9gJgpgziAXAbVABwnAlgFyxMJZABgBpiBdUkANwEMAbAVxiRAC0QBfU9TXfIRQBGclVqMWbTjz7Y8BImWHj6zVohABBbuJhQA5vCKgAZgCcIAWyRkQOCEgBM1NVM2mQ1BnQBGMBgAFfgUhEHMsAwALHG5eEAtrJFF7R0QXCXU2T1kEyxtEOwdk10kNECwoAH0ZeMSCjOL07wgINCVSU0Y4GHEffyCQwTYI6NjSrM1Kqp0uCi4gA
\begin{tikzcd}
Z \arrow[d, "f"'] \arrow[r, "id_Z"]                    & Z \arrow[ld, "f"] \\
A \arrow["id_A"', loop, distance=2em, in=305, out=235] &                  
\end{tikzcd}

We can see that since $f = f \circ id_Z$ in $\mathcal{C}$, this is compatible with the definition of a (pre-/right-)unit morphism in $\mathcal{C}_A$. Also, since the only maps post-$f$ are maps from $A \to A$, we have $id_A$ as the (post-/left-)unit for every morphism $f$ (ie, $f = id_A \circ f$. 

3.7.1.b) Composition

Taking 3 objects of the slice category ($f_1 \in (Z_1 \to A)$, $f_2 \in (Z_2 \to A)$ and $f_3 \in (Z_3 \to A)$), and two morphisms ($\sigma_A$ mapping $f_1$ to $f_2$ via a $\mathcal{C}$-morphism $\sigma \in (Z_1 \to Z_2)$, and $\tau_A$ mapping $f_2$ to $f_3$ via a $\mathcal{C}$-morphism $\tau \in (Z_2 \to Z_3)$), we have that $f_1 = f_2 \circ \sigma$ and $f_2 = f_3 \circ \tau$. This is expressed as the following commutative diagram.

% https://tikzcd.yichuanshen.de/#N4Igdg9gJgpgziAXAbVABwnAlgFyxMJZABgBpiBdUkANwEMAbAVxiRAC0B9ARhAF9S6TLnyEU3clVqMWbLgCZ+gkBmx4CReZOr1mrRB04BmJULWiiE7lN2yDAQX5SYUAObwioAGYAnCAFskMhAcCCQJEAY6ACMYBgAFYXUxSJgvHBAdGX0QAB1c7Fd-OlMQXwDw6lCkLWk9NnycOiZS8sDEYOrEIyz6gy8eTMiYuMTzDQMfLFcACwyBbz92iK6eursyzkUFsqWaqrDu3o2Bkz4KPiA
\begin{tikzcd}
Z_1 \arrow[r, "\sigma"] \arrow[rd, "f_1"'] & Z_2 \arrow[r, "\tau"] \arrow[d, "f_2"] & Z_3 \arrow[ld, "f_3"] \\
                                           & A                                      &                      
\end{tikzcd}

Composition of morphisms is then defined as $\tau_A \circ_A \sigma_A$ as a mapping from $f_1$ to $f_3$, such that $f_1 = f_3 \circ (\tau \circ \sigma)$. This can be understood through the following commutative diagram:

% https://tikzcd.yichuanshen.de/#N4Igdg9gJgpgziAXAbVABwnAlgFyxMJZABgBpiBdUkANwEMAbAVxiRAC0B9ARhAF9S6TLnyEUAJnJVajFmy4BmfoJAZseAkW6lu0+s1aIQAQX7SYUAObwioAGYAnCAFskZEDghJJMg2zs8INQMdABGMAwACsIaYiAOWJYAFjjK9k6uiNoeXog++nJGAUoC6S5u1J5I2QWGIAA69Th0TI0AxlgObY3Yls50ZnxAA
\begin{tikzcd}
Z_1 \arrow[rd, "f_1"'] \arrow[rr, "\tau \circ \sigma"] &   & Z_3 \arrow[ld, "f_3"] \\
                                                       & A &                      
\end{tikzcd}

Which commutes, because we have:

$$
\begin{aligned}
	f_1 &=&  f_2              \circ \sigma  \\
		&=& (f_3 \circ  \tau) \circ \sigma  \\
		&=&  f_3 \circ (\tau  \circ \sigma)
\end{aligned}
$$

Thus, we have a working composition of morphisms.

3.7.1.c) Associativity

We take 4 objects of the slice category ($f_1 \in (Z_1 \to A)$, $f_2 \in (Z_2 \to A)$, $f_3 \in (Z_3 \to A)$ and  $f_4 \in (Z_4 \to A)$), and three morphisms ($\sigma_A$ mapping $f_1$ to $f_2$, $\tau_A$ mapping $f_2$ to $f_3$, and $\upsilon_A$ mapping $f_3$ to $f_4$). Using composition defined as above, we have

$$
\begin{aligned}
f_1 &=& f_4 \circ ( \upsilon \circ (\tau  \circ \sigma)) \\
    &=& f_4 \circ ((\upsilon \circ  \tau) \circ \sigma ) \\
\Rightarrow && \\
& &  \upsilon_A \circ (\tau_A  \circ \sigma_A) \\
&=& (\upsilon_A \circ  \tau_A) \circ \sigma_A
\end{aligned}
$$

Through associativity in $\mathcal{C}$.


\subsubsection*{3.7.2) Coslice categories}

A coslice category $\mathcal{C}^A$ is similar, except it takes the morphisms coming \textit{from} a chosen object $A$, rather than those going \textit{to} this object $A$. Below is a commutative diagram in the style of the one of the textbook for slice categories.

% https://tikzcd.yichuanshen.de/#N4Igdg9gJgpgziAXAbVABwnAlgFyxMJZARgBoAGAXVJADcBDAGwFcYkQBBEAX1PU1z5CKcqWLU6TVuwBaAfWI8+IDNjwEiAJjESGLNohDzNPCTCgBzeEVAAzAE4QAtklEgcEJGRCN6AIxhGAAUBdWEQeywLAAscEBo9aUNbBSU7RxdEbw8kbUl9dgAdQuwLJ3o0kAdnVxocxDzEgyq5E25KbiA
\begin{tikzcd}
                         & A \arrow[ld, "f_1"'] \arrow[rd, "f_2"] &     \\
Z_1 \arrow[rr, "\sigma"] &                                        & Z_2
\end{tikzcd}

We can similarly show that this also defines a category.

\vspace{5mm}
\underline{3.7.2.a) Identity}

A generic identity morphism is expressed diagrammatically in $\mathcal{C}^A$ as:

% https://tikzcd.yichuanshen.de/#N4Igdg9gJgpgziAXAbVABwnAlgFyxMJZABgBpiBdUkANwEMAbAVxiRAEEQBfU9TXfIRQBGUsKq1GLNgC1uvEBmx4CRMuOr1mrRCDlcJMKAHN4RUADMAThAC2SMiBwQkoydrYWQ1BnQBGMAwACvwqQiAMMBY48pY29oiOzkgATJpSOiBePHF2qdTJiG6+AcGhgmxWWMYAFjHpHrpYUAD6+grWeYkFLt0REBBoRMIAHGQWjHAwEiWBIcoVulW19e7STa2cBlxAA
\begin{tikzcd}
A \arrow[rd, "f"] \arrow[d, "f"] \arrow["id_A"', loop, distance=2em, in=125, out=55] &   \\
Z \arrow[r, "id_Z"']                                                                 & Z
\end{tikzcd}

We can see that since $f = id_Z \circ f$ in $\mathcal{C}$, this is compatible with the definition of a (post-/left-)unit morphism in $\mathcal{C}^A$. Also, since the only maps pre-$f$ are maps from $A \to A$, we have $id_A$ as the (pre-/right-)unit for every morphism $f$ (ie, $f = f \circ id_A$. 

\vspace{5mm}
\underline{3.7.2.b) Composition}

Taking 3 objects of the slice category ($f_1 \in (A \to Z_1)$, $f_2 \in (A \to Z_2)$ and $f_3 \in (A \to Z_3)$), and two morphisms ($\sigma^A$ mapping $f_1$ to $f_2$ via a $\mathcal{C}$-morphism $\sigma \in (Z_1 \to Z_2)$, and $\tau^A$ mapping $f_2$ to $f_3$ via a $\mathcal{C}$-morphism $\tau \in (Z_2 \to Z_3)$), we have that $f_1 = \sigma \circ f_2$ and $f_2 = \tau  \circ f_3$. This is expressed as the following commutative diagram.

% https://tikzcd.yichuanshen.de/#N4Igdg9gJgpgziAXAbVABwnAlgFyxMJZABgBoBGAXVJADcBDAGwFcYkQAtAfXJAF9S6TLnyEU5CtTpNW7bgCZ+gkBmx4CReZJoMWbRJy4BmJULWiiE4lN2yDAQX5SYUAObwioAGYAnCAFskMhAcCCQJaT12AB1o7Fd-ehAaRnoAIxhGAAVhdTEQRhgvHFMQXwDwmlCkLUi7EFicemZkgvTMnPMNA0Li0vLAxCMqsMRg230ynlbUjOzciwMfLFcACxKBbz9B4ZDRiIn2Ly5FTbLtpF3qxFrDg2OTPko+IA
\begin{tikzcd}
                        & A \arrow[ld, "f_1"'] \arrow[d, "f_2"] \arrow[rd, "f_3"] &     \\
Z_1 \arrow[r, "\sigma"] & Z_2 \arrow[r, "\tau"]                                   & Z_3
\end{tikzcd}

Composition of morphisms is then defined as $\tau^A \circ^A \sigma^A$ as a mapping from $f_1$ to $f_3$, such that $f_3 = (\tau \circ \sigma) \circ f_1$. This can be understood through the following commutative diagram:

% https://tikzcd.yichuanshen.de/#N4Igdg9gJgpgziAXAbVABwnAlgFyxMJZABgBoBGAXVJADcBDAGwFcYkQAtAfXJAF9S6TLnyEUAJgrU6TVu24BmfoJAZseAkXKli0hizaIQAQX7SYUAObwioAGYAnCAFskZEDghJtMg+wA6-jj0zAAEgQDGWA4R4f7Yls70IDSM9ABGMIwACsIaYiCMMHY4yvZOroiSHl6I7mmZOXmi7A5YlgAWpTT6ckZ2PGUgji5I1Z7ePbKGw1xKfJR8QA
\begin{tikzcd}
                                    & A \arrow[ld, "f_1"'] \arrow[rd, "f_3"] &     \\
Z_1 \arrow[rr, "\tau \circ \sigma"] &                                        & Z_3
\end{tikzcd}

Which commutes, because we have:

$$
\begin{aligned}
	f_3 &=&  \tau \circ                f_2  \\
		&=&  \tau \circ (\sigma  \circ f_1) \\
		&=& (\tau \circ  \sigma) \circ f_1
\end{aligned}
$$

Thus, we have a working composition of morphisms.

\vspace{5mm}
\underline{3.7.2.c) Associativity}

We take 4 objects of the slice category ($f_1 \in (A \to Z_1)$, $f_2 \in (A \to Z_2)$, $f_3 \in (A \to Z_3)$ and  $f_4 \in (A \to Z_4)$), and three morphisms ($\sigma^A$ mapping $f_1$ to $f_2$, $\tau^A$ mapping $f_2$ to $f_3$, and $\upsilon^A$ mapping $f_3$ to $f_4$). Using composition defined as above, we have

$$
\begin{aligned}
f_4 &=& ( \upsilon \circ (\tau  \circ \sigma)) \circ f_1 \\
	&=& ((\upsilon \circ  \tau) \circ \sigma ) \circ f_1 \\
\Rightarrow && \\
    & &  \upsilon^A \circ (\tau^A  \circ \sigma^A) \\
    &=& (\upsilon^A \circ  \tau^A) \circ \sigma^A
\end{aligned}
$$

Through associativity in $\mathcal{C}$.


\subsection*{3.8)}

A subcategory $\mathcal{C'}$ of a category $\mathcal{C}$ consists of a collection of objects of $\mathcal{C}$, with morphisms $Hom_\mathcal{C'} (A, B) \subseteq Hom_\mathcal{C} (A, B)$ for all objects $A$, $B$ in $Obj(\mathcal{C'})$, such that identities and compositions in $\mathcal{C}$ make $\mathcal{C'}$ into a category. A subcategory $\mathcal{C'}$ is \textit{full} if $Hom_\mathcal{C'} (A, B) = Hom_\mathcal{C} (A, B)$ for all $A$, $B$ in $Obj(\mathcal{C'})$. Construct a category of \textit{infinite sets} and explain how it may be viewed as a full subcategory of $\mathbf{Set}$.

To put it less technically, a "subcategory" $\mathcal{C'}$ is just "picking only certain items of a base category $\mathcal{C}$, and making sure that things stay closed uneder morphism composition". It is "full" if \textit{all} morphisms between the objects that remain are also conserved.

We can construct a category $\mathbf{InfSet}$ of infinite sets by taking all the objects $A$ of $\mathbf{Set}$ such that $\nexists n \in \mathbb{N}, |A| = n$, and only homsets between these objects. This is clearly a subcategory of $\mathbf{Set}$, since it inherits all identity morphisms, composition works the same, and so does associativity; also, restricting the choice of homsets makes it so that the category is closed (you can't reach a finite set via a homset that went from an infinite to a finite set).

For this category to not be full, there would need to be some homset that loses a morphism, or fully disappears, in the ordeal. However, there is no restriction as to the kind of morphism that is conserved, so any homset that is kept is identical to its original version. Finally, homsets between infinite sets are also infinite sets, so they don't disappear in this operation.

Consequently $\mathbf{InfSet}$ defined as such is a full subcategory of $\mathbf{Set}$.


\subsection*{3.9)}

An alternative to the notion of multiset introduced in §2.2 is obtained by considering sets endowed with equivalence relations; equivalent elements are taken to be multiple instances of elements 'of the same kind'. Define a notion of morphism between such enhanced sets, obtaining a category $\mathbf{MSet}$ containing (a 'copy' of) $\mathbf{Set}$ as a full subcategory. (There may be more than one reasonable way to do this! This is intentionally an open-ended exercise.) Which objects in $\mathbf{MSet}$ determine ordinary multisets as defined in §2.2, and how? Spell out what a morphism of multisets would be from this point of view. (There are several natural notions of morphisms of multisets. Try to define morphisms in MSet so that the notion you obtain for ordinary multisets captures your intuitive understanding of these objects.) [§2.2, §3.2, 4.5]

Let us recall how multisets were defined in §2.2. Since duplicate elements do not exist in sets, multisets were instead defined as functions from a set $S$ to $\mathbb{N}*$, the set of (nonzero) positive integers. This allows each element in $S$ to have a "count", thereby encoding the intuitive notion of multiset. A similar, and equivalent (isomorphic), way of defining it is \textit{via} pairs $(s, n) \in S \times \mathbb{N}*$, which is simpler to think about. We'll call this category $\mathbf{CMSet}$, for "count multiset" (TODO: probably has a conventional and better name, but I don't know it). As for morphisms in $\mathbf{CMSet}$, we can consider that for any multisets $A = S_A \times \mathbb{N}*$ and $B = S_B \times \mathbb{N}*$, the homset from $A$ to $B$ is simply the set functions from $S_A \times \mathbb{N}*$ to $S_B \times \mathbb{N}*$ as usual.

We first notice that if we restrict $\mathbf{CMSet}$ to only the objects for which all elements have a count of $1$, and where morphisms only ever output to $\{ 1 \}$ in the second coordinate (a subcategory we can call $\mathbf{C1MSet}$, for example), we get a "copy" of $\mathbf{Set}$: $\mathbf{C1MSet}$ and $\mathbf{Set}$ are isomorphic. This is a full subcategory because there are no morphisms that map counts to anything else than $\{ 1 \}$ if we restrict our objects to this form; so all morphisms between the kept objects are also kept. 

Now let us do a similar construction, but based on equivalence classes instead. We know that each equivalence class over a set corresponds uniquely to a partition of that set. By considering only these partitions (these "sets of sets") as objects, we can build a category $\mathbf{EMSet}$ (for "equivalence multiset"). The "count" corresponds simply to the cardinal of a top-level element in the partition. For example, the top-level elements of $M = \{ S_1, S_2, S_3 \}= \{ \{a\}, \{b, c\}, \{d, e, f\} \}$ would be understood to have counts $|S_1| = 1$, $|S_2| = 2$ and $|S_3| = 3$ respectively.

As for morphisms in $\mathbf{EMSet}$, they simply map each top-level element of the domain multiset (a distinct subset of the original set) to some other top-level elements in the codomain multiset. This has precisely the same effect as mapping pairs of "value and count" as seen in the previous $\mathbf{CMSet}$ construction.

In this example, any set itself, when "injected" (by a functor) into $\mathbf{EMSet}$ would just nest all of its elements into singletons. I.e., $S = \{ a, b, c \}$ in $\mathbf{Set}$ would become $S = \{ \{a\}, \{b\}, \{c\} \}$ in $\mathbf{EMSet}$. This also shows how restricting $\mathbf{EMSet}$ to "only objects that are a set of (toplevel) singletons" makes $\mathbf{EMSet}$ have a "copy" of $\mathbf{Set}$ as a full subcategory (for similar arguments as above).



\subsection*{3.10)}

Since the objects of a category $\mathcal{C}$ are not (necessarily) sets, it is not clear how to make sense of a notion of 'subobject' in general. In some situations it does make sense to talk about subobjects, and the subobjects of any given object $A$ in $\mathcal{C}$ are in one-to-one correspondence with the morphisms $A \to \Omega$ for a fixed, special object $\Omega$ of $\mathcal{C}$, called a subobject classifier. Show that $\mathbf{Set}$ has a subobject classifier.

We define the set $\mathbb{B} = \{ 0, 1 \}$, aka the binary alphabet or booleans, as the subobject classifier of $\mathbf{Set}$. For any subset $A$ of $B$, there is a unique map $f: B \to \mathbb{B}$, such that $\forall b \in B, f(b) = 1 \Leftrightarrow b \in A$ (otherwise $f(b) = 0$, of course, as the equivalence and lack of alternatives to $0$ as an output imply). The map $f$ always fully describes $A$ from its relationship with $B$.



\subsection*{3.11)}

Draw the relevant diagrams and define composition and identities for the category $\mathcal{C}^{A,B}$ mentioned in Example 3.9. Do the same for the category $\mathcal{C}^{\alpha, \beta}$ mentioned in Example 3.10. [§5.5, 5.12]

For lack of a better term, we will refer to the categories of the form $\mathcal{C}_{A,B}$ represented by Example 3.9 as "bi-slice categories". The first part of the exercise is thus asking us to define and explain what "bi-coslice categories" (of the form $\mathcal{C}^{A,B}$) are.

Similarly, we will refer to the categories of the form $\mathcal{C}_{\alpha, \beta}$ represented by Example 3.10 as "fibered bi-slice categories". The second part of the exercise is thus asking us to define and explain what "fibered bi-coslice categories" (of the form $\mathcal{C}^{\alpha, \beta}$) are.

We will, of course, attempt to make more formal and pedagogical all definitions broached in the textbook's examples as well.


\subsubsection*{3.11.1) Bi-slice categories}

\vspace{5mm}
\underline{3.11.1.a) Objects and morphisms}

To make a bi-slice category $\mathcal{C}_{A,B}$, we pick 2 objects $A$ and $B$ of a base category $\mathcal{C}$, and consider for all other objects $Z$ of $\mathcal{C}$, all pairs of morphisms $(f, g) \in (Z \to A) \times (Z \to B)$. These pairs of morphisms are the objects of the bi-slice category $\mathcal{C}_{A,B}$. Morphisms $\sigma_{A,B}$ are defined from an object $p_1 = (f_1, g_1) \in (Z_1 \to A) \times (Z_1 \to B)$ to an object $p_2 = (f_2, g_2) \in (Z_2 \to A) \times (Z_2 \to B)$ so that we have both $f_1 = f_2 \circ \sigma$ and $g_1 = g_2 \circ \sigma$, for some $\sigma \in (Z_1 \to Z_2)$.

A generic object in $\mathcal{C}_{A,B}$ is of the form:

% https://tikzcd.yichuanshen.de/#N4Igdg9gJgpgziAXAbVABwnAlgFyxMJZARgBoAGAXVJADcBDAGwFcYkQBBEAX1PU1z5CKcqWLU6TVuwBaPPiAzY8BImQBMEhizaIQAIR4SYUAObwioAGYAnCAFskZEDghJRkneysgajegBGMIwACgIqwiA2WKYAFjjy1naOiM6uSOo02tJ6pr4g-kGh4ULsjDBWCdyU3EA
\begin{tikzcd}
                                   & A \\
Z \arrow[ru, "f"'] \arrow[rd, "g"] &   \\
                                   & B
\end{tikzcd}


\vspace{5mm}
\underline{3.11.1.b) Morphisms}

Morphisms are defined between objects as

% https://tikzcd.yichuanshen.de/#N4Igdg9gJgpgziAXAbVABwnAlgFyxMJZAFgBoBGAXVJADcBDAGwFcYkQAtAfQCYQBfUuky58hFOVIAGanSat2AQQFCQGbHgJFJPWQxZtEIAEIrhGsUSkU98w5y7kzakZvHIArNNsGlz9aJaKF66NPoKRqaC5oHuPDZhduz+rpYoAMwJcr5GArIwUADm8ESgAGYAThAAtkiZIDgQSJLZESBljiA0jPQARjCMAAqpQSAVWIUAFjhdIHCTWGUziFLR7VW1iPWNSPGt9oWd3X0DwxajjDBLzpU1SNYNTYhe++yHfMf9QyPiIJfXa1umweO0QZFeRg6fEBGyQADYaKCAOyJHIgAA66OwhWq9C4wEUpGM-Dy-CAA
\begin{tikzcd}
                                         & A &                                &    &                                         & A \\
Z_1 \arrow[ru, "f_1"'] \arrow[rd, "g_1"] &   & {} \arrow[r, "{\sigma_{A,B}}"] & {} & Z_2 \arrow[rd, "g_2"] \arrow[ru, "f_2"] &   \\
                                         & B &                                &    &                                         & B
\end{tikzcd}

such that the following diagram commutes

% https://tikzcd.yichuanshen.de/#N4Igdg9gJgpgziAXAbVABwnAlgFyxMJZARgBoAGAXVJADcBDAGwFcYkQBBEAX1PU1z5CKcqWLU6TVuwBaAfWI8+IDNjwEiZcTQYs2iEPIBMS-mqGbSRibukGAQjwkwoAc3hFQAMwBOEALZIZCA4EEiiknrsXgogNIz0AEYwjAAKAurCIIwwXjimIL4BQTShSEY6UvogADo12K7+9HHZSSnp5hoGOXkFRYGIwWWIAMyVUQausfFtaRkWBj5YrgAW+bzefgMVIWGIEbbVMSYzyXOdWUur68r95aV7Y5F2IFMnrWcdgl3ZueuU3CAA
\begin{tikzcd}
                                                             & A                                      \\
Z_1 \arrow[ru, "f_1"] \arrow[r, "\sigma"] \arrow[rd, "g_1"'] & Z_2 \arrow[u, "f_2"'] \arrow[d, "g_2"] \\
                                                             & B                                     
\end{tikzcd}


\vspace{5mm}
\underline{3.11.1.c) Identity}

It is clear that identity morphisms exist for all objects, simply by taking $Z = Z_1 = Z_2$, $f_1 = f_2$, $g_1 = g_2$ and $\sigma = id_Z$, in the diagram above.


\vspace{5mm}
\underline{3.11.1.d) Composition}

Let be 3 objects of $\mathcal{C}_{A,B}$, which we will name $p_1$, $p_2$ and $p_3$ (and define with the respective $(Z_n, f_n, g_n)$ triplet for $p_n$).

Composition $\tau_{A, B} \circ \sigma_{A, B} = p_1 \mapsto p_3$ of two morphisms $\sigma_{A, B} = p_1 \mapsto p_2$ and $\tau_{A, B} = p_2 \mapsto p_3$ is defined so that the following diagram commutes.

% https://tikzcd.yichuanshen.de/#N4Igdg9gJgpgziAXAbVABwnAlgFyxMJZARgBoAGAXVJADcBDAGwFcYkQBBEAX1PU1z5CKcqWLU6TVuwBaAfWI8+IDNjwEiZcTQYs2iEPIBMS-mqFEjYibukH5AZlMqB64SVJGbU-SABCPBIwUADm8ESgAGYAThAAtkhWIDgQSKKSeuyRciY0jPQARjCMAAquFgbRWCEAFjjOMfFIZMmpiOm2vtmKeYXFZeYaBowwkfW8UbEJiA40KWk6PllyTr1FpeVDIFW148qN0y3ziEmd7AA659ghcfQga-2bwiAjYw1TiXNtsxl2IJc4ejMe4vPobQbPV57SZNE5fJAAFkWmQMIRyIPy6wGgi2UPesKObSRv18aJ6oKxT3YOzq+OmP2OxLOqJWGLB2Lc7Dx3Eo3CAA
\begin{tikzcd}
                                                             & A                                                        &                                          \\
Z_1 \arrow[ru, "f_1"] \arrow[r, "\sigma"] \arrow[rd, "g_1"'] & Z_2 \arrow[u, "f_2"'] \arrow[r, "\tau"] \arrow[d, "g_2"] & Z_3 \arrow[lu, "f_3"'] \arrow[ld, "g_3"] \\
                                                             & B                                                        &                                         
\end{tikzcd}


\vspace{5mm}
\underline{3.11.1.e) Associativity}

Associativity follows from associativity of morphisms in $\mathcal{C}$, similarly to what was done for slice categories in exercise 3.7 .



\subsubsection*{3.11.2) Bi-coslice categories}

\vspace{5mm}
\underline{3.11.2.a) Objects and morphisms}

To make a bi-coslice category $\mathcal{C}^{A,B}$, we similarly pick 2 objects $A$ and $B$ of our base category $\mathcal{C}$, but instead consider, for all other objects $Z$ of $\mathcal{C}$, all pairs of morphisms $(f, g) \in (A \to Z) \times (B \to Z)$.

A generic object in $\mathcal{C}^{A,B}$ is of the form:

% https://tikzcd.yichuanshen.de/#N4Igdg9gJgpgziAXAbVABwnAlgFyxMJZARgBpiBdUkANwEMAbAVxiRAC0QBfU9TXfIRQAGUsKq1GLNgEFuvEBmx4CRUQCYJ9Zq0QgAQtwkwoAc3hFQAMwBOEALZIyIHBCSjJOtlfnW7jxHVqV3dqbWk9UxBqBjoAIxgGAAV+FSEQGyxTAAscIy4gA
\begin{tikzcd}
A \arrow[rd, "f"]  &   \\
                   & Z \\
B \arrow[ru, "g"'] &  
\end{tikzcd}


\vspace{5mm}
\underline{3.11.2.b) Morphisms}

Morphisms are defined between objects as

% https://tikzcd.yichuanshen.de/#N4Igdg9gJgpgziAXAbVABwnAlgFyxMJZAVgBoBGAXVJADcBDAGwFcYkQAtAfQCYQBfUuky58hFAAZSE6nSat2AQQFCQGbHgJEpPWQxZtEIAEIrhGsUXIU98w5y7kzakZvHIALNNsGlz9aJaKF66NPoKRqaC5oHuPDZhduz+rpYoAMwJcr5GArIwUADm8ESgAGYAThAAtkjWIDgQSJnZESBljiA0jPQARjCMAAqpQSCMMGU4zpU1SPENTYgt4faFnd19A8MWoxVYhQAWU9HtVbWIZAtIUq2rvF1jm0Mj4iB7h8eqM+deV4g3K3YHT4J2+SAAbDRGkgAOyJHIgAA6iOwhWq9AAesBFKRjPw8vwgA
\begin{tikzcd}
A \arrow[rd, "f_1"]  &     &                                &    & A \arrow[rd, "f_2"]  &     \\
                     & Z_1 & {} \arrow[r, "{\sigma^{A,B}}"] & {} &                      & Z_2 \\
B \arrow[ru, "g_1"'] &     &                                &    & B \arrow[ru, "g_2"'] &    
\end{tikzcd}

such that the following diagram commutes

% https://tikzcd.yichuanshen.de/#N4Igdg9gJgpgziAXAbVABwnAlgFyxMJZABgBpiBdUkANwEMAbAVxiRAEEQBfU9TXfIRRkAjFVqMWbAFoB9Ed14gM2PASIjSY6vWatEIOQCZFfVYKJkj43VIMAhbuJhQA5vCKgAZgCcIAWyQyEBwIJCMdSX0QL1kTagY6ACMYBgAFfjUhEAYYLxxTGL9AxGDQpE0JPTZYhQTk1IzzdQMfLFcACwKeb2KK6nLECKq7EAAdMexXfzoQepT0zIsDXPzC3wCkAGYBsKHI6oNXOLmchsXm7LbO7qUNkp2QvcrEhaaBFpy8goPR44UuBQuEA
\begin{tikzcd}
A \arrow[rd, "f_2"] \arrow[d, "f_1"'] &     \\
Z_1 \arrow[r, "\sigma"]               & Z_2 \\
B \arrow[ru, "g_2"'] \arrow[u, "g_1"] &    
\end{tikzcd}


\vspace{5mm}
\underline{3.11.2.c) Identity}

It is clear that identity morphisms exist for all objects, simply by taking $Z = Z_1 = Z_2$, $f_1 = f_2$, $g_1 = g_2$ and $\sigma = id_Z$, in the diagram above.

\vspace{5mm}
\underline{3.11.2.d) Composition}

Let be 3 objects of $\mathcal{C}^{A,B}$, which we will name $p_1$, $p_2$ and $p_3$ (and define with the respective $(Z_n, f_n, g_n)$ triplet for $p_n$).

Composition $\tau^{A, B} \circ \sigma^{A, B} = p_1 \mapsto p_3$ of two morphisms $\sigma^{A, B} = p_1 \mapsto p_2$ and $\tau^{A, B} = p_2 \mapsto p_3$ is defined so that the following diagram commutes.

% https://tikzcd.yichuanshen.de/#N4Igdg9gJgpgziAXAbVABwnAlgFyxMJZARgBoAGAXVJADcBDAGwFcYkQBBEAX1PU1z5CKcqWLU6TVuwBaAfWI8+IDNjwEiZcTQYs2iEPIBMS-mqGbSRibukGAQqZUD1w5EbE2p+w3IDMPBIwUADm8ESgAGYAThAAtkiiIDgQSB6SeuyRciY0jPQARjCMAAouFgaMMJE4TjHxiTQpSGQZdiDZinmFxWXmGgbRWCEAFrW8UbEJiK3NiOm2PgA6S9ghcfQg3UWl5QMgVTV1U0h+TanzOt7sITlbBz27-cIgQ6PjyvXTZ8kXrYs3BT3fI7PqCfaHD6TBqXX5IAAsV0yBhWOHozGOMKSc0RbR82QCEw6J0QPxxSPatwC216exebzGgW4QA
\begin{tikzcd}
                        & A \arrow[d, "f_2"] \arrow[ld, "f_1"'] \arrow[rd, "f_3"]  &     \\
Z_1 \arrow[r, "\sigma"] & Z_2 \arrow[r, "\tau"]                                    & Z_3 \\
                        & B \arrow[u, "g_2"'] \arrow[lu, "g_1"] \arrow[ru, "g_3"'] &    
\end{tikzcd}

\vspace{5mm}
\underline{3.11.2.e) Associativity}

Associativity follows from associativity of morphisms in $\mathcal{C}$, similarly to what was done for slice categories in exercise 3.7 .



\subsubsection*{3.11.3) Fibered bi-slice categories}

\vspace{5mm}
\underline{3.11.3.a) Objects}

To build a fibered bi-slice category $\mathcal{C}_{\alpha, \beta}$, one takes a base category $\mathcal{C}$, as well as a fixed pair of morphisms $\alpha : A \to C$ and $\beta : B \to C$ in $\mathcal{C}$, that point to a common object $C$ of $\mathcal{C}$. Our basic "fixed construct" from $\mathcal{C}$ looks like so: 

% https://tikzcd.yichuanshen.de/#N4Igdg9gJgpgziAXAbVABwnAlgFyxMJZABgBpiBdUkANwEMAbAVxiRAEEQBfU9TXfIRQBGUsKq1GLNgGFuvEBmx4CRMgCYJ9Zq0QgAQtwkwoAc3hFQAMwBOEALZIyIHBCSjJOtgB1vjNAAWdPLWdo6I6tSu7tTa0nq+AEYwOMHUDHTJDAAK-CpCIDZYpgE4RlxAA
\begin{tikzcd}
A \arrow[rd, "\alpha"] &   \\
                       & C \\
B \arrow[ru, "\beta"'] &  
\end{tikzcd}

The role of the category $\mathcal{C}_{\alpha, \beta}$ is now to study the morphisms into this construct. A generic object from this new category looks like so:

% https://tikzcd.yichuanshen.de/#N4Igdg9gJgpgziAXAbVABwnAlgFyxMJZARgBoAGAXVJADcBDAGwFcYkQBBEAX1PU1z5CKAEyli1Ok1bsAwjz4gM2PASJkRkhizaIQAIQX8VQouXFbpukAC0ekmFADm8IqABmAJwgBbJOZAcCCQyKR12AB0IpjQAC3ojEC9fJDFA4MRQ7Rk9KIAjGBwEmkZ6AsYABQFVYRBPLCdYnETkv0QAZhog-xps63cW7zbO9NSSsphK6tM9esbm3qt2J3tuIA
\begin{tikzcd}
                                   & A \arrow[rd, "\alpha"] &   \\
Z \arrow[ru, "f"] \arrow[rd, "g"'] &                        & C \\
                                   & B \arrow[ru, "\beta"'] &  
\end{tikzcd}

such that the diagram commutes. This means that valid object in $\mathcal{C}_{\alpha, \beta}$ are triplets $(Z, f, g)$, with $f : Z \to A$ and $g : Z \to B$, such that $\alpha \circ f = \beta \circ g$. In a caricatural way, this boils down to studying "the comparison of the different paths one can use to reach $C$, knowing that the last steps are on one hand, $\alpha$, and on the other, $\beta$".

\vspace{5mm}
\underline{3.11.3.b) Morphisms}

Morphisms are defined between objects as:

% https://tikzcd.yichuanshen.de/#N4Igdg9gJgpgziAXAbVABwnAlgFyxMJZARgBoAGAXVJADcBDAGwFcYkQBBEAX1PU1z5CKAEyli1Ok1bsAwjz4gM2PASJkRkhizaIQAIQX8VQouXFbpukAC0A+sSNKBq4cgDMFmtpl6nywTUUABYvKR12fxdTFABWMJ9rexEokyDkADYKSwi9Ll5jQLcAdgSrOVSioizNb3K9Q25JGCgAc3giUAAzACcIAFskcxAcCCQycN8QAB1ppjQAC3onXoGkMRGxxAnE9lmAIxgcZZpGekPGAAVooJAerFaFnBW+wcRPTaG63JAuhxe1u8aKN1qdzjArjdhHcHk8QN8pq1-gVfq8kKFPoh4pNrLNsK1+vQ7MBZvMlqQDkd6NwAW8spjSjj2H8UijVm9GSDEAAOBG4uaMRbLNloxAATmBW15TL0lOO8JAZwu1zS0Puj2eIsB9K5EplICRKTByqh7HVcKa3CAA
\begin{tikzcd}
                                         & A \arrow[rd, "\alpha"] &   &                                         &    &                                          & A \arrow[rd, "\alpha"] &   \\
Z_1 \arrow[ru, "f_1"] \arrow[rd, "g_1"'] &                        & C & {} \arrow[r, "{\sigma_{\alpha,\beta}}"] & {} & Z_2 \arrow[ru, "f_2"] \arrow[rd, "g_2"'] &                        & C \\
                                         & B \arrow[ru, "\beta"'] &   &                                         &    &                                          & B \arrow[ru, "\beta"'] &  
\end{tikzcd}

such that the following diagram commutes

% https://tikzcd.yichuanshen.de/#N4Igdg9gJgpgziAXAbVABwnAlgFyxMJZARgBpiBdUkANwEMAbAVxiRAC0B9AJhAF9S6TLnyEU3UgAYqtRizYBBfoJAZseAkQDM5GfWatEIAMLKh60UQnc9cwyABCZ1cI1jkk3dX3yjXYvwyMFAA5vBEoABmAE4QALZIniA4EEhksgZskTwg1Ax0AEYwDAAKrpZG0VghABY4zjHxadQpSBIZviAAOl2MaDV0DbEJiDrJqYjtPvY9RTiDeYXFZRaaldV1Q02ISa2j3nZsITmLRaXlayAMMJH1AlHDSAAsLRPp01mcAdRwNVi3SAAtNx7iBGiMXuMkGMPkZjt8QL9-vVJqdlhcxCAqrU7ipwc9XokDpkjD1sCE4gsrktzqtMVgwNhYLlEX8ATs+BQ+EA
\begin{tikzcd}
                                                                                                        &                                          & A \arrow[rd, "\alpha"] &   \\
Z_1 \arrow[rru, "f_1", shift left=2] \arrow[rrd, "g_1"', shift right=2] \arrow[r, "\sigma" description] & Z_2 \arrow[ru, "f_2"'] \arrow[rd, "g_2"] &                        & C \\
                                                                                                        &                                          & B \arrow[ru, "\beta"'] &  
\end{tikzcd}


\vspace{5mm}
\underline{3.11.3.c) Identity}

Once again, it is clear that identity morphisms exist for all objects, simply by taking $Z = Z_1 = Z_2$, $f_1 = f_2$, $g_1 = g_2$ and $\sigma = id_Z$, in the diagram above.


\vspace{5mm}
\underline{3.11.3.d) Composition}

Let be 3 objects of $\mathcal{C}_{\alpha, \beta}$, which we will name $p_1$, $p_2$ and $p_3$ (and define with the respective $(Z_n, f_n, g_n)$ triplet for $p_n$).

Composition $\tau_{\alpha, \beta} \circ \sigma_{\alpha, \beta} = p_1 \mapsto p_3$ of two morphisms $\sigma_{\alpha, \beta} = p_1 \mapsto p_2$ and $\tau_{\alpha, \beta} = p_2 \mapsto p_3$ is defined so that the following diagram commutes.

% https://tikzcd.yichuanshen.de/#N4Igdg9gJgpgziAXAbVABwnAlgFyxMJZARgBpiBdUkANwEMAbAVxiRAC0B9AJhAF9S6TLnyEUAZlIAGKrUYs2AQX6CQGbHgJEALOVn1mrRCADCKoRtFFJ3ffKMgAQubXDNY5FL3UDC412IXdREtFG5vOUM2LnF+WRgoAHN4IlAAMwAnCABbJC8QHAgkMki-EDSeEGoGOgAjGAYABTcrYywwbFgXTJzi6kKkcNKHAB0RxjQACzpurNzESQKixCHfUZH6nBnquobmy1CQDKxEyZxZ3sR8gYWfezZEyp36ppbD9s7WAXS5pF0lvrDNgVQLUOCTLBpc6IAC03G+5V+iH+N0WawenFBIHBkOhQxqL32ITERxOZwu8xRy3y6OMY2wiWy2xABL2bxJHywXTBEKheQRPXmAFZ+ssSrTypxYs82QcScdTucBUiRQDbkDjI9YsrLtdlqrWa85WxOdyNSAxlsmHE+EA
\begin{tikzcd}
                                                                                                          &                                                                                                 &                                          & A \arrow[rd, "\alpha"] &   \\
Z_1 \arrow[rrru, "f_1", shift left=2] \arrow[rrrd, "g_1"', shift right=2] \arrow[r, "\sigma" description] & Z_2 \arrow[rru, "f_2" description] \arrow[rrd, "g_2" description] \arrow[r, "\tau" description] & Z_3 \arrow[ru, "f_3"'] \arrow[rd, "g_3"] &                        & C \\
                                                                                                          &                                                                                                 &                                          & B \arrow[ru, "\beta"'] &  
\end{tikzcd}


\vspace{5mm}
\underline{3.11.3.e) Associativity}

Associativity follows from associativity of morphisms in $\mathcal{C}$, similarly to what was done for slice categories in exercise 3.7 .



\subsubsection*{3.11.4) Fibered bi-coslice categories}

\vspace{5mm}
\underline{3.11.4.a) Objects}

To build a fibered bi-coslice category $\mathcal{C}^{\alpha, \beta}$, one takes a base category $\mathcal{C}$, as well as a fixed pair of morphisms $\alpha : C \to A$ and $\beta : C \to B$ in $\mathcal{C}$, that originate from a common object $C$ of $\mathcal{C}$. Our basic "fixed construct" from $\mathcal{C}$ looks like so: 

% https://tikzcd.yichuanshen.de/#N4Igdg9gJgpgziAXAbVABwnAlgFyxMJZARgBoAGAXVJADcBDAGwFcYkQBBEAX1PU1z5CKcqWLU6TVuwDCPPiAzY8BImQBMEhizaIQAIR4SYUAObwioAGYAnCAFskZEDghJRknewA63pmgALehAaRnoAIxhGAAUBFWEQGyxTAJx5aztHRGdXJHUabWk9X0icYNCIqNjlIXZGGCs07kpuIA
\begin{tikzcd}
                                            & A \\
C \arrow[ru, "\alpha"'] \arrow[rd, "\beta"] &   \\
                                            & B
\end{tikzcd}

The role of the category $\mathcal{C}^{\alpha, \beta}$ is now to study the morphisms from this construct. A generic object from this new category looks like so:

% https://tikzcd.yichuanshen.de/#N4Igdg9gJgpgziAXAbVABwnAlgFyxMJZARgBoAGAXVJADcBDAGwFcYkQBBEAX1PU1z5CKAEyli1Ok1bsAWjz4gM2PASLlxkhizaIQAYQX8VQomRFbpukACEekmFADm8IqABmAJwgBbJGJAcCCQNKR12AB0IpjQAC3oQGkZ6ACMYRgAFAVVhEE8sJ1icIxAvXxCaIKQyMJk9d0SQZLTM7NM9fMLi3g9vP0QAZkrgxBrtOpAnRub0rJM1PUYYd27FMv6AqsGacesotJwEpNTZtoWm5e7KbiA
\begin{tikzcd}
                                            & A \arrow[rd, "f"'] &   \\
C \arrow[ru, "\alpha"'] \arrow[rd, "\beta"] &                    & Z \\
                                            & B \arrow[ru, "g"]  &  
\end{tikzcd}

such that the diagram commutes. This means that valid object in $\mathcal{C}^{\alpha, \beta}$ are triplets $(Z, f, g)$, with $f : A \to Z$ and $g : B \to Z$, such that $f \circ \alpha = g \circ \beta$. In a caricatural way, this boils down to studying "the comparison of the different paths one can build by starting from $C$, knowing that the choice of first step is on one hand, $\alpha$, and on the other, $\beta$".


\vspace{5mm}
\underline{3.11.4.b) Morphisms}

Morphisms are defined between objects as:

% https://tikzcd.yichuanshen.de/#N4Igdg9gJgpgziAXAbVABwnAlgFyxMJZARgBoAGAXVJADcBDAGwFcYkQBBEAX1PU1z5CKAGwVqdJq3Zde-bHgJEATKWISGLNohAAtAPrEefEBgVCi5NRqnaQAYWPzBSlAGZrNTdJ1PTAxWFkABZPSS12PzMXIIB2MO87A2UogIsUAFYE23ZHOX9zVxJSZRsInQAhVMKgsVKvHMqeCRgoAHN4IlAAMwAnCABbJA8QHAgkK3CfEAAdGaY0AAt6EBpGegAjGEYABTTXEF6sNsWcPz7BpHjR8cQyKbs5heXVkHWt3f3hQ+PT8-6hohJmMkKoHuxuoZXu9tnsauwjiczvkLoCABw0EGIMGJdhtKFrTawr7sRgwbrIkyopChG5ILLgnRzbBtAb0AB6wCejCW9FIcy2OHo3GhRM+8J0ZIp-0ud0xtzEjJAkJShI+cJiCN+lJ6AKQAE55UhFbidPjVW8xRrAqTyTrlXrECMsRilQKYELReqSZK7TLAdcsYa3TNBSs1cSJW8-dxKNwgA
\begin{tikzcd}
                                            & A \arrow[rd, "f_1"'] &     &                                         &    &                                             & A \arrow[rd, "f_2"'] &     \\
C \arrow[ru, "\alpha"'] \arrow[rd, "\beta"] &                      & Z_1 & {} \arrow[r, "{\sigma^{\alpha,\beta}}"] & {} & C \arrow[ru, "\alpha"'] \arrow[rd, "\beta"] &                      & Z_2 \\
                                            & B \arrow[ru, "g_1"]  &     &                                         &    &                                             & B \arrow[ru, "g_2"]  &    
\end{tikzcd}

such that the following diagram commutes

% https://tikzcd.yichuanshen.de/#N4Igdg9gJgpgziAXAbVABwnAlgFyxMJZARgBoAGAXVJADcBDAGwFcYkQBBEAX1PU1z5CKAEyli1Ok1bsAWgH1iPPiAzY8BIgGZxkhizaIQCkcv7qhRcrpr6ZRgMJnVAjcJKkRe6YZAAhHkkYKABzeCJQADMAJwgAWyQdEBwIJGspA3YAHSymNAALehAaRnoAIxhGAAVXSyNorBD8nGcY+LSaFKQyDPsQSMVikFKK6trNesbmobh8rEiWxGJeKNiExAAWTtSl2x92EMGS8sqaiwnhmAWZuevEAFpllTb1nq7EMV7fHOwQuKLjqMzoILlgwNhYK01h1kjtPnZfANTIDTuNhJdrit+tDNtskPD9kZDsjhicxud0Q0mi0sS9EnjcV9slkKjgAaSgWj2IwrjTKNwgA
\begin{tikzcd}
                                            & A \arrow[rd, "f_1"', shift right] \arrow[rrd, "f_2"] &                                     &     \\
C \arrow[ru, "\alpha"'] \arrow[rd, "\beta"] &                                                      & Z_1 \arrow[r, "\sigma" description] & Z_2 \\
                                            & B \arrow[ru, "g_1", shift left] \arrow[rru, "g_2"']  &                                     &    
\end{tikzcd}


\vspace{5mm}
\underline{3.11.4.c) Identity}

Once again, it is clear that identity morphisms exist for all objects, simply by taking $Z = Z_1 = Z_2$, $f_1 = f_2$, $g_1 = g_2$ and $\sigma = id_Z$, in the diagram above.


\vspace{5mm}
\underline{3.11.4.d) Composition}


Let be 3 objects of $\mathcal{C}^{\alpha, \beta}$, which we will name $p_1$, $p_2$ and $p_3$ (and define with the respective $(Z_n, f_n, g_n)$ triplet for $p_n$).

Composition $\tau^{\alpha, \beta} \circ \sigma^{\alpha, \beta} = p_1 \mapsto p_3$ of two morphisms $\sigma^{\alpha, \beta} = p_1 \mapsto p_2$ and $\tau^{\alpha, \beta} = p_2 \mapsto p_3$ is defined so that the following diagram commutes.

% https://tikzcd.yichuanshen.de/#N4Igdg9gJgpgziAXAbVABwnAlgFyxMJZARgBoAGAXVJADcBDAGwFcYkQBBEAX1PU1z5CKAEyli1Ok1bsAWgH1iPPiAzY8BIgGZxkhizaIQCkcv7qhRACy6a+mUYVazqgRuHJytqQfYBhFzVBTRQyET1pQxAAIR5JGCgAc3giUAAzACcIAFskGxAcCCQvHwcQAB1ypjQAC3oQGkZ6ACMYRgAFN0sjDKxEmpwXTJzimkKkMlKotMUGkCbWjq6QkF7+wZo4Gqw0wcRyXnSs3MQAVjGixEn7KMTZxpa2zosVxhhdua2dvYOVYZPJuNEGIpuxKthEtl6g9Fs9gsIQFgwNhYENjqMCpcQTd2DNTDCnssEUiUWxDiB-khzpikNjIuw7vj5o8li9icisKjyZTgRckDpQUZKjh6Mw5gtCWz2CTOWS-uj9nzEAKcUYZs4Caz4ew3h9NtsPogALQibkK6lAlX0ox3DXM2FE9hrAafA17U3ykaIfJA6mqirlVoi8UsuHuHXvQbcSjcIA
\begin{tikzcd}
                                            & A \arrow[rd, "f_1"'] \arrow[rrd, "f_2" description] \arrow[rrrd, "f_3", shift left=2]  &                                     &                                   &     \\
C \arrow[ru, "\alpha"'] \arrow[rd, "\beta"] &                                                                                        & Z_1 \arrow[r, "\sigma" description] & Z_2 \arrow[r, "\tau" description] & Z_3 \\
                                            & B \arrow[ru, "g_1"] \arrow[rru, "g_2" description] \arrow[rrru, "g_3"', shift right=2] &                                     &                                   &    
\end{tikzcd}


\vspace{5mm}
\underline{3.11.4.e) Associativity}

Associativity follows from associativity of morphisms in $\mathcal{C}$, similarly to what was done for slice categories in exercise 3.7 .
