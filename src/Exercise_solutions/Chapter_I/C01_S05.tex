\section*{Section 5)}

\subsection*{5.1)}

Prove that a final object in a category $\mathcal{C}$ is initial in the opposite category $\mathcal{C}^{op}$

Let $\mathcal{C}$ be a category. Let $\mathcal{C}^{op}$ be the dual category on $\mathcal{C}$. Let $F$ be a final object in $\mathcal{C}$. This means that for every object $Z$ in $\mathcal{C}$, there is a single morphism from $Z$ to $F$. We will call this morphism $f_Z$ (respectively).

We remind how we defined composition in $\mathcal{C}^{op}$ as $\star$, respecting:
$$
\begin{aligned}
	& \forall f \in Hom_{\mathcal{C}^{op}} (B, A) = Hom_{\mathcal{C}} (A, B), \\
	& \forall g \in Hom_{\mathcal{C}^{op}} (C, B) = Hom_{\mathcal{C}} (B, C), \\
	& \exists h \in Hom_{\mathcal{C}^{op}} (C, A) = Hom_{\mathcal{C}} (A, C), \\
	& f \star g \coloneqq g \circ f = h
\end{aligned}
$$

In this case, we see that $\forall Z \in Obj(\mathcal{C}^{op}) = Obj(\mathcal{C}), f_Z \in Hom_{\mathcal{C}^{op}} (F, Z) = Hom_{\mathcal{C}} (Z, F)$. This implies that the homset $Hom_{\mathcal{C}^{op}} (F, Z)$ contains a single morphism, $f_Z$. This means that $F$ is initial in $mathcal{C}^{op}$.




\subsection*{5.2)}

Prove that $\emptyset$ is the \textit{unique} initial object in \textbf{Set}.

First we will prove that it is initial, then that it is unique.

Initiality: we take an arbitrary set $Z$ in \textbf{Set}. We wish to study $Hom_{\text{\textbf{Set}}}(\emptyset, Z) = Z^\emptyset$. We recall that functions (in category theory) are defined as "applications" / "mappings" are in traditional set theory (i.e., as a relation between sets where every antecedent in the domain has a singleton image in the codomain; the key point being that "no input has no result when passed through the function"). Let $I$ be an initial element in \textbf{Set}. We write $|I| = n$ and $|Z| = m$. We know that $|Z^I| = |Z|^{|I|} = m^n$. For $I$ to be initial, this is true if and only if $m^n = 1$ for all $m$, and so if and only if $n = 0$. We recall that the empty set is the only set with $|\emptyset| = 0$, therefore $I = \emptyset$.

Now this is already enough to prove unicity, but let us spell it out for pedagogy's sake.

Unicity: We recall that two objects of \textbf{Set} are isomorphic if, and only if, there exists a bijection between them. This is equivalent to saying that two sets have the same cardinal. We once again recall that the empty set is the only set with $|\emptyset| = 0$; there are no bijections relating to the empty set, other than its identity, the unique morphism in $Hom_{\text{\textbf{Set}}}(\emptyset, \emptyset)$ . Using proposition 5.4 (that terminal objects are unique up-to-isomorphism), we finally deduce that $\emptyset$ is the unique initial object in \textbf{Set}.

NB: the unique function in $Z^\emptyset$ is always the empty function.



\subsection*{5.3)}

Prove that final objects are unique up to isomorphism.

Let us suppose we have a category $\mathcal{C}$ with two final objects, $F_1$ and $F_2$.

For every object $A$ of $\mathcal{C}$ there is at least one element in $Hom_{\mathcal{C}} (A, A)$, namely the identity $1_A$. If $F$ is final, then there is a unique morphism $F \to F$, which therefore must be the identity $1_F$.

Now assume $F_1$ and $F_2$ are both final in $\mathcal{C}$. Since $F_1$ is final, there is a unique morphism $f : F_2 \to F_1$ in $\mathcal{C}$. Since $F_2$ is final, there is a unique morphism $g : F_1 \to F_2$ in $\mathcal{C}$. Consider $gf : F_1 \to F_1$ ; as observed, necessarily $gf = 1_{F_2}$
since $F_1$ is final. By the same token $fg = 1{F_1}$, proving $f$ is an isomorphism. Thus $F_1 \simeq F_2$.



\subsection*{5.4)}

What are initial and final objects in the category of "pointed sets" (Example 3.8)? Are they unique?

We recall that a pointed set is just a regular set with a special, identified point, and that the category of pointed sets \textbf{Set*} is built upon the same objects as \textbf{Set}, but where each object $A$ in \textbf{Set} is multiplied into $|A|$ copies of itself in \textbf{Set*} (one for each choice of special point; this implies that the empty set is not a part of \textbf{Set*}, since it has no point). Morphisms in \textbf{Set*} are set functions, but with the restriction of mapping the special point in the domain to the special point in the codomain.

Given this information, we will prove that the initial and final objects in \textbf{Set*} are the singleton sets.

Let $(O, o)$ be a singleton set in \textbf{Set*}. Let $o$ be the single element of $O$; it is necessarily also the special point, as there is no other choice. For any codomain $(Z, z_0)$ in \textbf{Set*}, the condition that "special points map to special points" restricts our choice of function to the unique function ${(o, z_0)}$, thus, $O$ is initial. If $O$ had more than one element, there would exist some $Z$ (non-singletons) for which the other element would allow another degree of freedom (and thus $O$ would not be initial).

Similarly, now studying $Z$ as a domain and $O$ as a codomain, we see that that only function from $Z$ to $O$ is (like in \textbf{Set}) the function which maps everything (including $Z$'s special point) to $o$. Thus, $O$ is final. If $O$ had more than one element, there would similarly be many choices for any $Z$ of cardinal $\geq 2$, so long as the special point maps to the special point.

Every singleton pointed set is both initial and final in \textbf{Set*} and is thus a zero object. These are also the only such pointed sets.



\subsection*{5.5)}

What are the final objects in the category considered in §5.3?

The category considered in paragraph 5.3 is the coslice category over some set $A$. However, what is presented in this paragraph is some extra structure that arises when considering the statement "The quotient $A/\sim$ is universal with respect to the property of mapping $A$ to a set in such a way that equivalent elements have the same image". We thus give some equivalence relation $\sim$ on $A$ and study the quotient set $A/\sim$ in the general coslice category; to do this, we consider the subcategory $\mathcal{Q}$ of $\mathcal{C}_A$ where only $\varphi$ such that "equivalence is preserved" (i.e., such that $\forall a', a'' \in A, a' \sim a'' \Rightarrow \varphi(a') = \varphi(a'')$).

With:
\begin{itemize}
	\item $s$ the canonical surjection of $A$ onto its quotient $A/\sim$,
	\item $\varphi_1$ (resp. $\varphi_2$) being some arbitrary function from $A$ to some arbitrary $Z_1$ (resp. $Z_2$),
	\item $\sigma$ any function (if it exists) such that $\sigma \varphi_1 = \varphi_2$
	\item $f_1$ (resp. $f_2$) is the (unique!) function such that $s f_1 = \varphi_1$ (resp. $s f_2 = \varphi_2$)
\end{itemize}

The following diagram commutes, and summarizes the situation.

% https://tikzcd.yichuanshen.de/#N4Igdg9gJgpgziAXAbVABwnAlgFyxMJZABgBpiBdUkANwEMAbAVxiRAEEQBfU9TXfIRRkAjFVqMWbdgHoAOnOwBbbrxAZseAkRGkx1es1aIQALQD6I1X02Cd5cYakmLAJm7iYUAObwioADMAJwgVRDIQHAgkXQkjNgRqBjoAIxgGAAV+LSEQIKxvAAscaxBg0JjqKKRXA0ljMssQJNT0rNttE3yikp5AkLCI6sRauOcQBXogtEKsJpa0zOy7EywwbFhmkDSwKCQAWgBmYj6ygaQh6MRDuviTSbpp2fN3U-Kw0eGbsYaFbG8lHQtslFu0BJ08gViqV3pVIldvjs9ogjhEQW1lhC1htWLdxgEXh4uEA
\begin{tikzcd}
A \arrow[d, "s"'] \arrow[rd, "\varphi_1" description, bend right] \arrow[r, "\varphi_2"] & Z_2                      \\
A/\sim \arrow[r, "f_1"'] \arrow[ru, "f_2" description, bend right]                       & Z_1 \arrow[u, "\sigma"']
\end{tikzcd}

Objects in this category are denoted as $(\varphi, Z)$ and are obtained from what used to be \textit{morphisms} (regular functions) in \textbf{Set}. Morphisms are mappings $\sigma_{\mathcal{Q}} : (\varphi_1, Z_1) \to (\varphi_2, Z_2)$ such that one exists if and only if $\exists \sigma \in (Z_1 \to Z_2), \sigma \varphi_1 = \varphi_2$, and $\forall a', a'' \in A, a' \sim a'' \Rightarrow \varphi(a') = \varphi(a'')$.

Since the textbook also asks whether such a category has initial objects, we will first also answer this and consider all terminal objects.

The initial object of a general coslice category is $id_A$. This is easily verified by doing $\varphi_1 = id_A$, necessarily $\sigma \varphi_1 = \sigma id_A = \sigma = \varphi_2$, and so $\sigma$ always exists and is unique. We also see that this object satisfies the "equivalence preservation" condition, hence it exists in $\mathcal{Q}$, and is also the initial object in $\mathcal{Q}$.

A general coslice category has a final object $(t, F)$ (or many final objects $(t_i, F_i)$) iff $\mathcal{C}$ has a final object $F$ (or many final objects $F_i$). In that case, any final object $(t, F)$ in $\mathcal{C}_A$ corresponds to the unique morphism from $A$ to $F$ (for any final $F$) in $\mathcal{C}$. Let us verify this.

Let $F$ be final in $\mathcal{C}$, and $t$ be the unique morphism $t \in Hom_{\mathcal{C}} (A, F)$. Let $(\varphi, Z)$ be an arbitrary object of $\mathcal{C}_A$. Let be $\sigma$ such that $\sigma \varphi = t$. We consider the following diagram:

% https://tikzcd.yichuanshen.de/#N4Igdg9gJgpgziAXAbVABwnAlgFyxMJZABgBpiBdUkANwEMAbAVxiRAEEQBfU9TXfIRQBGclVqMWbAGLdeIDNjwEiZYePrNWiEAC1u4mFADm8IqABmAJwgBbJGRA4ISURK1scIanAAWWCy9EYh5LG3tEACZqZ1dqTSkdAB0k7GNbOm8QBjoAIxgGAAV+ZSEQKyxjXy9QkGs7BxiXKPjJbRAU+is0fyyc-KKSwTYKqpqKLiA
\begin{tikzcd}
A \arrow[r, "t"] \arrow[d, "\varphi"'] & F \\
Z \arrow[ru, "\sigma"']                &  
\end{tikzcd}

Since $F$ is final in $\mathcal{C}$, $\sigma$ is unique and always exists. Also, since $\sigma$ is unique and always exist, the choice of $\varphi$ is irrelevant: this same $\sigma$ works for all choices of $\varphi$ for a given arbitrary $Z$. This proves that $\sigma_{\mathcal{C}_A}$ exists and is unique for all $(\varphi, Z)$. Finally, since $\sigma$ works for all choices of $\varphi$, it works for those that satisfy the "equivalence preservation" condition, and so does $t$: this means that $(t, F)$ is indeed a final object in $\mathcal{Q}$.


% TODO the below might useful for a discussion of how one could find the facts about terminal objects used above.
% For an object $(i, I)$ to be initial in $\mathcal{C}_A$, we need it to have a single possible $\sigma_\mathcal{C}_A$ mapping it to every $(\varphi, Z)$. Between any two objects in a coslice category, there is at most one morphism $\sigma_\mathcal{C}_A$ iff the $\varphi_1$ (for the corresponding $\sigma$) is an epimorphism (see C01 S03 notes, "On the morphisms of slice and coslice categories"). There is and at least one morphism iff

% source ncatlab
% In a slice category: 
% If C has an initial object ∅, then C / X has an initial object, given by ⟨∅→X⟩.
% The final object of C / X is id X.
% source ChatGPT based on the above
% In a coslice category:
% If C has a final object 1, then X/C has a final object, given by ⟨X→1⟩.
% The initial object of X/C is idX.



\subsection*{5.6)}

Consider the category corresponding to endowing (as in Example 3.3) the set $\mathbb{Z}^+$ of positive integers with the divisibility relation. Thus there is exactly one morphism $d \to m$ in this category if and only if $d$ divides $m$ without remainder; there is no morphism between $d$ and $m$ otherwise. Show that this category has products and coproducts. What are their 'conventional' names? [§VII.5.1]

Like example 3.3, this is a case of "category made from an order relation over a set", since divisibility is an order relation (reflexive, antisymmetric, transitive).

Let us remind the definition of categorical products and coproducts. We consider some general category $\mathcal{C}$.

An object $A \prod B$ is the product of two objects $A$ and $B$ iff there is a unique morphism $\pi_A$ (resp. $\pi_B$) from $A \prod B$ to $A$ (resp. $B$), and for every $Z$ in $\mathcal{C}$, and for every pair of morphisms $f_A : Z \to A$ and $f_B : Z \to B$, there exists a single morphism $\sigma = f_A \prod f_B$ such that $\pi_A \sigma = f_A$ and $\pi_B \sigma = f_B$. This is summarized in the following commutative diagram.

% https://tikzcd.yichuanshen.de/#N4Igdg9gJgpgziAXAbVABwnAlgFyxMJZARgBoAGAXVJADcBDAGwFcYkQAtEAX1PU1z5CKcqWLU6TVuwCCPPiAzY8BImXE0GLNohAyABAB1DeALbx9AIXn9lQogCYxErdN3XuEmFADm8IqAAZgBOEKZIoiA4EEhkktrsgQD6cjSM9ABGMIwACgIqwiDBWD4AFjg2ICFhETTRSE7xbiDG2D6m9PoAvPrJBsZmFsnWaZnZeXaqulhg2LCV1eGIkfWIAMyaUjpVSR4Kiw11MYhxrtvGaFgpC6FLjasbTeeGl7sgo1m5+fa6xWUVnm4QA
\begin{tikzcd}
  & Z \arrow[ld, "f_A"'] \arrow[d, "\sigma = f_A \prod f_B" description] \arrow[rd, "f_B"] &   \\
A & A \prod B \arrow[l, "\pi_A"] \arrow[r, "\pi_B"']                                       & B
\end{tikzcd}

An object $A \coprod B$ is the coproduct of two objects $A$ and $B$ iff there is a unique morphism $i_A$ (resp. $i_B$) from $A$ (resp. $B$) into $A \coprod B$, and for every $Z$ in $\mathcal{C}$, and for every pair of morphisms $f_A : A \to Z$ and $f_B : B \to Z$, there exists a single $\sigma = f_A \coprod f_B)$ such that $\sigma i_A = f_A$ and $\sigma i_B = f_B$. This is summarized in the following commutative diagram.

% https://tikzcd.yichuanshen.de/#N4Igdg9gJgpgziAXAbVABwnAlgFyxMJZARgBpiBdUkANwEMAbAVxiRAC0QBfU9TXfIRQAGUsKq1GLNgEFuvEBmx4CRMuOr1mrRCBkACADqGAxhDQAnaPoBC8vssFEATGIlbpuu1wkwoAc3giUAAzKwBbJDIQHAgkUUltNhCAfTlqBjoAIxgGAAV+FSEQCyx-AAscexAwiEjEVxi4xASPHRBjbH9wun0AXn1Ug2MzS2tUuwzs3ILHVV0sMGxYatr6gGZqWPjNKXaJ1YiorebGtrYsNMO6pE2mpDO9i5TJkEyc-MKnXVKKqp8uEA
\begin{tikzcd}
A \arrow[rd, "f_A"'] \arrow[r, "i_A"] & A \coprod B \arrow[d, "\sigma = f_A \coprod f_B" description] & B \arrow[ld, "f_B"] \arrow[l, "i_B"'] \\
                                      & Z                                                             &                                      
\end{tikzcd}

We now return to our "divisibility order category". We write its objects as simple integers, and the (if it exists, unique) morphism representing "divisibility of $m$ by $n$" as $(n | m)$. The conventional name of the product for this category is "greatest common divisor" (or "meet"), and of the coproduct, "least common multiple" (or "join").

The following commutative diagrams represent this fact. Take two arbitrary naturals $m$ and $n$. Any number $k$ which divides both $m$ and $n$ also divides their GCD. Likewise, if $k$ is a multiple of both $n$ and $m$, then it is a multiple of their LCM. 

% https://tikzcd.yichuanshen.de/#N4Igdg9gJgpgziAXAbVABwnAlgFyxMJZARgBoAGAXVJADcBDAGwFcYkQBrEAX1PU1z5CKcqWLU6TVu0K9+2PASIAmMRIYs2iEAFsefEBgVCiZcTQ3TtAcQDCAEQAUYUgAIdASh4SYUAObwRKAAZgBOEHqIoiA4EEjEciBhEUjRsUjKicmRZDFxiADMWeGRqnlIRQbZqTTphdyU3EA
\begin{tikzcd}
            & k \arrow[ld] \arrow[rd] \arrow[d] &             \\
n \arrow[r] & {GCD(n, m)}                       & m \arrow[l]
\end{tikzcd}

% https://tikzcd.yichuanshen.de/#N4Igdg9gJgpgziAXAbVABwnAlgFyxMJZARgBpiBdUkANwEMAbAVxiRAGsQBfU9TXfIRQAGUsKq1GLNoR59seAkTLjq9Zq0QgAMgGEAsgAowpAAQBbAJTdeIDAsFEATGInrpW89wkwoAc3giUAAzACcIL0QyEBwIJFEQBjoAIxgGAAV+RSEQUKw-AAscGxDwyJcYuMQEpNSMrMctPMLiuRAwiKQAZmpY+LaOyOi+xCcBsu7eqoratMyHJSb8ou8uIA
\begin{tikzcd}
n \arrow[rd] \arrow[r] & {LCM(n, m)} \arrow[d] & m \arrow[ld] \arrow[l] \\
                       & k                     &                       
\end{tikzcd}



\subsection*{5.7)}

Redo Exercise 2.9 ("Show that if $A \simeq A'$ and $B \simeq B'$ , and further $A \cap B = \emptyset$ and $A' \cap B' = \emptyset$, then $A \cup B \simeq A' \cup B'$. Conclude that the operation $A \coprod B$ (as described in §1.4) is well-defined up to isomorphism") this time using Proposition 5.4. (the unicity up-to-isomorphism of terminal objects).

We define $\text{\textbf{Set}}^{A,B}$ as the "bicoslice category of $A$ and $B$ over \textbf{Set}". Objects in this category are pairs of morphisms $(f,g)$ from $A$ and $B$, respectively, into sets $Z$. They can be diagrammed as follows.

% https://tikzcd.yichuanshen.de/#N4Igdg9gJgpgziAXAbVABwnAlgFyxMJZABgBpiBdUkANwEMAbAVxiRAEEQBfU9TXfIRQBGUsKq1GLNgC1uvEBmx4CRMgCYJ9Zq0QgAQtwkwoAc3hFQAMwBOEALZIyIHBCSjJOtlfnW7jxHVqV3dqbWk9UyMuIA
\begin{tikzcd}
A \arrow[rd, "f"] &   \\
                  & Z \\
B \arrow[ru, "g"] &  
\end{tikzcd}

Morphisms are defined between objects as

% https://tikzcd.yichuanshen.de/#N4Igdg9gJgpgziAXAbVABwnAlgFyxMJZAVgBoBGAXVJADcBDAGwFcYkQAtAfQCYQBfUuky58hFAAZSE6nSat2AQQFCQGbHgJEpPWQxZtEIAEIrhGsUXIU98w5y7kzakZvHIALNNsGlz9aJaKF66NPoKRqaC5oHuPDZhduz+rpYoAMwJcr5GArIwUADm8ESgAGYAThAAtkjWIDgQSJnZESBljiA0jPQARjCMAAqpQSCMMGU4zpU1SPENTYgt4faFnd19A8MWoxVYhQAWU9HtVbWIZAtIUq2rvF1jm0Mj4iB7h8eqM+deV4g3K3YHT4J2+SAAbDRGkgAOyJHIgAA6iOwhWq9AAesBFKRjPw8vwgA
\begin{tikzcd}
A \arrow[rd, "f_1"]  &     &                                &    & A \arrow[rd, "f_2"]  &     \\
                     & Z_1 & {} \arrow[r, "{\sigma^{A,B}}"] & {} &                      & Z_2 \\
B \arrow[ru, "g_1"'] &     &                                &    & B \arrow[ru, "g_2"'] &    
\end{tikzcd}

such that the following diagram commutes

% https://tikzcd.yichuanshen.de/#N4Igdg9gJgpgziAXAbVABwnAlgFyxMJZABgBpiBdUkANwEMAbAVxiRAEEQBfU9TXfIRRkAjFVqMWbAFoB9Ed14gM2PASIjSY6vWatEIOQCZFfVYKJkj43VIMAhbuJhQA5vCKgAZgCcIAWyQyEBwIJCMdSX0QL1kTagY6ACMYBgAFfjUhEAYYLxxTGL9AxGDQpE0JPTZYhQTk1IzzdQMfLFcACwKeb2KK6nLECKq7EAAdMexXfzoQepT0zIsDXPzC3wCkAGYBsKHI6oNXOLmchsXm7LbO7qUNkp2QvcrEhaaBFpy8goPR44UuBQuEA
\begin{tikzcd}
A \arrow[rd, "f_2"] \arrow[d, "f_1"'] &     \\
Z_1 \arrow[r, "\sigma"]               & Z_2 \\
B \arrow[ru, "g_2"'] \arrow[u, "g_1"] &    
\end{tikzcd}

Let us call $I$ the following object of $\text{\textbf{Set}}^{A,B}$, where $A \coprod B$ is any choice of valid disjoint union of $A$ and $B$:

% https://tikzcd.yichuanshen.de/#N4Igdg9gJgpgziAXAbVABwnAlgFyxMJZABgBpiBdUkANwEMAbAVxiRAEEQBfU9TXfIRQBGUsKq1GLNuwAEAHXkBjCGgBO0WQCFuvEBmx4CRMgCYJ9Zq0QgdXCTCgBzeEVAAzDQFskZEDggkUUkrNiwAfU4eD28kU2oAoOpLaRsInWoGOgAjGAYABX4jIRA1LCcACxxuCi4gA
\begin{tikzcd}
A \arrow[rd, "i_A"]  &             \\
                     & A \coprod B \\
B \arrow[ru, "i_B"'] &            
\end{tikzcd}

By definition of a coproduct, we know that in such a configuration, a morphism $\sigma^{A,B}$ from this object into any other object of $\text{\textbf{Set}}^{A,B}$ exists and is unique, and so is the $\sigma$ on which it is based. This means that $I$ is initial in $\text{\textbf{Set}}^{A,B}$. Consequently, using prop 5.4., the fact that if an initial object exists, it is unique up-to-isomorphism, we conclude that $A \coprod B$ is unique up-to-isomorphism.



\subsection*{5.8)}

