\chapter*{Chapter II)}

\section*{Section 1)}

\subsection*{1.1)}

Write a careful proof that every group is the group of isomorphisms of a groupoid. In particular, every group is the group of automorphisms of some object in some category.

Let us first remind the definition of a groupoid: a groupoid is a category in which every morphism is an isomorphism.

(Side-note: by "group of isomorphisms", what Aluffi rather meant is "group of isomorphisms (of the single object) in a groupoid (that has a single object)". Otherwise, with multiple objects, things fail: for example, there are multiple identities, one per object, not universally applicable to each object. This is precisely why the notion of "groupoid" was invented, to extend the notion of group in such a way.)

Let $(G, \cdot)$ be a group, i.e., some form of algebraic structure with a set of elements $G$ and a binary operation $\cdot$ which is associative, unitary, and invertible. We want to show that there exists a groupoid $\mathcal{C}$ such that $(G, \cdot)$ is the group of isomorphisms of $\mathcal{C}$.

Let us define $\mathcal{C}$ as follows:
\begin{itemize}
	\item There is a single object $X$ in $\mathcal{C}$, and its elements are the elements of $G$ (we could call it $G$, of course, but we'll be distinct for pedagogy's sake).
	\item For any element $g$ of $X$, there is a morphism $f_g \in Hom(X, X)$ such that $\forall x \in X, f_g(x) = (x \mapsto g \cdot x)$.
	\item Composition of morphisms is defined as follows: $f_a \circ f_b = (x \mapsto a \cdot (b \cdot x)) = (x \mapsto (a \cdot b) \cdot x)$.
	\item There is an identity morphism $id_X = (x \to e \cdot x)$.
	\item It is immediate to see that these morphisms are associative since $(G, \cdot)$ is associative. Take $f_a$, $f_b$, and $f_c$, for $a,b,c \in X$: $(f_a f_b) f_c = (x \to ((a \cdot b) \cdot c) x) = (x \to (a \cdot (b \cdot c))x = f_a (f_b f_c)$.
	\item Every such morphism has an inverse, namely, $f_{g^{-1}} = (x \mapsto g^{-1} \cdot x)$ which by definition of a group necessarily exists. It is easy to verify that $f_g \cdot f_{g^{-1}} = f_{g^{-1}} = f_g = (x \mapsto x = id_X)$.
\end{itemize}

This is a groupoid, because, it is a category (composition, associativity, identity) where every morphism is an isomorphism (every morphism has an inverse), and the group of isomorphisms of $\mathcal{C}$ is precisely isomorphic to $(G, \cdot)$.


\subsection*{1.2)}

Consider the 'sets of numbers' listed in §1.1, and decide which are made into groups by conventional operations such as $+$ and $\cdot$. Even if the answer is negative (for example: $(\mathbb{R}, \cdot)$ is not a group), see if variations on the definition of these sets lead to groups (for example, $(\mathbb{R}^*, \cdot)$ is a group, cf. §1.4). [§1.2]

I suppose Aluffi is referring to §I.1.1, and not §(II.)1.1. In there he mentions:

\begin{itemize}
	\item $\mathbb{N}$: the set of natural numbers (that is, nonnegative integers);
	\item $\mathbb{Z}$: the set of integers;
	\item $\mathbb{Q}$: the set of rational numbers;
	\item $\mathbb{R}$: the set of real numbers;
	\item $\mathbb{C}$: the set of complex numbers.
\end{itemize}

Let us go through these sets one by one:

\begin{itemize}
	\item $\mathbb{N}$: $(\mathbb{N}, +)$ is not a group (needs negative numbers, $x + 5 = 0$ has no solution), and neither is $(\mathbb{N}, \cdot)$ (e.g., $5 \cdot x = 1$ has no solution in $\mathbb{N}$).
	\item $\mathbb{Z}$: $(\mathbb{Z}, +)$ is a group, but $(\mathbb{Z}, \cdot)$ is not because it is not invertible (e.g., $2 \cdot x = 1$ has no solution in $\mathbb{Z}$).
	\item $\mathbb{Q}$: $(\mathbb{Q}, +)$ is a group, but $(\mathbb{Q}, \cdot)$ is not because it is not fully invertible (e.g., $0 \cdot x = 1$ has no solution in $\mathbb{Q}$). However, remove $0$ from $\mathbb{Q}$ and it becomes a group.
	\item $\mathbb{R}$: $(\mathbb{R}, +)$ is a group, but $(\mathbb{R}, \cdot)$ is not because it is not fully invertible (e.g., $0 \cdot x = 1$ has no solution in $\mathbb{R}$). However, remove $0$ from $\mathbb{R}$ and it becomes a group.
	\item $\mathbb{C}$: $(\mathbb{C}, +)$ is a group, but $(\mathbb{C}, \cdot)$ is not because it is not fully invertible (e.g., $0 \cdot x = 1$ has no solution in $\mathbb{C}$). However, remove $0$ from $\mathbb{C}$ and it becomes a group.
\end{itemize}

We can see that $(\mathbb{N}, +)$ is not a group, but $(\mathbb{Z}, +)$, $(\mathbb{Q}, +)$, $(\mathbb{R}, +)$, and $(\mathbb{C}, +)$ are all groups. Also, $(\mathbb{N}, \cdot)$, $(\mathbb{Z}, \cdot)$, $(\mathbb{Q}, \cdot)$, $(\mathbb{R}, \cdot)$, and $(\mathbb{C}, \cdot)$ are not groups. However, $(\mathbb{Q}^*, \cdot)$, $(\mathbb{R}^*, \cdot)$, and $(\mathbb{C}^*, \cdot)$ are groups.



\section*{1.3)}

Prove that $(gh)^{-1} = h^{-1} g^{-1}$ for all elements $g, h$ of a group $G$.

$$(gh)(h^{-1}g^{-1}) = g(hh^{-1})g^{-1} = geg^{-1} = gg^{-1} = e$$
$$(h^{-1}g^{-1})(gh) = h^{-1}(g^{-1}g)h = h^{-1}eh = h^{-1}h = e$$

Therefore, $h^{-1} g^{-1}$ is a 2-sided inverse for $gh$, and since the inverse of an element in a group is unique, $(gh)^{-1} = (h^{-1}g^{-1})$



\section*{1.4)}

Suppose that $g^2 = e$ for all elements $g$ of a group $G$; prove that $G$ is commutative.

If $g^2 = e$, then every element is its own inverse, i.e., multiply by $g^{-1}$ on the left, and we have $\forall g \in G, g = g^{-1}$. Let be $a, b \in G$. Then, we have $ba = (ba)^{-1} = a^{-1}b^{-1} = ab$, which is the definition of commutativity. Therefore, $G$ is commutative.



\section*{1.5)}

The 'multiplication table' of a group is an array compiling the results of all multiplications $g \cdot h$ (with the value on each row being the left operand, and the value on each column being the right operand; of course the table depends on the order in which the elements are listed in the top row and leftmost column). Prove that every row and every column of the multiplication table of a group contains all elements of the group exactly once (like Sudoku diagrams!).

Another way to phrase this question is "prove that every element of the group can be reached in a single operation from any other". Written in function notation, this means:

$$\forall a, b \in G, \exists g_r, g_c \in G, b = g_r \cdot a \text { and } b = a \cdot g_c$$

For every element $a$, we can reach the identity with $a^{-1}$ (on either side). Every element $b$ can then be reached from the identity by multiplying it (on either side). We just have to pick the same side both times to find either $g_r = b \cdot a^{-1}$ and $g_c = a^{-1} \cdot b$. Since the group is closed under "multiplication", both these elements are guaranteed to exist.

This implies that every row and every column of the multiplication table of a group contains all elements of the group exactly once, since any element can be reached in a single multiplication. This also implies that the multiplication table is a "Latin square", which is a square array of $|G|$ symbols, each occurring exactly once in each row and exactly once in each column. 



\section*{1.6)}

Prove that there is only one possible multiplication table for $G$ if $G$ has exactly 1, 2, or 3 elements. Analyze the possible multiplication tables for groups with exactly 4 elements, and show that there are two distinct tables, up to reordering the elements of $G$. Use these tables to prove that all groups with $\leq 4$ elements are commutative.

These multiplication tables are usually called "Cayley tables".

For the unique table of the trivial group (only 1 element); $e$ is necessarily its own inverse. Some examples of this group are $(\{0\}, +)$ or $(\{1\}, \times)$. This group is trivially commutative.

$$
\begin{tabular}{|c||c|}
\hline
\cdot & e \\
\hline \hline
e     & e \\
\hline
\end{tabular}
$$

For the unique table of the group with 2 elements; both elements are necessarily their own inverse (or else they would be equal, and we would have only $1$ element in the group). An example of this group is $(\{1, -1 \}, \times)$, or $(\mathbb{Z}_2, +)$ (two-hour clock with addition). This group is commutative, since, like in the exercise 1.4, every element is its own inverse.

$$
\begin{tabular}{|c||c|c|}
\hline
\cdot & e & a \\ \hline \hline
e     & e & a \\ \hline
a     & a & e \\ \hline
\end{tabular}
$$

For the unique table of the group with 3 elements; the 3 elements cannot all be their own inverses, because if 2 elements are their own inverse, the third one cannot have an inverse. Another way of seeing this is the below: no matter the value given for each $?$, this cannot be a Latin square (if we put $e$ then we have multiple inverses, so we can deduce that $a$ and $b$ are inverses, and then we have multiple $a$'s, etc).

$$
\begin{tabular}{|c||c|c|c|}
\hline
\cdot & e & a & b \\ \hline \hline
e     & e & a & b \\ \hline
a     & a & e & ? \\ \hline
b     & b & ? & e \\ \hline
\end{tabular}
$$

Instead, the 2 non-identity elements are necessarily inverses of each other. Some examples of this group are $(\{1, j, \overline{j} \}, \times)$ (where $j = e^{\tau / 3}$ is the third root of unity), or $(\mathbb{Z}_3, +)$ (three-hour clock with addition). This group is commutative because the identity necessarily commutes with everything, and inverses necessarily commute together, so here, all elements commute.

$$
\begin{tabular}{|c||c|c|c|}
\hline
\cdot  & e      & a      & a^{-1} \\ \hline \hline
e      & e      & a      & a^{-1} \\ \hline
a      & a      & a^{-1} & e      \\ \hline
a^{-1} & a^{-1} & e      & a      \\ \hline
\end{tabular}
$$

For the case where $|G| = 4$, we have two possibilities.

The first one is a table where every element is a self-inverse. This gives you a group that is isomorphic to $(\mathbb{Z}_2 \times \mathbb{Z}_2, +)$ (the Klein four-group, a torus made up of 2 two-hour clocks). Note that the permutations of the elements of the group give the same table, so there is only one table for this group up to reordering. It is commutative for the same reason as exercise 1.4.

$$
\begin{tabular}{|c||c|c|c|c|}
\hline
\cdot & e & a & b & c \\ \hline \hline
e     & e & a & b & c \\ \hline
a     & a & e & c & b \\ \hline
b     & b & c & e & a \\ \hline
c     & c & b & a & e \\ \hline
\end{tabular}
$$

The second possibility where one element other than the identity is its own inverse, and the other two are mutual inverses. This gives you a group that is isomorphic to $(\mathbb{Z}_4, +)$ (a four-hour clock with addition, where $1$ and $-1$ are mutual inverses, and $0$ and $2$ are their own inverse). Note that the permutations of the elements of the group give the same table, so there is only one table for this group up to reordering.

(I can't find a neat reason why these are commutative other than relying on the specific example of $(\mathbb{Z}_4, +)$ and saying "addition is commutative lol".)

$$
\begin{tabular}{|c||c|c|c|c|}
\hline
\cdot & e & a & b & c \\ \hline \hline
e     & e & a & b & c \\ \hline
a     & a & e & c & b \\ \hline
b     & b & c & a & e \\ \hline
c     & c & b & e & a \\ \hline
\end{tabular}
$$

Note that the commutativity can also be seen in all these tables by the fact that they are their own transpose.



\subsection*{1.7)}
