\chapter*{Chapter 1}

\section*{Section 1}

\begin{itemize}
	\item Set (not a multiset): the set of all sets
	\item $\emptyset$: the empty set, containing no elements; % ∅
	\item $\mathbb{N}$: the set of natural numbers (that is, nonnegative integers);
	\item $\mathbb{Z}$: the set of integers;
	\item $\mathbb{Q}$: the set of rational numbers;
	\item $\mathbb{R}$: the set of real numbers;
	\item $\mathbb{C}$: the set of complex numbers.
	\item singleton:
	\item $\exists$: existential quantifier, "there exists" % ∃
	\item $\forall$: universal quantifier, "for all" % ∀
	\item inclusion:
	\item subset:
	\item cardinal:
	\item powerset:
	\item $\cup$: the union: % ∪
	\item $\cap$: the intersection: % ∩
	\item $\\$: the difference:
	\item $\coprod$: the disjoint union:
	\item $\times$: the (Cartesian) product: % ×
	\item complement of a subset
	\item relation
	\item order relation
	\item equivalence relation
	\item reflexivity
	\item symmetry
	\item antisymmetry
	\item transitivity
	\item partition
	\item $/\sim$: quotient by an equivalence relation
\end{itemize}


\section*{Section 2}

\begin{itemize}
	\item function
	\item graph
	\item (categorical, function) diagram
	\item identity function
	\item kernel (of a function)
	\item image (of a function)
	\item restriction (of a function to a subset)
	\item multiset
	\item composition
	\item commutative (diagram)
	\item injection
	\item surjection
	\item bijection
	\item isomorphism
	\item inverse
	\item pre-inverse, right-inverse
	\item post-inverse, left-inverse
	\item monomorphism
	\item epimorphism
	\item natural projection
	\item natural injection
	\item canonical decomposition (of a function)
\end{itemize}


\section*{Section 3}

\begin{itemize}
	\item category
	\item object
	\item morphism
	\item endomorphism
	\item operation
	\item discrete category
	\item small category
	\item locally small category
	\item slice category
	\item coslice category
	\item comma category (mentioned, undefined)
	\item pointed set
	\item $C^{A, B}$ category ?? (bislice, bicoslice, fibered bislice, fibered bicoslice)
	\item dual category
\end{itemize}


\section*{Section 4}

\begin{itemize}
	\item groupoid: category in which every morphism is invertible. A category of this sort can be viewed as augmented with a unary operation on the morphisms, called inverse by analogy with group theory.
	\item automorphism
\end{itemize}

\section*{Section 5}

\begin{itemize}
	\item universal property
	\item initial object
	\item final object
	\item terminal object
	\item (categorical) product
	\item (categorical) coproduct
	\item (categorical) pullback / fibered product
	\item (categorical) pushout / fibered coproduct
	\item (set) pullback / fibered product
	\item (set) pushout / fibered coproduct
\end{itemize}
