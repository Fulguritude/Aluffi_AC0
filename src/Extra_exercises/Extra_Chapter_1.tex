
Chapter I) 1) Set notation)

Write the following in set notation (as a list of numbers, and algebraically):
- the set of all odd integers
- the set of all integers that are not multiples of 3
- the set of integers from 10 (included) to 20 (included)
- the set of integers from 10 (included) to 20 (excluded)
- the set of pairs of integers with both elements of the same value
- the set of triplets of real numbers that together sum to 1
- the set of pairs of positive real numbers that together sum to 1
- the set of $n$-tuplets (for any $n$) of real number that together sum to 1
- the set of all natural numbers such that there exists at least one triplet of positive even numbers which are all different and which sum to that number.

Now take the sets in their algebraic notation, and represent them both as a list of numbers (as a logical sequence or just a couple of examples), and as a "description" of what they are:

 - $\{3n + 2 \; | \; n \in \mathbb{N} \}$
 - $\{3k + 2 \; | \; k \in \mathbb{Z} \}$
 - $\{ 2^i \; | \; i \in [[0, 10]] \}$
 - $\{ (x, y) \in \mathbb{R}^2 \; | \; x^2 + y^2 = 1 \}$
 - $\{ x \in \mathbb{R} \; | \; -2 \leq x \leq 2 \}$
 - $\{ (m, n, p) \in \mathbb{N}^3 \; | \; m + n + p = 10 \}$
