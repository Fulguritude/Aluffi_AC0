\part{Extra exercises by/for the group}

\section*{Chapter I) 1) Set notation)}

Write the following in set notation (as a list of numbers, and algebraically):
\begin{itemize}
	\item the set of all odd integers
	\item the set of all integers that are not multiples of 3
	\item the set of integers from 10 (included) to 20 (included)
	\item the set of integers from 10 (included) to 20 (excluded)
	\item the set of pairs of integers with both elements of the same value
	\item the set of triplets of real numbers that together sum to 1
	\item the set of pairs of positive real numbers that together sum to 1
	\item the set of $n$-tuplets (for any $n$) of real number that together sum to 1
	\item the set of all natural numbers such that there exists at least one triplet of positive even numbers which are all different and which sum to that number.
\end{itemize}

Now take the sets in their algebraic notation, and represent them both as a list of numbers (as a logical sequence or just a couple of examples), and as a "description" of what they are:

\begin{itemize}
	\item $\{3n + 2 \; | \; n \in \mathbb{N} \}$
	\item $\{3k + 2 \; | \; k \in \mathbb{Z} \}$
	\item $\{ 2^i \; | \; i \in [[0, 10]] \}$
	\item $\{ (x, y) \in \mathbb{R}^2 \; | \; x^2 + y^2 = 1 \}$
	\item $\{ x \in \mathbb{R} \; | \; -2 \leq x \leq 2 \}$
	\item $\{ (m, n, p) \in \mathbb{N}^3 \; | \; m + n + p = 10 \}$
\end{itemize}

\section

\section*{Chapter I) 3) Slices and coslices)}

Provide a concrete example of a slice category and of a coslice category based on the category of real vector spaces $Vect_{\mathbb{R}}$, or its subcategory of finite real vector spaces.
How does this relate to the transpose of a matrix, or of a product of matrices ?

\newpage
