
\documentclass[12pt, letterpaper, twoside]{report}

%\usepackage[utf8]{inputenc}
\usepackage[utf8x]{inputenc}
\usepackage{array}
\usepackage{amsmath}
\usepackage{amsfonts}
\usepackage{amssymb}
\usepackage{tikz-cd}
\usepackage{mathtools}

\begin{document}

\author{Multiversity \textit{Algebra Chapter 0} Reading Group}
\part{Summaries}

Chapter I)

Section 1) Explains fundamentals of set theory and basic set operations

Section 2) Explains set relations, set functions and some more advanced set operations

Section 3) Presents categories, and multiple examples of categories. Some are simple, some are advanced.

\newpage
\part{Lexicon}
\chapter*{Chapter 1}

\section*{Section 1}

\begin{itemize}
	\item Set (not a multiset)
	\item ∅: the empty set, containing no elements;
	\item N: the set of natural numbers (that is, nonnegative integers);
	\item Z: the set of integers;
	\item Q: the set of rational numbers;
	\item R: the set of real numbers;
	\item C: the set of complex numbers.
	\item Singleton:
	\item ∃: existential quantifier, "there exists"
	\item ∀: universal quantifier, "for all"
	\item inclusion:
	\item subset:
	\item cardinal:
	\item powerset:
	\item ∪: the union:
	\item ∩: the intersection:
	\item $\\$: the difference:
	\item $\coprod$: the disjoint union:
	\item ×: the (Cartesian) product:
	\item complement of a subset
	\item relation
	\item order relation
	\item equivalence relation
	\item reflexivity
	\item symmetry
	\item antisymmetry
	\item transitivity
	\item partition
	\item quotient by an equivalence relation
\end{itemize}


\section*{Section 2}

\begin{itemize}
	\item function
	\item graph
	\item (categorical, function) diagram
	\item identity function
	\item kernel (of a function)
	\item image (of a function)
	\item restriction (of a function to a subset)
	\item multiset
	\item composition
	\item commutative (diagram)
	\item injection
	\item surjection
	\item bijection
	\item isomorphism
	\item inverse
	\item pre-inverse, right-inverse
	\item post-inverse, left-inverse
	\item monomorphism
	\item epimorphism
	\item natural projection
	\item natural injection
	\item canonical decomposition (of a function)
\end{itemize}


\section*{Section 3}

\begin{itemize}
	\item category
	\item object
	\item morphism
	\item endomorphism
	\item operation
	\item discrete category
	\item small category
	\item locally small category
	\item slice category
	\item coslice category
	\item comma category (mentioned, undefined)
	\item pointed set
	\item $C^{A, B}$ category ??
\end{itemize}
\part{Exercise solutions}
\chapter*{Chapter I)}

\section*{Section 1)}

\subsection*{1.1)}

In a nutshell, Russell's paradox proves, by contradiction, that certain mathematical collections cannot be sets. It posits the existence of a "set of all sets that don't contain themselves". Such a set can neither contain itself (since in that case, it would be a "set that does contain itself", and should be excluded); nor can it exclude it itself (since in that case, it would be a "set that doesn't contain itself", and should be included).



\subsection*{1.2)}

Prove that any equivalence relation over a set $S$ defines a partition of $\mathcal{P}_S$.

a) $\mathcal{P}_S$ has no empty elements: any element in $S$ is part of at least one equivalence class, the class containing at least that element itself. Since there is no equivalence class constructed independently from elements, there are no empty equivalence classes.

b) Elements of $\mathcal{P}_S$ are disjoint: suppose there is an element $x$ that is part of $A$ and $B$, two distinct equivalence classes. $\forall a \in A, x \sim a$ and $\forall b \in B, x \sim b$. By transivity through $x$, $\forall a \in A, \forall b \in B, a \sim b$. Therefore, $A$ and $B$ are the same equivalence class: $A = B$. Contradiction. Therefore all elements of $\mathcal{P}_S$ are disjoint subsets of $S$.

c) The union of all elements of $\mathcal{P}_S$ makes up $S$: suppose $\exists x \in S$ such that $x \notin \bigcup_{S_i \in \mathcal{P}_S} S_i$. From the argument made in (a), $x$ exists in at least one equivalence class, the class which contains only $x$ itself. This is one of ou $S_i$ sets. Contradiction. Therefore, $\bigcup_{S_i \in \mathcal{P}_S} S_i = S$



\subsection*{1.3)}

Given a partition $\mathcal{P}$ on a set $S$, show how to define a relation $\sim$ on $S$ such that $P$ is the corresponding partition.

The insight here is to build an equivalence relation such that two elements are equivalent if and only if they are part of the same subset of $S$, which is understood as their common equivalence class.

We define $\sim$ such that $\forall S_i, S_j \in \mathcal{P}, \forall x \in S_i, \forall y \in S_j, x \sim y \Leftrightarrow S_i = S_j$.

Let us prove that $\sim$ is an equivalence relation.

a) Reflexivity:
$$\forall A \in \mathcal{P}, \forall x \in A, A = A \Rightarrow x \sim x$$

b) Symmetry:
$$\forall S_i, S_j \in \mathcal{P}, \forall x \in S_i, \forall y \in S_j, x \sim y \Leftrightarrow S_i = S_j \Leftrightarrow S_j = S_i \Leftrightarrow y \sim x$$

c) Transitivity:
$$
\forall S_i, S_j, S_k \in \mathcal{P}, \forall x \in S_i, \forall y \in S_j, \forall z \in S_k, x \sim y \cap y \sim z \\
	\Leftrightarrow \\
S_i = S_j \cap S_j = S_k \Rightarrow S_i = S_k \\
	\Leftrightarrow \\
x \sim z
$$

Therefore, $\sim$ is indeed an equivalence relation, and is generated uniquely by the partition.



\subsection*{1.4)}

How many different equivalence relations may be defined on the set $\{1, 2, 3\}$?

If we start with the 1 element set, we have only one possible partition, one possible equivalence class.

With the 2 element set, there are 2 partitions, $\{\{1, 2\}\}$ and $\{\{1\}, \{2\}\}$.

With the 3 element set, there is:
\begin{itemize}
	\item 1 partition of type 1-1-1: $\{\{1\}, \{2\}, \{3\}\}$.
	\item 3 partitions of type 2-1: $\{\{1\}, \{2, 3\}\}$, $\{\{2\}, \{1, 3\}\}$, and $\{\{3\}, \{1, 2\}\}$.
	\item 1 partition of type 3: $\{\{1, 2, 3\}\}$.
\end{itemize}

Hence, there are five equivalence classes on the 3 element set.

See the Bell numbers: https://oeis.org/A000110



\subsection*{1.5)}

Give an example of a relation that is reflexive and symmetric, but not transitive. What happens if you attempt to use this relation to define a partition on the set?

In terms of graph theory, if we express a set with an internal relation as a graph, we can represent elements are nodes and relationships are edges. Reflexivity means that every node has a loop (unary, self-edge). Symmetry means that the graph is not directed (since every relationship goes both ways). Transitivity means that every connected subset of nodes is a maximal clique (synonymously, every connected component is a complete subgraph).

In a relation which is reflexive and symmetric, but not transitive, you would have connected components of this graph which are not cliques. For these, there is ambiguity as to how you would group their nodes. Two obvious choices would be either:

\begin{itemize}
	\item to remove the minimal number of edges to obtain n distinct cliques (thereby gaining the transitive restriction of the relation) from a given non-clique;

	\item to complete the connected subgraph into a clique (thereby gaining the transitive closure of the relation).
\end{itemize}



\subsection*{1.6)}

Define a relation $\sim$ on the set $\mathbb{R}$ of real numbers, by setting $a \sim b \Leftrightarrow b - a \in \mathbb{Z}$. Prove that this is an equivalence relation, and find a 'compelling' description for $\mathbb{R}/\sim$. Do the same for the relation $\approx$ on the plane $\mathbb{R} \times \mathbb{R}$ defined by declaring $(a_1, a_2) \approx (b_1, b_2) \Leftrightarrow b_1 - a_1 \in \mathbb{Z} \text{ and } b_2 - a_2 \in \mathbb{Z}$.

$b - a \in \mathbb{Z}$ means that 2 real numbers differ by an integral amount. This means that the equivalence relation algebraically describes the idea that "with this relation, 2 real numbers are the same iff they have the same fractional component $x$ (or $1 - x$ for negative numbers)". Eg, $4.76 \sim 1024.76 \sim -5.34$, since $-5.34 + 10 = 4.76$, etc.

To make an algebraic quotient of a set by an equivalence relation, we take the function which maps each element to its corresponding equivalence class, in the set (partition) containing these equivalence class. Intuitively, this is similar to keeping only one representative element per equivalence class. For the example class above, we can keep the representative $0.76$. There is such an equivalence class for every fractional part possible, that is, one for every number in $[0, 1[$. The corresponding map is the "real remainder of division modulo 1". This map is well-defined because each real number has only one output for this map, and all real numbers that are equivalent through $\sim$ are mapped to the same value in the output set.

We should also notice that since $0 \sim 1$, this space loops around on itself. Intuitively, if you increase linearly in the input space $\mathbb{R}$, it goes back to $0$ after $0.9999999...$ in the output space. This output space is thus a circle of perimeter $1$.

Similarly, $b_1 - a_1 \in \mathbb{Z} \text{ and } b_2 - a_2 \in \mathbb{Z}$ means that 2 points in the 2D plane are the same iff they differ in each coordinate by an integral amount. This boils down to combining two such loops from the first part of the exercise: one in the $x$ direction and one in the $y$ direction: what this gives is the small square $[0, 1[ \times [0, 1[$, which loops to $x = 0$ (resp. $y = 0$) when $x = 1$ (resp. $y = 1$) is reached. This space functions like a small torus, of area $1$.
\section*{Section 2)}

\subsection*{2.1)}

How many different bijections are there between a set $S$ with $n$ elements and itself?

Any bijection is a choice of a pairs from 2 sets of the same size, where each element is used only once, and each pair has one element from each set. At first there are n choices in each set. If we pick a pair, we pick from $n^2$ possibilities 

There are then $(n-1)^2$ choices of pairs left, etc.

Ccl°: $\prod_{i=1}^{i=n} i^2 = (n!)^2$



\subsection*{2.2)}

Prove that a function has a right-inverse iff it is surjective (can use AC)

Let $f \in (A \to B)$.

\subsubsection*{2.2.a) $\Rightarrow$: Suppose that $f$ has a right-inverse (pre-inverse).}
We have $\exists g \in (B \to A), f \circ g = id_B$

Suppose that $f$ is not a a surjection. This means $\exists b \in B, \nexists a \in A, b = f(a)$

$f(g(b))= id_B (b) = b$ Necessarily, $g(b)$ is such an $a$, so $\exists a \in A, b = f(a)$. Contradiction.

Ccl°:: f is a surjection.

\subsubsection*{2.2.b) $\Leftarrow$: Suppose that f is a surjection.}

$\forall b \in B, \exists a \in A, b = f(a)$

We will construct a pre-inverse for $f$.

The insight here is to realize that a surjection divides its input set into a partition, where each 2-by-2 disjoint subset corresponds to $f^{-1}(\{q\})$, for every $q$ in the output set. More formally, each "fiber" (preimage of a singleton) is a disjoint subset of the input set, and the union of fibers is the input set itself. You can see this in the following diagram:

(add diagram)
1234 to ab
1a 2a (fiber from a)
3b 4b (fiber from b)
https://tex.stackexchange.com/questions/157450/producing-a-diagram-showing-relations-between-sets
https://tex.stackexchange.com/questions/79009/drawing-the-mapping-of-elements-for-sets-in-latex

Using AC, we select a single element from each such fiber. For each $q \in B$, we name $p_q \in f^{-1}(\{q\})$ the chosen element. We define $g$ as $g \in (B \to A), g = (q \mapsto p_q)$. With this, $\forall b \in B, f \circ g (b) = b$, and so $f \circ g = id_A$. Thus, $f$ has a preinverse.

A summary of this idea: all surjection preinverses are simply a choice of a representative for each fiber of the surjection as the output to the respective singleton.



\subsection*{2.3)}

Prove that the inverse of a bijection is a bijection, and that the composition
of two bijections is a bijection.

\subsubsection*{2.3.a) Using the fact that a function is a bijection iff it has a two-sided }inverse (\subsection*{Corollary 2.2)}

 we can see from this defining fact, $f \in (A \to B) \text{bijective } \Leftrightarrow, \exists f^{-1} \in (B \to A), f^{-1} \circ f = id_A \text { and } f \circ f^{-1} = id_B$ that $f$ is naturally $f^{-1}$'s (unique) two-sided inverse, and so $f^{-1}$ is also a bijection.

\subsubsection*{2.3.b) Let be $f \in (A \to B), g \in (B \to C)$, both bijective (hence with }inverses in the respective function spaces). Let $h \in (A \to C), h = g \circ f$ and $h^{-1} \in (C \to A), h^{-1} = f^{-1} \circ g^{-1}$. We have:

$$
\begin{aligned}
h^{-1} \circ h &= (f^{-1} \circ g^{-1}) \circ (g \circ f) \\
               &=  f^{-1} \circ g^{-1}  \circ  g \circ f  \\
               &=  f^{-1} \circ          id_B    \circ f  \\
               &=  f^{-1} \circ                        f  \\
               &=  id_A
\end{aligned}
$$

$$
\begin{aligned}
h \circ h^{-1} &= (g \circ f) \circ (f^{-1} \circ g^{-1}) \\
               &=  g \circ f  \circ  f^{-1} \circ g^{-1}  \\
               &=  g \circ     id_B         \circ g^{-1}  \\
               &=  g \circ                        g^{-1}  \\
               &=  id_C
\end{aligned}
$$

Therefore $h$ and $h^{-1}$ are two-sided inverses of each other, and thus bijections. From this we conclude that the composition of any two bijections is also a bijection.



\subsection*{2.4)}

Prove that ‘isomorphism’ is an equivalence relation (on any set of sets).

\subsubsection*{2.4.a) Problem statement}

Let $\mathcal{A}$ be a set of sets. We define the relation $\simeq$ between the elements of $\mathcal{A}$ as the following:

$$\forall X, Y \in \mathcal{A}, \; X \simeq Y \Leftrightarrow \text {there exists a bijection between $X$ and $Y$}$$

Let us show that $\simeq$ is an equivalence relation.


\subsubsection*{2.4.b) Reflexivity}

For any set $A \in \mathcal{A}$, the identity mapping on $A$ is a bijection. This means that $\forall A \in \mathcal{A}, A \simeq A$, ie, $\simeq$ is reflexive.


\subsubsection*{2.4.c) Symmetry}

$$
\begin{aligned}
\forall X, Y \in \mathcal{A}, \; X \simeq Y & \Rightarrow \exists f      \in (X \to Y) \text{ bijective} \\
                                            & \Rightarrow \exists f^{-1} \in (Y \to X) \text{ bijective} \\
                                            & \Rightarrow Y \simeq X
\end{aligned}
$$

Therefore, $\simeq$ is symmetric.


\subsubsection*{2.4.d) Transitivity}

Let be $X, Y, Z \in \mathcal{A}$.
Suppose that $X \simeq Y$ and $Y \simeq Z$.
This means $\exists f \in (X \to Y), g \in (Y \to Z)$, both bijections.
Let be $h \in (X \to Z), h = g \circ f$. $h$ is also a bijection since the composition of \subsection*{two bijections is also a bijection (exercise 2.3)}


The existence of $h$ implies $X \simeq Z$.

Therefore $\simeq$ is transitive.


\subsubsection*{2.4.e) Conclusion}

Isomorphism, $\simeq$, is a relation on an arbitrary set (of sets) which is always reflexive, symmetric and transitive. It is thus an equivalence relation.



\subsection*{2.5)}

Formulate a notion of epimorphism and prove that epimorphisms and surjections are equivalent.

See "notes" file: section "Proofs of mono/inj and epi/surj equivalence".


\subsection*{2.6)}

With notation as in Example 2.4, explain how any function $f \in (A \to B)$ determines a section of $\pi_A$.

A section is the preinverse of a surjection. Here, the surjection in question is $\pi_A$ the projection of $A \times B$ onto $A$.

Let $f \in (A \to B)$.

% We remind that all functions are technically themselves sets, more precisely, they are relations between pairs of sets, hence $(A \to B) \subseteq (A \times B)$. We first define the map $\iota \in ((A \to B) \to (A \times B)), \; \iota = (f \mapsto \Gamma_f)$, which canonically injects a function from the function space $(A \to B)$ into the cartesian product $(A \times B)$, by simply keeping every pair $(a, f(a))$ as-is. $\Gamma_f$ is the \textit{graph} or \textit{graphical representation} of $f$.

We now consider the function which maps an input $a \in A$ of $f$ to its "geometric representation" (its coordinates in the enclosing space $A \times B$, corresponding to a point of the graph $\Gamma_f$). 
$$\hat{f} \in (A \to (A \times B)), \hat{f} = ( \; a \mapsto (a, f(a)) \; )$$
We notice that $\hat{f}(A) = \Gamma_f$.

Naturally, $\pi_A \circ \hat{f} = (a \mapsto a) = id_A$, therefore, $\hat{f}$ is a pre-inverse (section) of $\pi_A$.

This set of relationships can be expressed in the following commutative diagram:

% incorrect because of iota's nature; shifting rank (sets of sets to simple sets) is a bad idea
% https://tikzcd.yichuanshen.de/#N4Igdg9gJgpgziAXAbVABwnAlgFyxMJZARgBoAGAXVJADcBDAGwFcYkQAdDgIxgHMsYYAFt6OAE5YAHgF8AggAIueYfAUAKRcogKAQgEolXDhvqkFAM31cYYKCLGTZIGaXSZc+QinKli1OiZWdi5eASFRCWkZBS1jBXojW3tIpxkXNxAMbDwCIgAmPwCGFjZETh5+QQco2Q04nB0DIy5LJLsatIz3HK8iMnzioLKKsOrU6K4AcXphUQB9CyM4Zm44GBx6oxU1ZuMTdTNLQ-1DGw6J5xkAmCg+eCJQC3EIYSRfEEakMkDS9jR5nIQDRGPReIwAAoeXLeECSPgACxwwJACJg9Cg7BwAHcIGiMQhXE8Xm9EB8vohCr9guUAcBNNsmvp0iCwTBIdC+uVGDALMiaPjMeUcXj0VBCZlnq9vjQKQBmGglGkVBFiYAWFkgUHgqG9PLleFIlE4ehYRjsBEQCAAaxRcARWD5MpAvDs7yJIClpKp8sVwxCHHwJpR2vZus8+q1vP5qLFSDAzEYjFlpvN5UtNu6npJSAVnwgzqVIy4aCwgJDbI5ethhpjgqxuMFCBo9sdyMQxGuMiAA
% \begin{tikzcd}
%                                                                              & {\begin{matrix}A \times (A \to B) \\ (a, f)\end{matrix}} \arrow[ld, "p_A"', two heads] \arrow[rd, "p_{(A \to B)}", two heads] &                                                                      \\
% \begin{matrix} A \\ a \end{matrix} \arrow[rd, "\hat{f}"', hook, shift right] &                                                                                                                               & \begin{matrix} (A \to B) \\ f \end{matrix} \arrow[ld, "\iota", hook] \\
%                                                                              & {\begin{matrix}\Gamma_f \subseteq (A \times B) \\ (a, f(a)) \end{matrix}} \arrow[lu, "\pi_A"', two heads, shift right]          &                                                                     
% \end{tikzcd}
%Where $p_A$ and $p_B$ are the appropriate left and right projectors for the corresponding space.


%this one is much simpler and neater
% https://tikzcd.yichuanshen.de/#N4Igdg9gJgpgziAXAbVABwnAlgFyxMJZARgBoAGAXVJADcBDAGwFcYkQAdDgIxgHMsYYAFt6OAE5YAHgF8ABFwDi9YaID6AMwUc4zbnBg45AQW15h8OQCFtXOQAp6pORscBKN9phgoIsZNkQGVJ0TFx8QhRyUmJqOiZWdi5eASFRCWl5Uy47ei8fPwzA4NDsPAIiACYYuIYWNkROHn5BQoD5GxyOF3d833T2oLiYKD54IlANcQhhJDIQHAgkaPj6pI4ACzFgDRkQGjgNrA0cJABaeZx6LEZ2DYgIAGsgkJApmeWaRbmDo5Pz+YbGD0KDsHAAdwgQJBCBodUSjS4aCwamML0m01miEuS0Q1VWCLe+xAjHovEYAAUwuVIiBJHwNqcSm9MZ8Frj8dDQY0IVDgVBYSBDsdTni4QkGk1kWorEMZEA
\begin{tikzcd}
                                                                                            & {\begin{matrix} \Gamma_f \subseteq A \times B \\ (a, f(a)) \end{matrix}} \arrow[ld, "\pi_A", two heads, shift left] \arrow[rd, "\pi_B", two heads, shift right=2] &                                       \\
\begin{matrix} A \\ a \end{matrix} \arrow[ru, "\hat{f}", hook, shift left] \arrow[rr, "f"'] &                                                                                                                                                                 & \begin{matrix} B \\ f(a) \end{matrix}
\end{tikzcd}


PS: see "On sections and fibers" in the "notes" file for a worked example.



\subsection*{2.7)}

Let $f = (A \to B)$ be any function. Prove that the graph $\Gamma_f$ of $f$ is isomorphic to $A$.

Using the elements from the previous exercise, we know that $\hat{f}$ is injective from $A$ into $A \times B$. This property is inherited to any restriction of the codomain $Z \subseteq B$, and corresponding implied restriction of the domain to $Y = f^{-1}(Z) \subseteq A$. In particular, here, $Y = A$ and $Z = \Gamma_f = \hat{f}(A)$. We now consider $\overline{f} \in (A \to \Gamma_f), \overline{f} = (a \mapsto \hat{f}(a))$. We can see that $\overline{f}$ is injective from being a restriction of an injective function to a smaller codomain. We also know that $\overline{f}$ is surjective, since its domain is its image. Therefore, $\overline{f}$ is a bijection. This means that $A \simeq \Gamma_f$.



\subsection*{2.8)}

Describe as explicitly as you can all terms in the canonical decomposition of the function $f \in (\mathbb{R} \to \mathbb{C})$ defined by $f = (r \mapsto e^{2 \pi i r})$. (This exercise matches one assigned previously, which one?)

Firstly, elements of $\mathbb{R}$ are equivalent by this map (they have the same output) if they vary by $1$ from each other. This is a reference to the equivalence relation $\sim$ in exercise 1.6. Therefore, we will use $\mathbb{R}/\sim \simeq S^1$ in our decomposition. Obviously, the map from $(\mathbb{R} \to \mathbb{R}/\sim)$, which maps each element of $\mathbb{R}$ to respective their equivalence class is a surjection (since there's no empty equivalence class).

Secondly, as mentioned, we have a bijection $\tilde{f}$ between $\mathbb{R}/\sim$ and $S^1$, the circle group of unit complex numbers, namely $\tilde{f} = (x \mapsto e^{2 \pi i x}$, where each element $x$ of $\mathbb{R}/\sim$ can be understood to correspond to a (class representative) value in the interval $[0, 1[$.

Finally, we do the canonical injection of $S^1$ into its superset $\mathbb{C}$.



\subsection*{2.9)}

Show that if $A \simeq A'$ and $B \simeq B'$ , and further $A \cap B = \emptyset$ and $A' \cap B' = \emptyset$, then $A \cup B \simeq A' \cup B'$. Conclude that the operation $A \coprod B$ (as described in §1.4) is well-defined up to isomorphism.

We suppose the aforementioned.

Let $f_A$ be a bijection from $A \to A'$, and $f_B$ be a bijection from $B \to B'$.

We define the following:

$$
f \in (A \cup B \to A' \cup B'),
\text{ such that }
\begin{cases}
	\forall a \in A, \; f(a) = f_A(a) \\
	\forall b \in B, \; f(b) = f_B(b)
\end{cases}
$$

This function is a well-defined function, since $A \cap B = \emptyset$: every element of the domain has one, and only one, possible image.

Similarly, we define:

$$
g \in (A' \cup B' \to A \cup B),
\text{ such that }
\begin{cases}
	\forall a \in A', \; g(a) = f_A^{-1}(a) \\
	\forall b \in B', \; g(b) = f_B^{-1}(b)
\end{cases}
$$

Similarly, because $A' \cap B' = \emptyset$, $g$ is well-defined.

Let us study $g \circ f$. We have:
$$
\begin{cases}
	\forall a \in A, \; g(f(a)) = f_A^{-1}(f_A(a)) = a \\
	\forall b \in B, \; g(f(b)) = f_B^{-1}(f_B(b)) = b
\end{cases}
$$

Hence, $g \circ f = id_{A \cup B}$.
Similarly, $f \circ g = id_{A' \cup B'}$.
Therefore, $g = f^{-1}$, $f$ is a bijection, and $A \cup B \simeq A' \cup B'$.

We'll now do a shift in notation. Let be some arbitrary sets $A$ and $B$. Let be $A_1, A_2, B_1, B_2$ such that $A_1 = \{ 1 \} \times A$, $A_2 = \{ 2 \} \times A$, $B_1 = \{ 1 \} \times B$, and $B_2 = \{ 2 \} \times B$. This means $A \simeq A_1$, $A \simeq A_2$, $B \simeq B_1$, and $B \simeq B_2$. It also means $A_1 \cap A_2 = \emptyset$ and $B_1 \cap B_2 = \emptyset$. From the above, this implies $A_1 \cup B_1 \simeq A_2 \cup B_2$.

This means that the disjoint union of $A$ and $B$ is indeed well-defined, up to isomorphism: so long as 2 respective copies of $A$ and $B$ are made in a way that their intersection is empty, the 2 respective unions of 1 copy each will be isomorphic.



\subsection*{2.10)}

Show that if $A$ and $B$ are finite sets, then $|B^A| = |B|^{|A|}$.

The number of $|B^A|$ functions in $B^A = (A \to B)$ can be counted in the following way.

For each element $a$ of $A$, of which there are $|A|$, we can pick any element of $|B|$ as the image. This means $|B| \times ... \times |B|$, a total of $|A|$ times. Hence, $|B^A| = |B|^{|A|}$.



\subsection*{2.11)}

In view of Exercise 2.10, it is not unreasonable to use $2^A$ to denote the set of functions from an arbitrary set $A$ to a set with $2$ elements (say $\mathbb{B} = \{0, 1\}$). Prove that there is a bijection between $2^A$ and the power set $\mathcal{P}(A)$ of $A$.

Simply put, every subset $A_i$ of $A$ is built through a series of $|A|$ choices: for each element $a$ in $A$, do we keep the element $a$ in our subset $A_i$ (output $1$) or do we remove it (output $0$) ? It is then easy to see that such a series of choices can easily be encoded as a unique function in $A \to \mathbb{B}$. The totality of such series of choices thus corresponds both to the space $A \to \mathbb{B}$, and to the powerset $\mathcal{A}$, and there is a bijection between the two. 
\section*{Section 3)}

\subsection*{3.1)}

Let $\mathcal{C}$ be a category. Consider a structure $\mathcal{C}^{op}$ with:
 - $Obj(\mathcal{C}^{op}) \coloneqq Obj(\mathcal{C})$;
 - for $A$, $B$ objects of $\mathcal{C}^{op}$ (hence, objects of $\mathcal{C}$), $Hom_{\mathcal{C}^{op}} (A, B) \coloneqq Hom_{\mathcal{C}} (B, A)$
Show how to make this into a category.

\subsubsection*{3.1.a) Composition}

First, to make things clearer and more rigorous, let us distinguish composition in $\mathcal{C}$ as $\circ$ and composition in $\mathcal{C}^{op}$ as $\star$. We define $\star$ as:
$$
\begin{aligned}
	& \forall f \in Hom_{\mathcal{C}^{op}} (B, A) = Hom_{\mathcal{C}} (A, B), \\
	& \forall g \in Hom_{\mathcal{C}^{op}} (C, B) = Hom_{\mathcal{C}} (B, C), \\
	& \exists h \in Hom_{\mathcal{C}^{op}} (C, A) = Hom_{\mathcal{C}} (A, C), \\
	& f \star g \coloneqq g \circ f = h
\end{aligned}
$$

We will now show that $\mathcal{C}^{op}$ with $\star$ verifies the other axioms of a category (namely identity and assossiativity of composition).

\subsubsection*{3.1.b) Identity}

Since $\mathcal{C}$ is a category, since $\mathcal{C}^{op}$ has the same objects, and since, by definition, for all object $A$, we have $Hom_{\mathcal{C}^{op}} (A, A) = Hom_{\mathcal{C}} (A, A)$, we can take every $id_A \in Hom_{\mathcal{C}}(A, A)$ as the same identity in $\mathcal{C}^{op}$. We can verify that this is compatible with $\star$:

$$
\begin{aligned}
	\forall A, B & \in Obj (\mathcal{C})        &=& \;  Obj (\mathcal{C}^{op})        , \\
	\exists id_A & \in Hom_{\mathcal{C}} (A, A) &=& \;  Hom_{\mathcal{C}^{op}} (A, A) , \\
	\exists id_B & \in Hom_{\mathcal{C}} (B, B) &=& \;  Hom_{\mathcal{C}^{op}} (B, B) , \\
	\forall f    & \in Hom_{\mathcal{C}} (A, B) &=& \;  Hom_{\mathcal{C}^{op}} (B, A) , \\
	f            & =   f    \circ id_A          &=& \;  id_A \star f                  , \\
	f            & =   id_B \circ    f          &=& \;  f    \star id_B                 \\
\end{aligned}
$$

\subsubsection*{3.1.c) Associativity}

Using associativity in $\mathcal{C}$, we have:

$$
\begin{aligned}
	\forall A, B, C, D & \in Obj (\mathcal{C})        &=& \;  Obj (\mathcal{C}^{op})        , \\
	\forall f          & \in Hom_{\mathcal{C}} (A, B) &=& \;  Hom_{\mathcal{C}^{op}} (B, A) , \\
	\forall g          & \in Hom_{\mathcal{C}} (B, C) &=& \;  Hom_{\mathcal{C}^{op}} (C, B) , \\
	\forall h          & \in Hom_{\mathcal{C}} (C, D) &=& \;  Hom_{\mathcal{C}^{op}} (D, C) , \\
\end{aligned}
$$
$$
\begin{aligned}
	h \star (g \star f) &=&  h \star (f  \circ g) \\
						&=& (f \circ  g) \circ h  \\
						&=&  f \circ  (g \circ h) \\
						&=&  (g \circ h) \star f  \\
						&=&  (h \star g) \star f  \\
\end{aligned}
$$

Therefore, $\star$ is associative.

We conclude that $\mathcal{C}^{op}$ is a category.



\subsection*{3.2)}

If $A$ is a finite set, how large is $End_{\text{Set}}(A)$ ?

We know that, in Set, $End_{\text{Set}}(A) = (A \to A) = A^A$. From a previous exercise, we know that $|B^A| = |B|^|A|$, therefore $|End_{\text{Set}}(A)| = |A|^|A|$.



\subsection*{3.3)}

Formulate precisely what it means to say that "$1_a$ is an identity with respect to composition" in Example 3.3, and prove this assertion.

Example 3.3 is that of a category over a set $S$ with a (reflexive, transitive) relation $\sim$, where the objects of the category are the elements of $S$, and the homset between two elements $a$ and $b$ is the singleton $(a,b)$ if $a \sim b$, and $\emptyset$ otherwise. Composition $\circ$ is given by transitivity of $\sim$, where $(b,c) \circ (a,b) = (a,c)$. Reflexivity gives the identities ($id_a = (a,a)$ for any element $a$).

In this context, to say that "$1_a$ is an identity with respect to composition" means that we can cancel out an element of the form $(a,a)$ from a composition.

Formally, we have:

$$\forall a,b \in S, (b,b) \circ (a,b) = (a,b) = (a,b) \circ (a,a)$$

proving that $(b,b)$ is indeed a post-identity, and $(a,a)$ a pre-identity, in this context.



\subsection*{3.4)}

Can we define a category in the style of Example 3.3, using the relation $<$ on the set $\mathbb{Z}$ ?

(Description of example 3.3 in the exercise 3.3 just above.)

Naively, saying like in example 3.3 "there is a singleton homset $\text{Hom}(a,b)$ each time we have $a < b$", we cannot define such a category, since $<$ is not reflexive, and we would thus lack identity morphisms.

However, in a roundabout way, we can define a category over the \textit{negation} of $<$: "there is a singleton homset $\text{Hom}(a,b)$ each time we DO NOT have $a < b$". Namely this corresponds to the relation $\ge$, which is, itself, reflexive, transitive (and antisymmetric), and is a valid instance of the kind of category presented in example 3.3.

In fact, the pair $(\mathbb{Z}, \geq)$ is an instance of what is called a "totally ordered set", which is a more restrictive kind of "partially ordered set" (also called "poset" for short). Consequently, this kind of category is called a "poset category".



\subsection*{3.5)}

Explain in what sense Example 3.4 is an instance of the categories considered in Example 3.3.

(Description of example 3.3 in the exercise 3.3 just above.)

Example 3.4 describes a category $\hat{S}$ where the objects are the subsets of a set $S$ (equivalently: elements of the powerset $\mathcal{P}(S)$ of $S$), and morphisms between two subsets $A$ and $B$ of $S$ are singleton (or empty) homsets based on whether the inclusion is true (or false).

Inclusion of sets, $\subset$, is also an order relation, this time between the elements of a set of sets (here, $\mathcal{P}(S)$). This means inclusion is reflexive, transitive, and antisymmetric. This makes $\hat{S}$ a poset category, and thus another instance of example 3.3. 



\subsection*{3.6)}

Define a category $V$ by taking $Obj(V) = \mathbb{N}$, and $Hom_V(n, m) = Mat_\mathbb{R}(m, n)$, the set of $m \times n$ matrices with real entries, for all $n, m \in \mathbb{N}$. (I will leave the reader the task to make sense of a matrix with 0 rows or columns.) Use product of matrices to define composition. Does this
category 'feel' familiar ?

The formulation of the exercise is strange. It says to use the product of matrices to define composition, and to have homsets be sets of matrices, but objects of the category are supposed to be integers. I don't know of any matrix with real entries that maps an integer to an integer in this way.

We thus infer that the meaning of the exercise can be one of two things.

Either we suppose the set of objects could rather be understood as "something isomorphic to $\mathbb{N}$", ie, the collection of real vector spaces with finite bases (ie, $\forall n \in \mathbb{N}, \mathbb{R}^n$). In which case, this is just the category of real vector spaces with finite basis (and linear maps as morphisms), which is a subcategory of the category real vector spaces (commonly called $Vect_{\mathbb{R}}$). In this context, any morphism starting from $0 \simeq \mathbb{R}^0 = \{0\}$ is just the injection of the origin into the codomain; and any morphism ending at $0$ is the mapping of all elements to the origin.

Otherwise, we understand this as "yes, the objects of the category are integers: this means you should ignore the actual content of the matrices, and instead consider only their effect on the dimensionality of domains and codomains". In this case, this category is a complete directed graph over $\mathbb{N}$ where each edge corresponds to the change in dimension (from domain to codomain) caused by a given linear map.



\subsection*{3.7)}

Define carefully objects and morphisms in Example 3.7, and draw the diagram corresponding to composition.

Example 3.7 (on coslice categories) refers to example 3.5 (on slice categories). Let's go over slice categories (since example 3.5 asks the reader to "check all [their various properties]").

3.7.1) Slice categories

Slice categories are categories made by singling out an object (say $A$) in some parent (larger) category (say $\mathcal{C}$), and studying all morphisms into that object. These morphisms become the objects of a new category (ie, for any $Z$ of $\mathcal{C}$, $f \in (Z \to A)$ is an object of the slice category, called $\mathcal{C}_A$ in this context). In the slice category, morphisms are defined as those morphism in $\mathcal{C}$ that preserve composition between 2 morphisms into $A$.

Note that there exist pairs of morphisms $f_1 \in (Z_1 \to A)$ and $f_2 \in (Z_2 \to A)$ between which there is no morphism that exists in the slice category. One such example we can make is in $(Vect_\mathbb{R})_{\mathbb{R}^2}$. If we take the maps:

$$f_1 = \begin{bmatrix} 1 & 0 \\ 0 & 0 \end{bmatrix} \in \mathcal{L}(\mathbb{R}^2)$$
$$f_2 = \begin{bmatrix} 0 & 0 \\ 0 & 1 \end{bmatrix} \in \mathcal{L}(\mathbb{R}^2)$$

There exists no map $\sigma$ such that the following diagram commutes (since the output of $f_1$ will always be null in its second coordinate, and the output of $f_2$ will always be null in the first):

% https://tikzcd.yichuanshen.de/#N4Igdg9gJgpgziAXAbVABwnAlgFyxMJZABgBpiBdUkANwEMAbAVxiRAB12BbOnACwBGA4ACUAvgD0ATCDGl0mXPkIoAjOSq1GLNpx78hoyTLkLseAkTKrN9Zq0QduvQcPHTZmmFADm8IqAAZgBOEFxIZCA4EEhS1HY6joEA+qog1Ax0AjAMAAqKFiogwVg+fDiy8iAhYUjqUTGIcVr2bCkmVTXhiJHRdfHaDk7YPjyeYkA
\begin{tikzcd}
\mathbb{R}^2 \arrow[d, "f_1"'] \arrow[r, "\sigma"] & \mathbb{R}^2 \arrow[ld, "f_2"] \\
\mathbb{R}^2                                       &                               
\end{tikzcd}

Now, let us prove that $\mathcal{C}_A$ is indeed a category for an arbitrary object $A$ of an arbitrary category $\mathcal{C}$.

3.7.1.a) Identity

A generic identity morphism is expressed diagrammatically in $\mathcal{C}_A$ as:

% https://tikzcd.yichuanshen.de/#N4Igdg9gJgpgziAXAbVABwnAlgFyxMJZABgBpiBdUkANwEMAbAVxiRAC0QBfU9TXfIRQBGclVqMWbTjz7Y8BImWHj6zVohABBbuJhQA5vCKgAZgCcIAWyRkQOCEgBM1NVM2mQ1BnQBGMBgAFfgUhEHMsAwALHG5eEAtrJFF7R0QXCXU2T1kEyxtEOwdk10kNECwoAH0ZeMSCjOL07wgINCVSU0Y4GHEffyCQwTYI6NjSrM1Kqp0uCi4gA
\begin{tikzcd}
Z \arrow[d, "f"'] \arrow[r, "id_Z"]                    & Z \arrow[ld, "f"] \\
A \arrow["id_A"', loop, distance=2em, in=305, out=235] &                  
\end{tikzcd}

We can see that since $f = f \circ id_Z$ in $\mathcal{C}$, this is compatible with the definition of a (pre-/right-)unit morphism in $\mathcal{C}_A$. Also, since the only maps post-$f$ are maps from $A \to A$, we have $id_A$ as the (post-/left-)unit for every morphism $f$ (ie, $f = id_A \circ f$. 

3.7.1.b) Composition

Taking 3 objects of the slice category ($f_1 \in (Z_1 \to A)$, $f_2 \in (Z_2 \to A)$ and $f_3 \in (Z_3 \to A)$), and two morphisms ($\sigma_A$ mapping $f_1$ to $f_2$ via a $\mathcal{C}$-morphism $\sigma \in (Z_1 \to Z_2)$, and $\tau_A$ mapping $f_2$ to $f_3$ via a $\mathcal{C}$-morphism $\tau \in (Z_2 \to Z_3)$), we have that $f_1 = f_2 \circ \sigma$ and $f_2 = f_3 \circ \tau$. This is expressed as the following commutative diagram.

% https://tikzcd.yichuanshen.de/#N4Igdg9gJgpgziAXAbVABwnAlgFyxMJZABgBpiBdUkANwEMAbAVxiRAC0B9ARhAF9S6TLnyEU3clVqMWbLgCZ+gkBmx4CReZOr1mrRB04BmJULWiiE7lN2yDAQX5SYUAObwioAGYAnCAFskMhAcCCQJEAY6ACMYBgAFYXUxSJgvHBAdGX0QAB1c7Fd-OlMQXwDw6lCkLWk9NnycOiZS8sDEYOrEIyz6gy8eTMiYuMTzDQMfLFcACwyBbz92iK6eursyzkUFsqWaqrDu3o2Bkz4KPiA
\begin{tikzcd}
Z_1 \arrow[r, "\sigma"] \arrow[rd, "f_1"'] & Z_2 \arrow[r, "\tau"] \arrow[d, "f_2"] & Z_3 \arrow[ld, "f_3"] \\
                                           & A                                      &                      
\end{tikzcd}

Composition of morphisms is then defined as $\tau_A \circ_A \sigma_A$ as a mapping from $f_1$ to $f_3$, such that $f_1 = f_3 \circ (\tau \circ \sigma)$. This can be understood through the following commutative diagram:

% https://tikzcd.yichuanshen.de/#N4Igdg9gJgpgziAXAbVABwnAlgFyxMJZABgBpiBdUkANwEMAbAVxiRAC0B9ARhAF9S6TLnyEUAJnJVajFmy4BmfoJAZseAkW6lu0+s1aIQAQX7SYUAObwioAGYAnCAFskZEDghJJMg2zs8INQMdABGMAwACsIaYiAOWJYAFjjK9k6uiNoeXog++nJGAUoC6S5u1J5I2QWGIAA69Th0TI0AxlgObY3Yls50ZnxAA
\begin{tikzcd}
Z_1 \arrow[rd, "f_1"'] \arrow[rr, "\tau \circ \sigma"] &   & Z_3 \arrow[ld, "f_3"] \\
                                                       & A &                      
\end{tikzcd}

Which commutes, because we have:

$$
\begin{aligned}
	f_1 &=&  f_2              \circ \sigma  \\
		&=& (f_3 \circ  \tau) \circ \sigma  \\
		&=&  f_3 \circ (\tau  \circ \sigma)
\end{aligned}
$$

Thus, we have a working composition of morphisms.

3.7.1.c) Associativity

We take 4 objects of the slice category ($f_1 \in (Z_1 \to A)$, $f_2 \in (Z_2 \to A)$, $f_3 \in (Z_3 \to A)$ and  $f_4 \in (Z_4 \to A)$), and three morphisms ($\sigma_A$ mapping $f_1$ to $f_2$, $\tau_A$ mapping $f_2$ to $f_3$, and $\upsilon_A$ mapping $f_3$ to $f_4$). Using composition defined as above, we have

$$
f_1 = f_4 \circ ( \upsilon \circ (\tau  \circ \sigma))
	= f_4 \circ ((\upsilon \circ  \tau) \circ \sigma )
\Rightarrow
   \upsilon_A \circ (\tau_A  \circ \sigma_A)
= (\upsilon_A \circ  \tau_A) \circ \sigma_A
$$
Through associativity in $\mathcal{C}$.


3.7.2) Coslice categories

A coslice category $\mathcal{C}^A$ is similar, except it takes the morphisms coming \textit{from} a chosen object $A$, rather than those going \textit{to} this object $A$. Below is a commutative diagram in the style of the one of the textbook for slice categories.

% https://tikzcd.yichuanshen.de/#N4Igdg9gJgpgziAXAbVABwnAlgFyxMJZARgBoAGAXVJADcBDAGwFcYkQBBEAX1PU1z5CKcqWLU6TVuwBaAfWI8+IDNjwEiAJjESGLNohDzNPCTCgBzeEVAAzAE4QAtklEgcEJGRCN6AIxhGAAUBdWEQeywLAAscEBo9aUNbBSU7RxdEbw8kbUl9dgAdQuwLJ3o0kAdnVxocxDzEgyq5E25KbiA
\begin{tikzcd}
                         & A \arrow[ld, "f_1"'] \arrow[rd, "f_2"] &     \\
Z_1 \arrow[rr, "\sigma"] &                                        & Z_2
\end{tikzcd}

We can similarly show that this also defines a category.

3.7.2.a) Identity

A generic identity morphism is expressed diagrammatically in $\mathcal{C}^A$ as:

% https://tikzcd.yichuanshen.de/#N4Igdg9gJgpgziAXAbVABwnAlgFyxMJZABgBpiBdUkANwEMAbAVxiRAEEQBfU9TXfIRQBGUsKq1GLNgC1uvEBmx4CRMuOr1mrRCDlcJMKAHN4RUADMAThAC2SMiBwQkoydrYWQ1BnQBGMAwACvwqQiAMMBY48pY29oiOzkgATJpSOiBePHF2qdTJiG6+AcGhgmxWWMYAFjHpHrpYUAD6+grWeYkFLt0REBBoRMIAHGQWjHAwEiWBIcoVulW19e7STa2cBlxAA
\begin{tikzcd}
A \arrow[rd, "f"] \arrow[d, "f"] \arrow["id_A"', loop, distance=2em, in=125, out=55] &   \\
Z \arrow[r, "id_Z"']                                                                 & Z
\end{tikzcd}

We can see that since $f = id_Z \circ f$ in $\mathcal{C}$, this is compatible with the definition of a (post-/left-)unit morphism in $\mathcal{C}^A$. Also, since the only maps pre-$f$ are maps from $A \to A$, we have $id_A$ as the (pre-/right-)unit for every morphism $f$ (ie, $f = f \circ id_A$. 

3.7.2.b) Composition

Taking 3 objects of the slice category ($f_1 \in (A \to Z_1)$, $f_2 \in (A \to Z_2)$ and $f_3 \in (A \to Z_3)$), and two morphisms ($\sigma^A$ mapping $f_1$ to $f_2$ via a $\mathcal{C}$-morphism $\sigma \in (Z_1 \to Z_2)$, and $\tau^A$ mapping $f_2$ to $f_3$ via a $\mathcal{C}$-morphism $\tau \in (Z_2 \to Z_3)$), we have that $f_1 = \sigma \circ f_2$ and $f_2 = \tau  \circ f_3$. This is expressed as the following commutative diagram.

% https://tikzcd.yichuanshen.de/#N4Igdg9gJgpgziAXAbVABwnAlgFyxMJZABgBoBGAXVJADcBDAGwFcYkQAtAfXJAF9S6TLnyEU5CtTpNW7bgCZ+gkBmx4CReZJoMWbRJy4BmJULWiiE4lN2yDAQX5SYUAObwioAGYAnCAFskMhAcCCQJaT12AB1o7Fd-ehAaRnoAIxhGAAVhdTEQRhgvHFMQXwDwmlCkLUi7EFicemZkgvTMnPMNA0Li0vLAxCMqsMRg230ynlbUjOzciwMfLFcACxKBbz9B4ZDRiIn2Ly5FTbLtpF3qxFrDg2OTPko+IA
\begin{tikzcd}
                        & A \arrow[ld, "f_1"'] \arrow[d, "f_2"] \arrow[rd, "f_3"] &     \\
Z_1 \arrow[r, "\sigma"] & Z_2 \arrow[r, "\tau"]                                   & Z_3
\end{tikzcd}

Composition of morphisms is then defined as $\tau^A \circ^A \sigma^A$ as a mapping from $f_1$ to $f_3$, such that $f_3 = (\tau \circ \sigma) \circ f_1$. This can be understood through the following commutative diagram:

% https://tikzcd.yichuanshen.de/#N4Igdg9gJgpgziAXAbVABwnAlgFyxMJZABgBoBGAXVJADcBDAGwFcYkQAtAfXJAF9S6TLnyEUAJgrU6TVu24BmfoJAZseAkXKli0hizaIQAQX7SYUAObwioAGYAnCAFskZEDghJtMg+wA6-jj0zAAEgQDGWA4R4f7Yls70IDSM9ABGMIwACsIaYiCMMHY4yvZOroiSHl6I7mmZOXmi7A5YlgAWpTT6ckZ2PGUgji5I1Z7ePbKGw1xKfJR8QA
\begin{tikzcd}
                                    & A \arrow[ld, "f_1"'] \arrow[rd, "f_3"] &     \\
Z_1 \arrow[rr, "\tau \circ \sigma"] &                                        & Z_3
\end{tikzcd}

Which commutes, because we have:

$$
\begin{aligned}
	f_3 &=&  \tau \circ                f_2  \\
		&=&  \tau \circ (\sigma  \circ f_1) \\
		&=& (\tau \circ  \sigma) \circ f_1
\end{aligned}
$$

Thus, we have a working composition of morphisms.

3.7.2.c) Associativity

We take 4 objects of the slice category ($f_1 \in (A \to Z_1)$, $f_2 \in (A \to Z_2)$, $f_3 \in (A \to Z_3)$ and  $f_4 \in (A \to Z_4)$), and three morphisms ($\sigma^A$ mapping $f_1$ to $f_2$, $\tau^A$ mapping $f_2$ to $f_3$, and $\upsilon^A$ mapping $f_3$ to $f_4$). Using composition defined as above, we have

$$
f_4 = ( \upsilon \circ (\tau  \circ \sigma)) \circ f_1
	= ((\upsilon \circ  \tau) \circ \sigma ) \circ f_1
\Rightarrow
   \upsilon^A \circ (\tau^A  \circ \sigma^A)
= (\upsilon^A \circ  \tau^A) \circ \sigma^A
$$
Through associativity in $\mathcal{C}$.


\subsection*{3.8)}

A subcategory $\mathcal{C'}$ of a category $\mathcal{C}$ consists of a collection of objects of $\mathcal{C}$, with morphisms $Hom_\mathcal{C'} (A, B) \subseteq Hom_\mathcal{C} (A, B)$ for all objects $A$, $B$ in $Obj(\mathcal{C'})$, such that identities and compositions in $\mathcal{C}$ make $\mathcal{C'}$ into a category. A subcategory $\mathcal{C'}$ is \textit{full} if $Hom_\mathcal{C'} (A, B) = Hom_\mathcal{C} (A, B)$ for all $A$, $B$ in $Obj(\mathcal{C'})$. Construct a category of \textit{infinite sets} and explain how it may be viewed as a full subcategory of $\mathbf{Set}$.

To put it less technically, a "subcategory" $\mathcal{C'}$ is just "picking only certain items of a base category $\mathcal{C}$, and making sure that things stay closed uneder morphism composition". It is "full" if \textit{all} morphisms between the objects that remain are also conserved.

We can construct a category $\mathbf{InfSet}$ of infinite sets by taking all the objects $A$ of $\mathbf{Set}$ such that $\nexists n \in \mathbb{N}, |A| = n$, and only homsets between these objects. This is clearly a subcategory of $\mathbf{Set}$, since it inherits all identity morphisms, composition works the same, and so does associativity; also, restricting the choice of homsets makes it so that the category is closed (you can't reach a finite set via a homset that went from an infinite to a finite set).

For this category to not be full, there would need to be some homset that loses a morphism, or fully disappears, in the ordeal. However, there is no restriction as to the kind of morphism that is conserved, so any homset that is kept is identical to its original version. Finally, homsets between infinite sets are also infinite sets, so they don't disappear in this operation.

Consequently $\mathbf{InfSet}$ defined as such is a full subcategory of $\mathbf{Set}$.


\subsection*{3.9)}

An alternative to the notion of multiset is obtained by considering sets endowed with equivalence relations; equivalent elements are taken to be multiple instances of elements 'of the same kind'. Define a notion of morphism between such enhanced sets, obtaining a category $\mathbf{MSet}$ containing (a 'copy' of) $\mathbf{Set}$ as a full subcategory. (There may be more than one reasonable way to do this! This is intentionally an open-ended exercise.) Which objects in $\mathbf{MSet}$ determine ordinary multisets as defined in §2.2, and how? Spell out what a morphism of multisets would be from this point of view. (There are several natural notions of morphisms of multisets. Try to define morphisms in MSet so that the notion you obtain for ordinary multisets captures your intuitive understanding of these objects.) [§2.2, §3.2, 4.5]


\subsection*{3.10)}

Since the objects of a category $\mathcal{C}$ are not (necessarily) sets, it is not clear how to make sense of a notion of 'subobject' in general. In some situations it does make sense to talk about subobjects, and the subobjects of any given object $A$ in $\mathcal{C}$ are in one-to-one correspondence with the morphisms $A \to \Omega$ for a fixed, special object $\Omega$ of $\mathcal{C}$, called a subobject classifier. Show that $\mathbf{Set}$ has a subobject classifier.


\subsection*{3.11)}

Draw the relevant diagrams and define composition and identities for the category $\mathcal{C}^{A,B}$ mentioned in Example 3.9. Do the same for the category $\mathcal{C}^{\alpha, \beta}$ mentioned in Example 3.10. [§5.5, 5.12]
\newpage
\part{Extra exercises by/for the group}

\section*{Chapter I) 1) Set notation)}

Write the following in set notation (as a list of numbers, and algebraically):
\begin{itemize}
	\item the set of all odd integers
	\item the set of all integers that are not multiples of 3
	\item the set of integers from 10 (included) to 20 (included)
	\item the set of integers from 10 (included) to 20 (excluded)
	\item the set of pairs of integers with both elements of the same value
	\item the set of triplets of real numbers that together sum to 1
	\item the set of pairs of positive real numbers that together sum to 1
	\item the set of $n$-tuplets (for any $n$) of real number that together sum to 1
	\item the set of all natural numbers such that there exists at least one triplet of positive even numbers which are all different and which sum to that number.
\end{itemize}

Now take the sets in their algebraic notation, and represent them both as a list of numbers (as a logical sequence or just a couple of examples), and as a "description" of what they are:

\begin{itemize}
	\item $\{3n + 2 \; | \; n \in \mathbb{N} \}$
	\item $\{3k + 2 \; | \; k \in \mathbb{Z} \}$
	\item $\{ 2^i \; | \; i \in [[0, 10]] \}$
	\item $\{ (x, y) \in \mathbb{R}^2 \; | \; x^2 + y^2 = 1 \}$
	\item $\{ x \in \mathbb{R} \; | \; -2 \leq x \leq 2 \}$
	\item $\{ (m, n, p) \in \mathbb{N}^3 \; | \; m + n + p = 10 \}$
\end{itemize}

\newpage
\part{Notes}
\chapter*{Chapter 1, Section 1}

Go check out the extra exercises on set notation.
\chapter*{Chapter 1, Section 2}

\section*{On injections and surjections}

\subsection*{Injections}

Injections (which aren't also surjections) have multiple left-inverses (post-inverses). Eg:

$$A = \{ a, b    \}$$
$$B = \{ 1, 2, 3 \}$$
$$f : A \to B = \{ (a, 2), (b, 3) \} \text{, injective}$$

$$g_1 : B \to A = \{ (1, a), (2, a), (3, b) \}$$
$$g_2 : B \to A = \{ (1, b), (2, a), (3, b) \}$$

$$g_1 \circ f = g_2 \circ f = id_A$$

It is precisely the free element with no antecedent in $B$ (here, $1$) which leaves room for multiple choices, but doesn't affect the overall inversion process.


\subsection*{Surjections}

Surjections (which aren't also injections) have multiple right-inverses (pre-inverses), called sections.



$$B = \{ 1, 2, 3 \}$$
$$A = \{ a, b    \}$$
$$f : B \to A = \{ (1, a), (2, a), (3, b) \} \text{, surjective}$$

$$g_1 : A \to B = \{(a, 1), (b, 3) \}$$
$$g_2 : A \to B = \{(a, 2), (b, 3) \}$$

$$f \circ g_1 = f \circ g_2 = id_A$$

It is precisely the fact that there are multiple elements that map to the same element (here, $1$ and $2$ to $a$) which leaves room for multiple choices, but doesn't affect the overall inversion process.



\subsection*{Cancellations}

Function Cancellation Lemma: If a composition of functions cancels out, then the first of the pair is an injection, and the second of the pair is a surjection. Algebraically:
$$
\forall A, B \in Obj(\textbf{Set}),
f \in (A \to B), g \in (B \to A), \;
	g \circ f = id_A
\Rightarrow
	\begin{cases}
		f \text{ is injective} \\
		g \text{ is surjective}
	\end{cases}
$$
Corollary 1: any post-inverse of an injection is a surjection.
Corollary 2: any pre-inverse of a surjection is an injection.

Proof: Let be 
$$A, B \in Obj(\textbf{Set}), f \in (A \to B), g \in (B \to A), \; g \circ f = id_A$$

a) Suppose $f$ is not an injection. This means:
$$\exists x, y \in B, \; x \neq y \text{ and } g(x) = g(y)$$
However, with such an $f$, we have:
$$g(x) = g(y) \Rightarrow f(g(x)) = f(g(y)) = id_A(x) = id_A(y) = x = y$$
This means that $f$ is an injection. Contradiction.
Conclusion: in this context, $f$ must be an injection.

b) Suppose $g$ is not a surjection. This means:
$$\exists a \in A, \; a \notin g(B)$$
Since $g \circ f = id_A$, that means that $g(f(a)) = id_A(a) = a$, this means that $a \in g(B)$. Contradiction.
Conclusion: in this context, $g$ must be a surjection.



\section*{On sections and fibers}

Section example with a tangent bundle.

Consider the cylinder $S^1 \times \mathbb{R}$, and the function $f : S^1 \times \mathbb{R} \to S^1$, the projection onto the circle. The cylinder is itself the space in which one can easily represent maps of $(S^1 \to \mathbb{R})$. Each such map corresponds to a section.

Let be 

$$
\begin{aligned}
g_1 : S^1    & \longrightarrow  S^1 \times \mathbb{R} \\
      \theta & \longmapsto      (\theta, 1)
\end{aligned}
$$


$$
\begin{aligned}
g_2 : S^1    & \longrightarrow  S^1 \times \mathbb{R} \\
      \theta & \longmapsto      (\theta, cos(\theta))
\end{aligned}
$$

We have
$$f \circ g_1 = f \circ g_2 = id_{S^1}$$

(TODO add diagrams for S1xR, g1 and g2)

A fiber is the preimage of a singleton. In the case of $f$ above, for every $q \in S^1$, $f^{-1}({q})$ is the copy of the real line on the cylinder that passes by $q$.

(TODO add diagram)






\section*{Alternative characterization of a bijection}
"$f$ is bijective" $\Leftrightarrow$ ("every element of $B$ has a non-empty fiber" (surjection) and "every fiber is a singleton" (injection))





\section*{On monomorphisms and epimorphisms}

\subsection*{Failing the mono/epi condition}


\subsubsection*{Example of failing the monomorphism definition for a non-injection}

Monomorphism definition:

$$
\text{$f : A \to B$ is a monomorphism}
\; \; \Leftrightarrow \; \; 
\forall Z \in \text{Obj}(\mathcal{C}), \;
\forall g_1, g_2 \in \text{Hom}(Z, A), \;
(f \circ g_1 = f \circ g_2 \Rightarrow g_1 = g_2)
$$



$$A = \{ a, b, c \}$$
$$B = \{ 1, 2    \}$$
$$Z = \{ x, y    \}$$
$$f : A \to B = \{ (a, 1), (b, 1), (c, 2) \} \text{, not injective}$$

$$g_1 : Z \to A = \{ (x, a), (y, c) \}$$
$$g_2 : Z \to A = \{ (x, b), (y, c) \}$$

$$f \circ g_1 = f \circ g_2 = \{(x, 1), (y, 2)\} \in (Z \to B)$$

The multiple choice of elements (here, $a$ and $b$) in $A$ which map to $1$ in $B$ is precisely what allows the overall composition to be equal, even when $g_1 \neq g_2$. This provides insight into a proof of "$f$ is a monomorphism implies that $f$ is an injection". If you suppose that $f$ is a monomorphism and not an injection, you can easily reach a contradiction, since you can use elements like $1$ and $2$ that both map to the same $a$ to construct a counter-example to the implication that defines a monomorphism.



\subsubsection*{Example of failing the epimorphism definition for a non-surjection}

Epimorphism definition:

$$
\text{$f : A \to B$ is an epimorphism}
\; \; \Leftrightarrow \; \; 
\forall Z \in \text{Obj}(\mathcal{C}), \;
\forall g_1, g_2 \in \text{Hom}(B, Z), \;
(g_1 \circ f = g_2 \circ f \Rightarrow g_1 = g_2)
$$



$$g_1 : Z \to A = \{ (x, a), (y, c) \}$$
$$g_2 : Z \to A = \{ (x, b), (y, c) \}$$

$$f \circ g_1 = f \circ g_2 = \{(x, 1), (y, 2)\} \in (Z \to B)$$


$$A = \{ a, b    \}$$
$$B = \{ 1, 2, 3 \}$$
$$Z = \{ x, y    \}$$
$$f : A \to B = \{ (a, 1), (b, 2) \} \text{, not surjective}$$


$$g_1 : B \to Z = \{ (1, x), (2, y), (3, x) \}$$
$$g_2 : B \to Z = \{ (1, x), (2, y), (3, y) \}$$

$$g_1 \circ f = g_2 \circ f = \{(a, x), (b, y)\} \in (A \to Z)$$

The element $3$ in $B$ not being reached by $f$ is precisely that which provides the opportunity to build $g_1 \neq g_2$ such that they compose to the same result with $f$, since the output of $3$ for them doesn't affect the overall composition. This provides insight into a proof of "$f$ is an epimorphism implies that $f$ is a surjection". If you suppose that $f$ is an epimorphism and not a surjection, you can easily reach a contradiction, since you can use elements like $3$ that are not reached by $f$ to construct a counter-example to the implication that defines an epimorphism.



\subsection*{Proofs of mono/inj and epi/surj equivalence}

Let $f : A \to B$.

The parts which are "Injection => Monomorphism" and "Surjection => Epimorphism" both use the respective sided inverses to prove the implication.

The other parts use the following tautology to prove the implication by contradiction. "Suppose that $p$ and $\neg q$, show that it leads to a contradiction".

$$
(p \Rightarrow q)
\Leftrightarrow ((\neg  p) \cup      q )
\Leftrightarrow ( \neg (p  \cap \neg q))
$$


\subsubsection*{Injection => Monomorphism}

Suppose that $f$ is an injection. It thus has post-inverses.

$$\exists g \in (B \to A), g \circ f = id_A$$

From there:

$$
\forall Z \in \text{Obj}(\mathcal{C}), \;
\forall a, b \in \text{Hom}(Z, A),
$$
$$
\begin{array}{ccccc} \\
f \circ a = f \circ b & \Rightarrow &  g \circ (f  \circ a) &=&  g \circ (f  \circ b) \\
                      & =           & (g \circ  f) \circ a  &=& (g \circ  f) \circ b  \\
                      & =           &         id_A \circ a  &=&         id_A \circ b  \\
                      & =           &                    a  &=&                    b
\end{array}
$$

We conclude that $f$ is also a monomorphism.


\subsubsection*{Surjection => Epimorphism}

Suppose that $f$ is a surjection. It thus has pre-inverses.

$$\exists g \in (B \to A), f \circ g = id_B$$

From there:

$$
\forall Z \in \text{Obj}(\mathcal{C}), \;
\forall a, b \in \text{Hom}(B, Z),
$$
$$
\begin{array}{ccccc} \\
a \circ f = b \circ f & \Rightarrow & (a \circ  f) \circ g  &=& (b \circ  f) \circ g  \\
                      & =           &  a \circ (f  \circ g) &=&  b \circ (f  \circ g) \\
                      & =           &  a \circ  id_B        &=&  b \circ  id_B        \\
                      & =           &  a                    &=&  b
\end{array}
$$

We conclude that $f$ is also an epimorphism.


\subsubsection*{Monomorphism => Injection}

Suppose that $f$ is a monomorphism.

$$
\forall Z \in \text{Obj}(\mathcal{C}), \;
\forall g_1, g_2 \in \text{Hom}(Z, A), \;
f \circ g_1 = f \circ g_2 \Rightarrow g_1 = g_2
$$

Suppose now that $f$ is not an injection. Algebraically, this means that:

$$\exists (x, y) \in A^2, \text{ such that } x \neq y \text{ and } f(x) = f(y)$$

We can construct $g_1$ and $g_2$ such that $f \circ g_1 = f \circ g_2$ but $g_1 \neq g_2$, using such a pair $(x, y)$. Thereby, we prove that $f$ is not an monomorphism and arrive at a contradiction.

(If $Z$ is the empty set, being initial in $\mathbf{Set}$, there is only 1 map into $A$ (the empty map) and $a = b$ always hold. Therefore, any counterexample to the epimorphism definition needs to have at least 1 element.)

Let $Z = \{a\}$.

$$g_1(a) = x$$
$$g_2(a) = y$$

Clearly, $g_1 \neq g_2$. However, we also have:

$$
f(g_1(a)) = f(x) = f(y) = f(g_2(a)) \Rightarrow
f \circ g_1 = f \circ g_2
$$

This means that $f$ is not a monomorphism: contradiction.

Conclusion: $f$ is an injection.


\subsubsection*{Epimorphism => Surjection}

Suppose that $f$ is an epimorphism.

$$
\forall Z \in \text{Obj}(\mathcal{C}), \;
\forall g_1, g_2 \in \text{Hom}(B, Z), \;
g_1 \circ f = g_2 \circ f \Rightarrow g_1 = g_2
$$

Suppose now that $f$ isn't a surjection. Algebraically, this means that:

$$\exists x \in B, x \notin f(A)$$

We can construct $g_1$ and $g_2$ such that $g_1 \circ f = g_2 \circ f$ but $g_1 \neq g_2$, using such an $x \notin f(A)$. Thereby, we prove that $f$ is not an epimorphism and arrive at a contradiction.

(If $Z$ is the singleton set, being terminal in $\mathbf{Set}$, there is only 1 map into $Z$ and $a = b$ always hold. Therefore, any counterexample to the epimorphism definition needs to have at least 2 elements. We will however use a 3-element set, since it makes things more intuitive and pedagogical.)

Let $Z = \{a, b, c\}$.

$$
g_1 =
\begin{cases}
	\forall x    \in f(A), g_1(x) = a \\
	\forall x \notin f(A), g_1(x) = b
\end{cases}
$$

$$
g_2 =
\begin{cases}
	\forall x    \in f(A), g_2(x) = a \\
	\forall x \notin f(A), g_2(x) = c
\end{cases}
$$

Clearly, $g_1 \neq g_2$. However, since $A$ is the domain of $f$, of $g_1 \circ f$, and of $g_2 \circ f$, we have:

$$
g_1 \circ f = g_2 \circ f = (x \mapsto a) \in (A \to Z)
$$

This means that $f$ is not an epimorphism: contradiction.

Conclusion: $f$ is a surjection.
\chapter*{Chapter 1, Section 3}

\section*{On terminal and initial objects in \textbf{Set}}

If $\empty$ is initial and $\{ \star \}$ is terminal, it is because a function in $Set$ (in categorical terms) must always have an output for every input. Ie, in category theory, all functions are maps ("applications").

Said algebraically:

$$
\forall A, B \in \text{Obj}(\bold{Set}), \;
\forall a \in A, \;
\forall f \in \text{Hom}(A, B), \;
\exists f(a) \in B
$$

In the case of $\empty$ as the input set, and there is only one function $f: \empty \to Z$ for any $Z$: $f$ is the empty mapping. But any $Z \to \empty$ (expept $\empty \to \empty$) contains no mapping (since we'd necessarily be ignoring at least one element of $Z$).

Similarly, in the case of the (unique up-to-isomorphism) singleton set $\{ \star \}$ as the output, you'd have multiple functions (precisely $2^{|Z|}$) into it, if you could ignore some elements of the input set. However, if all elements of the input set are required, that leaves you with only one function possible from $Z \to \{ \star \}$: the function which maps all elements to $\star$.




\section*{Restrictions and extensions of functions, and its consequences on a function's nature}

8 possibilities to study, based on the following binary dichotomies:
\begin{itemize}
	\item injection or surjection
	\item enlarging or restricting
	\item domain or codomain
\end{itemize}

Note that enlarging the domain sometimes implies enlarging the codomain, and restricting the codomain sometimes implies restricting the domain.

Legend: Yes, No, Depends

\begin{tabular}{c c c c c}
			& enlarge dom	& restrict dom	& enlarge cod	& restrict cod \\
injection	& D				& Y				& Y				& Y            \\
surjection	& Y				& D				& N				& Y
\end{tabular}


Theorems:

A) if $f \in (A \to B), f \text{ injective (resp. surjective)}$, then $\forall Z \subseteq B, \hat{f} \in ((f^{-1}(Z) \subseteq A) \to Z), \hat{f} = f|_{f^{-1}(Z)}$, the restriction of the function to the corresponding smaller codomain, is also an injection (resp. surjection).

B) if $f \in (A \to B), f \text{ injective (resp. surjective)}$, then $\forall Z \supseteq B, \hat{f} \in (A \to Z), \hat{f} = \iota \circ f$ (with the $\iota$ the canonical injection of $b \in B$ into its superset $Z$), is also an injection (resp. is never a surjection).

C) if $f \in (A \to B), f \text{ injective}$, then $\forall Z \subseteq A, \hat{f} \in (Z \to B), f = \iota_{(Z \to A)} \circ \hat{f}$, we have that $\hat{f}$ is also an injection. However, one can construct $Z \subseteq A$ such that $f$ stops being a surjection.

D) if $f \in (A \to B), f \text{ surjective}$, then $\forall Z \supseteq A, \hat{f} \in (Z \to (B \cup f(Z))), f = \iota_{(Z \to A)} \circ \hat{f}$, we have that $\hat{f}$ is also a surjection. However, one can construct $Z \subseteq A$ such that $f$ stops being a injection.

Proof: TODO

\newpage

\end{document}
